% Introduction. In this chapter we outline the problem of global optimization and some of the considerations we have. We explain the issue of vast configuration spaces and expensive black-boxes.
\chapter{Introduction}
\label{chap:introduction}
This project concerns itself with the problem of global optimization of expensive back-box functions in vast configuration spaces. The combination of these two properties, lead to a problem, that is exceedingly difficult to solve. The expensiveness of the black-box functions, means that it is not feasible to evaluate it frequently, and the vastness of the configuration space requires us to limit our search to sections of the configuration space.  The flavor that we will be studying, is the search for the optimal configuration of atomic clusters, which is usually the configuration with the lowest energy.
\\
In the first part of the project, we will introduce the concept of a "surrogate model". The purpose of this model, is to limit the amount of times we need to evaluate the black-box function. We will then introduce a variety of search methods, that we use to probe the surrogate model. These will be benchmarked against one another, on a simple 1d-function. In the second section, we apply these on a real physical system, using the AGOX-framework develop by Hammer and {}. In the final section, we evaluate the results of our search in the context of Hückle theory, which is a simple method that can be used to predict the energy of an atomic cluster. 
\section*{Complexity of atomic clusters}
\label{sec:complexity_of_atomic_clusters}
Given some atomic cluster of $N$ atoms, we wish to find the configuration of the atoms, that minimizes the energy of the system. We can immediately see, that the dimensionality of the system is equal to $3N$, corresponding to the possible directions each atom can move in. The dimensionality of the configuration space, is thereby a quickly growing function of the number of atoms. In addition to this, calculating the energy of an atomic cluster, is a complicated matter aswell. In order to get a precise answer, that enables us to distinguish between similar configurations, advanced methods like \textit{density functional theory}. These methods have an enormous time complexity (source?) and can take multiple minutes, to calculate energy of even simple systems on a regular computer. In order to overcome this, we will now introduce \textit{surrogate models}
