\documentclass[working, oneside]{../../Preambles/tuftebook}
% Import xcolor and define some colors
\usepackage{{xcolor}}
\definecolor{{background}}{{HTML}}{{{background}}}
\definecolor{{foreground}}{{HTML}}{{{foreground}}}
\definecolor{{math}}{{HTML}}{{{color6}}}

%%%%%%%%%%%%%%%%%%%%%%%%%%%%%%%%%%%%%%%% IMPORTS %%%%%%%%%%%%%%%%%%%%%%%%%%%%%%%%%%%%%%%%
\documentclass[11pt,onesize,a4paper,titlepage]{article}

%%%%%%%%%%%%%%% Formatting %%%%%%%%%%%%%%% 
\usepackage[english]{babel}
\usepackage[utf8]{inputenc}
\usepackage{adjustbox}
\usepackage{geometry} % Margins
\usepackage{sectsty} % Custom Sections

%%%%%%%%%%%%%%% Font %%%%%%%%%%%%%%% 
\usepackage{Archivo}
\usepackage[T1]{fontenc}
\sffamily

%%%%%%%%%%%%%%% Graphics %%%%%%%%%%%%%%% 
\usepackage{fontawesome5} % Icons
\usepackage{graphicx} % Images
\usepackage[most]{tcolorbox} % Color Box
\usepackage{xcolor} % Colors
\usepackage{tikz} % For Drawing Shapes
%%%\usepackage{emoji} % For flags
\tcbuselibrary{breakable}
%%%\usepackage{academicons}

%%%%%%%%%%%%%%% Miscelanous %%%%%%%%%%%%%%% 
\usepackage{lipsum} % Lorem Ipsum
\usepackage{hyperref} % For Hyperlinks

%%%%%%%%%%%%%%% Colors %%%%%%%%%%%%%%% 
\definecolor{title}{HTML}{b5bff5} % Color of the title
\definecolor{bars}{HTML}{889af0} % Color of the title
\definecolor{backdrop}{HTML}{f2f2f2} % Color of the side column
\definecolor{lightgray}{HTML}{dfdfdf} % Color for the skill bars

%%% TU green: #639a00
%%% TU gray: #e6e6e6
%\definecolor{title}{HTML}{639a00} % Color of the title TU
%\definecolor{bars}{HTML}{889af0} % Color of the title TU

% \definecolor{backdrop}{HTML}{f2f2f2} % Color of the side column
\definecolor{backdrop}{HTML}{e6e6e6} % Color of the side column

\definecolor{subtitle}{HTML}{606060} % 


%%%%%%%%%%%%%%% Section Format %%%%%%%%%%%%%%% 
\sectionfont{                     
    \LARGE % Font size
    \sectionrule{0pt}{0pt}{-8pt}{1pt} % Rule under Section name
}

\subsectionfont{
    \Large % Font size
    \fontfamily{phv}\selectfont % Font family
    %\sectionrule{0pt}{0pt}{-8pt}{1pt} % Rule under Subsection name
    \sectionrule{5pt}{0pt}{0pt}{0pt} % Rule under Subsection name
}

%%%%%%%%%%%%%%% Margins and Headers %%%%%%%%%%%%%%%
\geometry{
  a4paper,
  left=7mm,
  right=7mm,
  bottom=10mm,
  top=10mm
}

\pagestyle{empty} % Empty Headers

\usepackage{marvosym}

% \renewcommand\qedsymbol{\CoffeeCup}

\usepackage{changepage}

\newenvironment{subexercise}[1]{%
    \begin{mdframed}[linewidth=0.5pt, linecolor=foreground, backgroundcolor=background, leftmargin=0cm, innerleftmargin=1em, innertopmargin=0pt, innerbottommargin=0pt, innerrightmargin=0pt, topline=false, rightline=false, bottomline=false]
    \par\noindent\textcolor{foreground}{\textbf{#1.}}\hspace{1em}\ignorespaces
}{%
    \par\addvspace{\baselineskip}\end{mdframed}\ignorespacesafterend
}
\newenvironment{solution}{%
    % \par\addvspace{\baselineskip}\noindent\makebox[\textwidth]{\textcolor{foreground}{\textbullet\hspace{1em}\textbullet\hspace{1em}\textbullet}}\par\addvspace{\baselineskip}
    \begin{mdframed}[linewidth=0.5pt, linecolor=foreground, backgroundcolor=background, rightmargin=0cm, innerleftmargin=0cm, innertopmargin=0pt, innerbottommargin=0pt, innerrightmargin=1em, topline=false, leftline=false, bottomline=false]
    \par\noindent\textcolor{foreground}{\textit{Solution.}}\hspace{1em}\ignorespaces
}{%
    \par\addvspace{\baselineskip}\noindent\hfill\textcolor{foreground}{\Coffeecup}\par\addvspace{\baselineskip}\end{mdframed}\ignorespacesafterend
}
% Exercise environment

\declaretheoremstyle[
    name= \textcolor{foreground}{Exercise},
    postheadspace = \newline,
    bodyfont = \normalfont\color{foreground},
    postheadhook={\textcolor{math}{\rule[.4ex]{\linewidth}{0.5pt}}\\},
    % numberwithin=chapter,
    mdframed={
        backgroundcolor = background,
        linecolor = foreground,
        linewidth = 0.5pt,
        rightline =  true,
        topline = true,
        bottomline = true,
        skipabove=20pt,
        skipbelow=20pt,
        innerleftmargin=15pt,
        innertopmargin=10pt,
        innerrightmargin=15pt,
        innerbottommargin=10pt}
    ]{exercise}
\declaretheorem[style=exercise,numbered=no]{exercise}

% \etocsetlevel{exercise}{2}

% \AtEndEnvironment{exercise}{%
%   \etoctoccontentsline{exercise}{\protect\numberline{\theexercise}}%
% }%
% \etocsetstyle{exercise}
% {}
% {}
% % this will be rendered like a non-numbered section, but we could have used
% % \numberline here also
% {\etocsavedsectiontocline{Exercise \etocnumber}{\etocpage}}
%     {}

% theorem environment

\declaretheoremstyle[
    name= \textcolor{foreground}{Theorem},
    postheadspace = \newline,
    bodyfont = \normalfont\color{foreground},
    postheadhook={\textcolor{math}{\rule[.4ex]{\linewidth}{1pt}}\\},
    mdframed={
        backgroundcolor = background,
        linecolor = foreground,
        linewidth = 1pt,
        rightline =  true,
        topline = true,
        bottomline = true,
        skipabove=20pt,
        skipbelow=20pt,
        innerleftmargin=15pt,
        innertopmargin=10pt,
        innerrightmargin=15pt,
        innerbottommargin=10pt}
    ]{theorem}
\declaretheorem[style=theorem,numbered=yes]{theorem}

\declaretheoremstyle[
    name= \textcolor{foreground}{Definition},
    postheadspace = \newline,
    bodyfont = \normalfont\color{foreground},
    postheadhook={\textcolor{math}{\rule[.4ex]{\linewidth}{1pt}}\\},
    mdframed={
        backgroundcolor = background,
        linecolor = foreground,
        linewidth = 1pt,
        rightline =  true,
        topline = true,
        bottomline = true,
        skipabove=20pt,
        skipbelow=20pt,
        innerleftmargin=15pt,
        innertopmargin=10pt,
        innerrightmargin=15pt,
        innerbottommargin=10pt}
    ]{definition}
\declaretheorem[style=definition,numbered=yes]{definition}
% Example environment

\declaretheoremstyle[
name= \quad \underline{Proof:},
     headfont = \bfseries\sffamily,
     postheadspace = \newline,
     % notebraces = \bfseries{(}{)a},
     headpunct = {},
     bodyfont = ,
     postheadhook={\textcolor{foreground}{\rule[0.4ex]{\linewidth}{0pt}}\\},
     qed=\qedsymbol,
    % spacebelow = 10pt,
    mdframed={
  backgroundcolor = background,
  linecolor = foreground,
  linewidth = 1pt,
  skipabove=10pt,
  skipbelow=10pt,
  rightline = false,
  topline = false,
  leftline = false,
  bottomline = false,
  innerleftmargin=15pt,
  innertopmargin=15pt,
  innerrightmargin=15pt,
  innerbottommargin=15pt}
]{pro}
    % \declaretheorem[style=pro,numbered=no]{Proof}

\declaretheoremstyle[
name= \quad \underline{\textcolor{foreground}{Example}},
     headfont = \bfseries\sffamily,
     postheadspace = \newline,
     % notebraces = \bfseries{(}{)a},
     headpunct = {},
     bodyfont = \normalfont\color{foreground},
     postheadhook={\textcolor{foreground}{\rule[0.4ex]{\linewidth}{0pt}}\\},
     % spacebelow = 10pt,
    mdframed={
  backgroundcolor = background,
  linecolor = foreground,
  linewidth = 1pt,
  skipabove=10pt,
  skipbelow=10pt,
  rightline = false,
  topline = false,
  leftline = false,
  bottomline = false,
  innerleftmargin=15pt,
  innertopmargin=15pt,
  innerrightmargin=15pt,
  innerbottommargin=15pt}
]{ex}
\declaretheorem[style=ex,numbered=no]{example}

\declaretheoremstyle[
     name=,
     headfont = \bfseries\sffamily,
     notebraces = \bfseries{},
     headpunct = { -},
     bodyfont = \color{foreground}\normalfont,
     % postheadhook={\textcolor{black}{\rule[.4ex]{\linewidth}{0.2pt}}\\},
    % spacebelow = 10pt,
    mdframed={
  backgroundcolor = background,
  linecolor = foreground,
  linewidth = 1pt,
  skipabove=0pt,
  skipbelow=0pt,
  innerleftmargin=10pt,
  innertopmargin=10pt,
  innerrightmargin=10pt,
  innerbottommargin=10pt,
  rightline = false,
  topline = false,
  leftline = false,
  bottomline = true}
]{subexercise}
% \declaretheorem[style=subexercise,numbered=no]{subexercise}

\declaretheoremstyle[
     name= \color{losning}Løsning,
     headfont = \bfseries\sffamily,
     notebraces = \bfseries{},
     postheadspace = \newline,
     headpunct = {:},
     bodyfont = \normalfont,
     % qed = ,
     % postheadhook={\textcolor{black}{\rule[.4ex]{\linewidth}{0.2pt}}\\},
    % spacebelow = 10pt,
    mdframed={
  backgroundcolor = background,
  linecolor = losning!75,
  linewidth = 1pt,
  skipabove=0pt,
  skipbelow=10pt,
  innerleftmargin=10pt,
  innertopmargin=10pt,
  innerrightmargin=10pt,
  innerbottommargin=10pt,
  leftline = false,
  rightline = true,
  topline = false,
  bottomline = true}
]{solution}

\newenvironment{SimpleBox}[1]{%
  \begin{mdframed}%
    \noindent\textbf{#1}\\[1ex]
}{%
  \end{mdframed}%
}


\begin{document}
\let\cleardoublepage\clearpage
\thispagestyle{fancy}
\chapter{Opgave 3 - Klein-Gordon Field in Space-Time}
\begin{exercise}[3.1]
The previous problem showed how to quantize the Klein-Gordon field in the 
Schrödinger picture of quantum mechanics where the operators are independent 
of time while the state vectors carry all the time-dependence. Here we will 
consider the (equivalent) Heisenberg picture where the state vectors are 
time-independent and the operators carry the time-dependence. The definition 
of an operator in the Heisenberg picture is straightforward
\[
\mathcal{O}_H(x) = \mathcal{O}_H(x,t) = e^{iHt} \mathcal{O}_S(x) e^{-iHt}, \tag{30}
.\]
\color{foreground} 
where $\mathcal{O}_S(x)$ is an operator in the Schrödinger picture and $H$ is the 
Hamiltonian operator which we assume has no explicit time-dependence in this 
problem. Assume that $\mathcal{O}$ does not have any explicit dependence on time $t$. 
Derive the Heisenberg equation of motion - Klein-Gordon Field in Space-Time
\[
i \frac{\partial }{\partial t} \mathcal{O}_H = \left[ \mathcal{O}_H, H \right] 
.\] 
\end{exercise}
\begin{solution}
\begin{align*}
    i\frac{\partial}{\partial t} \left( e^{i\hat{H}t} \hat{O}(x) e^{-i\hat{H}t} \right) &= i\hat{H}e^{i\hat{H}t}\hat{O}\left( x \right) e^{-i\hat{H}t} +e^{i\hat{H}t}\hat{O}\left( x \right)\left( -i\hat{H} \right)  e^{-i\hat{H}t}  \\
&= -i \left( \hat{H} \hat{O}_H - \hat{O}_H \hat{H} \right) \\
&= [\hat{O}_H, \hat{H}]
\end{align*}
\end{solution}
\begin{exercise}[3.2]
The quantized version of the Hamiltonian for the Klein-Gordon field follows 
from Eq. (8) above and is:
\[
H = \int d^3x \left( \pi(x, t)^\dagger \pi(x, t) 
+ \nabla \phi(x, t)^\dagger \cdot \nabla \phi(x, t) 
+ m^2 \phi(x, t)^\dagger \phi(x, t) \right). \tag{32}
\]

Calculate \( [\phi(x, t), H] \) and show that:
\[
i \frac{\partial}{\partial t} \phi(x, t) = i \pi(x, t)^\dagger. \tag{33}
\]

You will need the equal-time commutator in Eq. (11) and the fact that all 
combinations like \( [\phi, \pi^\dagger] = 0 \) vanish.

\end{exercise}
\begin{solution}
Calculate,
\[
[\phi(x, t), H]
\]

where
\[
H = \int d^3x \left( \pi^\dagger \pi + \nabla \phi^\dagger \cdot \nabla \phi + m^2 \phi^\dagger \phi \right)
\]

Step-by-step:
\[
[\phi, H] = \int d^3x \left( [\phi, \pi^\dagger \pi] + [\phi, \nabla \phi^\dagger \cdot \nabla \phi] + [\phi, m^2 \phi^\dagger \phi] \right)
\]

The second and third terms are zero:
\[
[\phi, H] = \int d^3x [\phi, \pi^\dagger \pi] = \int d^3x \left( \pi^\dagger [\phi, \pi] + [\phi, \pi^\dagger] \pi \right)
\]

Using the canonical commutation relation \( [\phi(x), \pi(x')] = i \delta^3(x - x') \), we get:
\[
[\phi, H] = \int d^3x \pi^\dagger i \delta^3(x - x') = i \pi^\dagger(x', t)
\]

We also know that:
\[
i \frac{\partial}{\partial t} \phi(x, t) = [\phi(x, t), H] = i \pi^\dagger(x', t)
\]
\end{solution}
\begin{exercise}[3.3]
3) By calculating the commutator \( [\pi(x, t), H] \), show that:
\[
i \frac{\partial}{\partial t} \pi(x, t) = -i \left( -\nabla^2 + m^2 \right) \phi(x, t)^\dagger. \tag{34}
\]

(Hint: You will need to do partial integration and throw away a boundary term 
which we assume vanishes at infinity.)
\end{exercise}
\begin{solution}
We do a similar calculation as the one before, but since the terms with gradient dont commute with $\pi$, we do partial integration to collect the gradients $\phi ^\dagger$, which we then prove commutes with $\pi$.
\end{solution}
\begin{exercise}[3.4]
Use the results of 2) and 3) to show that the field operator $\phi(x,t)$ obeys the Klein-Gordon equation.
\end{exercise}
\begin{solution}
\begin{align*}
-i\frac{\partial^2}{\partial t^2} \phi(x,t) &= i \frac{\partial}{\partial t} \pi(x,t) = -i(-\nabla^2 + m^2)\phi(x,t), \\
\frac{\partial^2}{\partial t^2} \phi(x,t) - (-\nabla^2 + m^2)\phi(x,t) &= 0, \\
\left( \frac{\partial^2}{\partial t^2} - \nabla^2 + m^2 \right)\phi(x,t) &= 0, \\
\left( \partial_\mu \partial^\mu + m^2 \right)\phi(x,t) &= 0.
\end{align*}
\end{solution}
\begin{exercise}[3.5]
Consider now the expansion of the field operator in modes, i.e.
$$\phi(x) = \int \frac{d^3p}{(2\pi)^3}\frac{1}{\sqrt{2\omega_p}}(a_pe^{ip\cdot x} + c_p^\dagger e^{-ip\cdot x}).$$
We would now like to use the creation and annihilation operators for the modes, $a_p$, $a_p^\dagger$, $c_p$, $c_p^\dagger$, to obtain the time-dependence explicitly. Since in the expansion they are the only quantities that are operators, all we need is to determine how they evolve in time in the Heisenberg picture. \\
Use the commutator relations like those in Eq. (28) to deduce the relations

$$H^na_p = a_p(H - \omega_p)^n$$

for any integer $n$. (Hint: Start with $n = 1$.) Deduce an analogous relation for $a_p^\dagger$ where the minus becomes a plus on the right-hand side. Argue that identical relations can be deduced for the $c_p$ and $c_p^\dagger$ operators.
\end{exercise}
\begin{solution}

We shall use induction. Let’s start by showing \( n = 1 \):

\begin{align*}
H a_p &= a_p H - \omega_p a_p, \\
&= a_p (H - \omega_p),
\end{align*}

where I have used the commutator.

We now assume that it holds for \( n \) i.e. $H^na_p = a_p \left( H - \omega_p \right)^{n} $, and want to show that this implies it holds for \( n+1 \):

\begin{align*}
H^{n+1} a_p &= H \left( H^n a_p \right), \\
&= H \left( a_p (H - \omega_p)^n \right), \\
&= \left( H a_p \right) (H - \omega_p)^n, \\
&= a_p (H - \omega_p) (H - \omega_p)^n, \\
&= a_p (H - \omega_p)^{n+1}.
\end{align*}

Trivial to show that:

\[
H^n a_p^\dagger = a_p^\dagger (H + \omega_p)^n.
\]
\end{solution}
\begin{exercise}[3.6]
Use the result of 5) to show that,
\[
e^{iHt}a_{p}e^{iHt}=a_{p}e^{-i\omega_p}
.\] 
and similar equations for $a_{p}^\dagger$ and the $c$s.
\end{exercise}
\begin{solution}
We write this as an infinite series:
\begin{align*}
e^{iHt} a_p e^{-iHt} &= \sum_{n=0}^\infty \frac{(iHt)^n}{n!} a_p \sum_{n=0}^\infty \frac{(-iHt)^n}{n!}, \\
&= \sum_{n=0}^\infty \frac{(iHt)^n}{n!} H^n a_p \sum_{n=0}^\infty \frac{(-iHt)^n}{n!}, \\
&= a_p \sum_{n=0}^\infty \frac{(it)^n}{n!} (H - \omega_p)^n \sum_{n=0}^\infty \frac{(-iHt)^n}{n!}, \\
&= a_p e^{i(H - \omega_p)t} e^{-iHt}, \\
&= a_p e^{-i\omega_p t}.
\end{align*}

Trivial to show for \( a_p^\dagger \).
\end{solution}

\begin{exercise}[7]
Show that the field operator in the Heisenberg picture can be written in the elegant form
\begin{align*}
    \phi(\mathbf{x}, t) = \int \frac{d^3p}{(2\pi)^3} \frac{1}{\sqrt{2\omega_p}} \left( a_p e^{-ip \cdot x} + c_p^\dagger e^{ip \cdot x} \right), \tag{38}
\end{align*}
where \( p = p^\mu = (\omega_p, \mathbf{p}) \), \( x = x^\mu = (t, \mathbf{x}) \), and \( p \cdot x \) denotes the contraction of the two four-vectors, i.e. \( p \cdot x = p^\mu x_\mu \).

Notice how the field operator in the Heisenberg picture in Eq. (38) is an expansion in modes that correspond to the solutions of the free Klein-Gordon equation (plane waves with argument \( p \cdot x \)). This is a very important result when doing perturbation calculations in quantum field theories.
\end{exercise}
\begin{solution}
\begin{align*}
\phi(x, t) = \int \frac{d^3p}{(2\pi)^3 \sqrt{2\omega_p}} \left( a_p e^{-i p \cdot x} + c_p^\dagger e^{i p \cdot x} \right)
\end{align*}
where $p = p^\mu = (\omega_p, \vec{p})$, $x = x^\mu = (t, \vec{x})$ and $p \cdot x$ denotes the contraction.

We just multiply it in on both sides:
\begin{align*}
\phi(x, t) = e^{iHt} \phi(x) e^{-iHt} = \int \frac{d^3p}{(2\pi)^3 \sqrt{2\omega_p}} e^{iHt} \left( a_p e^{i \vec{p} \cdot \vec{x}} + c_p^\dagger e^{-i \vec{p} \cdot \vec{x}} \right) e^{-iHt}
\end{align*}
Each of these hug $a_p$ and $c_p^\dagger$ respectively, so they produce energy/time terms.
\begin{align*}
= \int \frac{d^3p}{(2\pi)^3 \sqrt{2\omega_p}} \left( a_p e^{-i \omega_p t} e^{i \vec{p} \cdot \vec{x}} + c_p^\dagger e^{i \omega_p t} e^{-i \vec{p} \cdot \vec{x}} \right)
\end{align*}
Note $p^\mu x_\mu = (\omega_p t, -(\vec{p} \cdot \vec{x}))$ (since there is a metric tensor hidden)
\begin{align*}
= \int \frac{d^3p}{(2\pi)^3 \sqrt{2\omega_p}} \left( a_p e^{-i p^\mu x_\mu} + c_p^\dagger e^{i p^\mu x_\mu} \right)
\end{align*}
\end{solution}
\begin{exercise}[8]
The energy \( \omega_p = \sqrt{\mathbf{p}^2 + m^2} > 0 \) is positive by definition. Argue that if the exponential factors in Eq. (38) are interpreted as single-particle wave functions, they would correspond to states with positive and negative energies respectively. (Hint: Consider the time-evolution (phase) factor for a stationary state in the Schrödinger equation.)
\end{exercise}
\begin{solution}
Lets interpret the wave functions as single particle wave functions
\begin{align*}
\Psi(x,t) &= e^{- ip^{\mu }x_{\mu}} \\
&= e^{i (\vec{p} \cdot \vec{x})} e^{-i \omega_p t} \\
&= \Psi(x) e^{-i \omega_p t}.
\end{align*}
Then if we interpret $\omega_p$ as an energy, this corresponds to a positive energy solution.

We can see this by use of the Schrödinger equation
\begin{align*}
i \frac{\partial}{\partial t} \Psi(x,t) &= i \frac{\partial}{\partial t} \Psi(x) e^{-i \omega_p t} \\
&= \omega_p \Psi(x) e^{-i \omega_p t} \\
&= \omega_p \Psi(x,t)
\end{align*}
$\omega_p > 0$
\end{solution}
\begin{exercise}[9]
Combine the notion of particles and anti-particles with the positive and negative energy solutions in the expansion of the field operator, and argue (following Richard Feynman) that the destruction of a particle with four-momentum \( p^\mu \) (and thus positive energy) is equivalent to the creation of an anti-particle with four-momentum \( -p^\mu \) (and negative energy), and vice versa.

Feynman pushed the analogy a bit further by stating that negative energy solutions corresponds to positive energy anti-particles propagating backwards in time. However, when using the most common momentum-space approach to diagramatic perturbation theory this distinction and interpretation is not important, so in modern presentations this is typically either not discussed or only briefly mentioned.
\end{exercise}

\begin{exercise}[10]
Define the three-momentum operator
\begin{align*}
    \mathbf{P} = -\int d^3x \left[ \pi(\mathbf{x}, t)^\dagger \left( \nabla \phi(\mathbf{x}, t) \right)^\dagger + \left( \nabla \phi(\mathbf{x}, t) \right) \pi(\mathbf{x}, t) \right]. \tag{39}
\end{align*}
The operator has been properly symmetrized since we work with complex fields. Show that
\begin{align*}
    \mathbf{P} = \int \frac{d^3p}{(2\pi)^3} \mathbf{p} \left( a_p^\dagger a_p + c_p^\dagger c_p \right), \tag{40}
\end{align*}
where \( \mathbf{p} \) is the three-momentum.
\end{exercise}
\begin{solution}
Define the three-momentum operator
\begin{align*}
P = \int d^3x \left[ \pi(x,t)^\dagger (\nabla \phi(x,t)) + \nabla \phi(x,t) \pi(x,t) \right]
\end{align*}
We should probably do this in a couple of steps
\begin{align*}
\nabla \phi(x,t) &= \int \frac{d^3p}{(2\pi)^3} \frac{1}{\sqrt{2\omega_p}} i \vec{p} \left( a_p e^{-i p \cdot x} + c_p^\dagger e^{i p \cdot x} \right) \\
\pi(x,t) &= i \int \frac{d^3p}{(2\pi)^3} \sqrt{\frac{\omega_p}{2}} \left( a_p^\dagger e^{i p \cdot x} - c_p e^{-i p \cdot x} \right)
\end{align*}
Alright, now lets compute the right-hand side.
\begin{align*}
-&\int d^3x \int d^3q \int d^3p \frac{1}{2(2\pi)^6} \vec{p} \left( a_p e^{-i p \cdot x} + c_p^\dagger e^{i p \cdot x} \right) \left( a_q^\dagger e^{i q \cdot x} - c_q e^{-i q \cdot x} \right) \\
-&= \int d^3x \int d^3q \int d^3p \frac{i}{2(2\pi)^6} \vec{p} \left( a_p a_q^\dagger e^{i(q-p) \cdot x} - a_p c_q e^{-i(q+p) \cdot x} \right. \\
&\qquad \left. + c_p^\dagger a_q^\dagger e^{i(q+p) \cdot x} - c_p^\dagger c_q e^{i(p-q) \cdot x} \right)
\end{align*}
Now, we need to be a bit careful here as $p, q, x$ are now four vectors. So we shall do the following rewrite before continuing
\begin{align*}
(q-p)^\mu x_\mu = ((w_p - w_q)t - (\vec{q} - \vec{p}) \cdot \vec{x})
\end{align*}
Having done this we can apply the $x$ integral. But lets just start by writing it
\begin{align*}
&= - \int d^3x \int d^3q \int d^3p \frac{1}{2} \frac{1}{(2\pi)^6} \vec{p} \left( a_p a_q^\dagger e^{-i(\vec{q} - \vec{p}) \cdot \vec{x}} e^{i(w_p - w_q)t} \right. \\
&\qquad - a_p c_q e^{i(\vec{q} + \vec{p}) \cdot \vec{x}} e^{-i(w_p + w_q)t} + c_p^\dagger a_q^\dagger e^{-i(\vec{q} + \vec{p}) \cdot \vec{x}} e^{i(w_p + w_q)t} \\
&\qquad \left. - c_p^\dagger c_q e^{-i(\vec{p} - \vec{q}) \cdot \vec{x}} e^{i(w_q - w_p)t} \right)
\end{align*}
We apply the $x$ integral. This yields delta functions, and a factor of $2\pi^{3}$
\begin{align*}
&= - \int d^3p \int d^3q \frac{1}{2} \frac{1}{(2\pi)^3} \vec{p} \left( a_p a_q^\dagger \delta^3(\vec{q} - \vec{p}) e^{i(w_p - w_q)t} - a_p c_q \delta^3(\vec{p} + \vec{q}) e^{-i(w_p + w_q)t} \right. \\
&\qquad \left. + c_p^\dagger a_q^\dagger \delta^3(\vec{p} + \vec{q}) e^{i(w_p + w_q)t} - c_p^\dagger c_q \delta^3(\vec{q} - \vec{p}) e^{i(w_q - w_p)t} \right)
\end{align*}

Now we can resolve the q integral.
\begin{align*}
&= - \int d^3p \frac{1}{2} \frac{1}{(2\pi)^3} \vec{p} \left( a_p a_p^\dagger - a_p c_{-p} e^{-i(w_p + w_p)t} + c_p^\dagger a_{-p}^\dagger e^{i(w_p + w_p)t} - c_p^\dagger c_p \right)
\end{align*}
At this point we cant reduce the RHS anymore, so lets proceed with the LHS
\begin{align*}
&\int d^3x \left[ \pi^\dagger(x,t) \nabla \phi(x,t) \right] \\
&= - \int d^3x \int d^3q' \int d^3p' \frac{1}{2} \frac{1}{(2\pi)^6} \vec{p}' \left[ \left( a_{q'} e^{-i q' \cdot x} - c_{q'}^\dagger e^{i q' \cdot x} \right) \left( a_{p'}^\dagger e^{i p' \cdot x} + c_{p'} e^{-i p' \cdot x} \right) \right] \\
&= - \int d^3x \int d^3q' \int d^3p' \frac{1}{2} \frac{1}{(2\pi)^6} \vec{p}' \left( a_{q'} a_{p'}^\dagger e^{i(q'-p') \cdot x} + a_{q'} c_{p'} e^{-i(q'+p') \cdot x} \right. \\
&\quad \left. - c_{q'}^\dagger a_{p'}^\dagger e^{i(q'+p') \cdot x} - c_{q'}^\dagger c_{p'} e^{i(p'-q') \cdot x} \right)
\end{align*}
We apply the x-integral
\begin{align*}
&= - \int d^3q' \int d^3p' \frac{1}{2}\frac{1}{2\pi^{3}} \vec{p}' \left( a_{q'} a_{p'}^\dagger \delta^3(\vec{q'} - \vec{p}') e^{i(w_{q'} - w_{p'})t} + a_{q'} c_{p'} \delta^3(\vec{q'} + \vec{p}') e^{-i(w_{q'} + w_{p'})t} \right. \\
&\quad \left. - c_{q'}^\dagger a_{p'}^\dagger \delta^3(\vec{q'} + \vec{p}') e^{i(w_{q'} + w_{p'})t} - c_{q'}^\dagger c_{p'} \delta^3(\vec{q'} - \vec{p}') e^{i(w_{p'} - w_{q'})t} \right)
\end{align*}
Resolve the q' integral
\begin{align*}
&= \int \frac{1}{2} \frac{1}{(2\pi)^3} \vec{p}' \left( a_{p'} a_{p'}^\dagger + a_{-p'} c_{p'} e^{-i(w_{p'} + w_{p'})t} - c_{-p'}^\dagger a_{p'}^\dagger e^{i(w_{p'} + w_{p'})t} - c_{p'}^\dagger c_{p'} \right)
\end{align*}
Okay, now we can collect them. We pull the minus sign and $\frac{1}{2} \frac{i}{(2\pi)^3}$ outside.
\begin{align*}
P &= (-1) \frac{1}{2} \frac{i}{(2\pi)^3} \left( \int d^3p \vec{p} (a_p a_p^\dagger - a_{-p} c_p e^{-i(2w_p)t} + c_{-p}^\dagger a_p^\dagger e^{i(2w_p)t} - c_p^\dagger c_p) \right. \\
&\qquad \left. + \int d^3p' \vec{p}' (a_{p'} a_{p'}^\dagger + a_{p'} c_{-p'} e^{-i(2w_{p'})t} + c_{p'}^\dagger a_{-p'}^\dagger e^{i(w_{p'}2t)} - c_{p'}^\dagger c_{p'}) \right)
\end{align*}
Now before finishing up note some stuff.
\begin{align*}
a_p a_p^\dagger = - a_p^\dagger a_p + \delta^3(0)
\end{align*}
So we can switch by flipping the sign for free under the integral
\begin{align*}
\text{Also } a_{-p} c_p = c_{-p} a_p \text{ under the integral}
\end{align*}
We can also by linearity $p \rightarrow p'$ for free. We can therefore see immediately
\begin{align*}
P = \int \frac{d^3p}{(2\pi)^3} \vec{p} (a_p^\dagger a_p + c_p^\dagger c_p)
\end{align*}
\end{solution}
\begin{exercise}[11]
Show that $[P, a_p^\dagger] = p a_p^\dagger$ and $[P, a_p] = -p a_p$. Use these relations to show that
\begin{align*}
P^n a_p^\dagger &= a_p^\dagger (P + p)^n \text{ and }
P^n a_p = a_p (P - p)^n,
\end{align*}
for any integer $n$. Finally, derive the translation identities
\begin{align*}
e^{-i P \cdot x} a_p e^{i P \cdot x} = a_p e^{i p \cdot x} \text{ and }
e^{-i P \cdot x} a_p^\dagger e^{i P \cdot x} = a_p^\dagger e^{-i p \cdot x}.
\end{align*}
Argue that identical relations hold for the $c_p$ and $c_p^\dagger$ operators.
\end{exercise}
\begin{solution}
\begin{align*}
[P, a_p^\dagger] &= \int \frac{d^3p'}{(2\pi)^3} \vec{p}' [(a_{p'}^\dagger a_{p'} + c_{p'}^\dagger c_{p'}), a_p^\dagger] \\
&= \int \frac{d^3p'}{(2\pi)^3} \vec{p}' [a_{p'}^\dagger a_{p'}, a_p^\dagger] \\
&= \int \frac{d^3p'}{(2\pi)^3} \vec{p}' a_{p'}^\dagger \delta^3(p' - p) \\
&= \vec{p} a_p^\dagger
\end{align*}

Show,
\begin{align*}
P^n a_p^\dagger = a_p^\dagger (P + \vec{p})^n
\end{align*}

Show for $n = 1$.
\begin{align*}
P a_p^\dagger &= a_p^\dagger P + \vec{p} a_p^\dagger \\
&= a_p^\dagger (P + \vec{p})
\end{align*}

Assume $n$ show $\Rightarrow n+1$
\begin{align*}
P^{n+1} a_p^\dagger &= P (P^n a_p^\dagger) \\
&= P (a_p^\dagger (P + \vec{p})^n) \\
&= (P a_p^\dagger) (P + \vec{p})^n \\
&= a_p^\dagger (P + \vec{p})^{n+1}
\end{align*}
And for the last equality, we use the Baker-Campbell-Hausdorff lemma
\begin{align*}
e^{-i P \cdot x} a_p^\dagger e^{i P \cdot x} &= \sum_{n=0}^\infty \frac{- [\left( iP\cdot x \right)^{n} , a_p^\dagger] }{n!} 
\end{align*}
Where the numerator is an $n$-times nested commutator,
\[
\left[ \left( iP\cdot x \right) ^{n}, a_{p}^\dagger \right] = a_p^\dagger + [i(P \cdot x), a_p^\dagger] + \frac{ [i(P \cdot x), [i(P \cdot x), a_p^\dagger]] }{2!} + \dots
.\] 
I will use induction to prove the form of the $n$-th term. We start by showing $n=1$
\begin{align*}
[i(P \cdot x), a_p^\dagger] &= -i x \cdot [P, a_p^\dagger] = -i x \cdot p a_p^\dagger
\end{align*}
And now we make an assumption on $n$,
\begin{align*}
[i(P \cdot x)^{n}, a_p^\dagger]= (-i x \cdot p)^n a_p^\dagger
\end{align*}
And prove that that this implies $n+1$.
\begin{align*}
[[i P \cdot x)^{n+1}, a_p^\dagger] &= [i P \cdot x, [i P \cdot x)^n, a_p^\dagger]] \\
&= [i P \cdot x, (ix \cdot p)^n a_p^\dagger] \\
&= (ix \cdot p)^n [i P \cdot x, a_p^\dagger] \\
&= (ix \cdot p)^{n+1} a_p^\dagger
\end{align*}
Insert.
\begin{align*}
\sum_{n=0}^\infty \frac{-[[i P \cdot x)^n, a_p^\dagger]}{n!} &= \sum_{n=0}^\infty \frac{-i x \cdot p}{n!} a_p^\dagger \\
&= a_p^\dagger e^{-ix \cdot p}
\end{align*}
\end{solution}
\begin{exercise}[12]
Define the four-momentum operator $P^\mu = (H, P)$. Using the relations above show that the $\phi$ field can be translated in space and time by
\begin{align*}
\phi(x,t) = e^{i P^\mu x_\mu} \phi(0,0) e^{-i P^\mu x_\mu},
\end{align*}
where $x^\mu = (t, x)$.
\end{exercise}
\begin{solution}
\begin{align*}
e^{i P^\mu x_\mu} \phi(0,0) e^{-i P^\mu x_\mu} &= e^{i P^\mu x_\mu} \int \frac{d^3p}{(2\pi)^3} k (a_p + c_p^\dagger) e^{i P^\mu x_\mu} \\
&= e^{i H t} e^{i P \cdot x} \int \frac{d^3p}{(2\pi)^3} k (a_p + c_p^\dagger) e^{i P \cdot x} e^{i H t}.
\end{align*}
Apply the relations,
\begin{align*}
\int \frac{d^3p}{(2\pi)^3} k (a_p e^{i p \cdot x - i w_p t} + c_p^\dagger e^{-i p \cdot x + i w_p t}) &= \phi(x,t)
\end{align*}
The only thing worth considering here, is why I can move momentum term inside the integral.
\end{solution}
\end{document}
