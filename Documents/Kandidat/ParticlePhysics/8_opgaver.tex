\documentclass[working, oneside]{../../Preambles/tuftebook}
% Import xcolor and define some colors
\usepackage{{xcolor}}
\definecolor{{background}}{{HTML}}{{{background}}}
\definecolor{{foreground}}{{HTML}}{{{foreground}}}
\definecolor{{math}}{{HTML}}{{{color6}}}

%%%%%%%%%%%%%%%%%%%%%%%%%%%%%%%%%%%%%%%% IMPORTS %%%%%%%%%%%%%%%%%%%%%%%%%%%%%%%%%%%%%%%%
\documentclass[11pt,onesize,a4paper,titlepage]{article}

%%%%%%%%%%%%%%% Formatting %%%%%%%%%%%%%%% 
\usepackage[english]{babel}
\usepackage[utf8]{inputenc}
\usepackage{adjustbox}
\usepackage{geometry} % Margins
\usepackage{sectsty} % Custom Sections

%%%%%%%%%%%%%%% Font %%%%%%%%%%%%%%% 
\usepackage{Archivo}
\usepackage[T1]{fontenc}
\sffamily

%%%%%%%%%%%%%%% Graphics %%%%%%%%%%%%%%% 
\usepackage{fontawesome5} % Icons
\usepackage{graphicx} % Images
\usepackage[most]{tcolorbox} % Color Box
\usepackage{xcolor} % Colors
\usepackage{tikz} % For Drawing Shapes
%%%\usepackage{emoji} % For flags
\tcbuselibrary{breakable}
%%%\usepackage{academicons}

%%%%%%%%%%%%%%% Miscelanous %%%%%%%%%%%%%%% 
\usepackage{lipsum} % Lorem Ipsum
\usepackage{hyperref} % For Hyperlinks

%%%%%%%%%%%%%%% Colors %%%%%%%%%%%%%%% 
\definecolor{title}{HTML}{b5bff5} % Color of the title
\definecolor{bars}{HTML}{889af0} % Color of the title
\definecolor{backdrop}{HTML}{f2f2f2} % Color of the side column
\definecolor{lightgray}{HTML}{dfdfdf} % Color for the skill bars

%%% TU green: #639a00
%%% TU gray: #e6e6e6
%\definecolor{title}{HTML}{639a00} % Color of the title TU
%\definecolor{bars}{HTML}{889af0} % Color of the title TU

% \definecolor{backdrop}{HTML}{f2f2f2} % Color of the side column
\definecolor{backdrop}{HTML}{e6e6e6} % Color of the side column

\definecolor{subtitle}{HTML}{606060} % 


%%%%%%%%%%%%%%% Section Format %%%%%%%%%%%%%%% 
\sectionfont{                     
    \LARGE % Font size
    \sectionrule{0pt}{0pt}{-8pt}{1pt} % Rule under Section name
}

\subsectionfont{
    \Large % Font size
    \fontfamily{phv}\selectfont % Font family
    %\sectionrule{0pt}{0pt}{-8pt}{1pt} % Rule under Subsection name
    \sectionrule{5pt}{0pt}{0pt}{0pt} % Rule under Subsection name
}

%%%%%%%%%%%%%%% Margins and Headers %%%%%%%%%%%%%%%
\geometry{
  a4paper,
  left=7mm,
  right=7mm,
  bottom=10mm,
  top=10mm
}

\pagestyle{empty} % Empty Headers

\usepackage{marvosym}

% \renewcommand\qedsymbol{\CoffeeCup}

\usepackage{changepage}

\newenvironment{subexercise}[1]{%
    \begin{mdframed}[linewidth=0.5pt, linecolor=foreground, backgroundcolor=background, leftmargin=0cm, innerleftmargin=1em, innertopmargin=0pt, innerbottommargin=0pt, innerrightmargin=0pt, topline=false, rightline=false, bottomline=false]
    \par\noindent\textcolor{foreground}{\textbf{#1.}}\hspace{1em}\ignorespaces
}{%
    \par\addvspace{\baselineskip}\end{mdframed}\ignorespacesafterend
}
\newenvironment{solution}{%
    % \par\addvspace{\baselineskip}\noindent\makebox[\textwidth]{\textcolor{foreground}{\textbullet\hspace{1em}\textbullet\hspace{1em}\textbullet}}\par\addvspace{\baselineskip}
    \begin{mdframed}[linewidth=0.5pt, linecolor=foreground, backgroundcolor=background, rightmargin=0cm, innerleftmargin=0cm, innertopmargin=0pt, innerbottommargin=0pt, innerrightmargin=1em, topline=false, leftline=false, bottomline=false]
    \par\noindent\textcolor{foreground}{\textit{Solution.}}\hspace{1em}\ignorespaces
}{%
    \par\addvspace{\baselineskip}\noindent\hfill\textcolor{foreground}{\Coffeecup}\par\addvspace{\baselineskip}\end{mdframed}\ignorespacesafterend
}
% Exercise environment

\declaretheoremstyle[
    name= \textcolor{foreground}{Exercise},
    postheadspace = \newline,
    bodyfont = \normalfont\color{foreground},
    postheadhook={\textcolor{math}{\rule[.4ex]{\linewidth}{0.5pt}}\\},
    % numberwithin=chapter,
    mdframed={
        backgroundcolor = background,
        linecolor = foreground,
        linewidth = 0.5pt,
        rightline =  true,
        topline = true,
        bottomline = true,
        skipabove=20pt,
        skipbelow=20pt,
        innerleftmargin=15pt,
        innertopmargin=10pt,
        innerrightmargin=15pt,
        innerbottommargin=10pt}
    ]{exercise}
\declaretheorem[style=exercise,numbered=no]{exercise}

% \etocsetlevel{exercise}{2}

% \AtEndEnvironment{exercise}{%
%   \etoctoccontentsline{exercise}{\protect\numberline{\theexercise}}%
% }%
% \etocsetstyle{exercise}
% {}
% {}
% % this will be rendered like a non-numbered section, but we could have used
% % \numberline here also
% {\etocsavedsectiontocline{Exercise \etocnumber}{\etocpage}}
%     {}

% theorem environment

\declaretheoremstyle[
    name= \textcolor{foreground}{Theorem},
    postheadspace = \newline,
    bodyfont = \normalfont\color{foreground},
    postheadhook={\textcolor{math}{\rule[.4ex]{\linewidth}{1pt}}\\},
    mdframed={
        backgroundcolor = background,
        linecolor = foreground,
        linewidth = 1pt,
        rightline =  true,
        topline = true,
        bottomline = true,
        skipabove=20pt,
        skipbelow=20pt,
        innerleftmargin=15pt,
        innertopmargin=10pt,
        innerrightmargin=15pt,
        innerbottommargin=10pt}
    ]{theorem}
\declaretheorem[style=theorem,numbered=yes]{theorem}

\declaretheoremstyle[
    name= \textcolor{foreground}{Definition},
    postheadspace = \newline,
    bodyfont = \normalfont\color{foreground},
    postheadhook={\textcolor{math}{\rule[.4ex]{\linewidth}{1pt}}\\},
    mdframed={
        backgroundcolor = background,
        linecolor = foreground,
        linewidth = 1pt,
        rightline =  true,
        topline = true,
        bottomline = true,
        skipabove=20pt,
        skipbelow=20pt,
        innerleftmargin=15pt,
        innertopmargin=10pt,
        innerrightmargin=15pt,
        innerbottommargin=10pt}
    ]{definition}
\declaretheorem[style=definition,numbered=yes]{definition}
% Example environment

\declaretheoremstyle[
name= \quad \underline{Proof:},
     headfont = \bfseries\sffamily,
     postheadspace = \newline,
     % notebraces = \bfseries{(}{)a},
     headpunct = {},
     bodyfont = ,
     postheadhook={\textcolor{foreground}{\rule[0.4ex]{\linewidth}{0pt}}\\},
     qed=\qedsymbol,
    % spacebelow = 10pt,
    mdframed={
  backgroundcolor = background,
  linecolor = foreground,
  linewidth = 1pt,
  skipabove=10pt,
  skipbelow=10pt,
  rightline = false,
  topline = false,
  leftline = false,
  bottomline = false,
  innerleftmargin=15pt,
  innertopmargin=15pt,
  innerrightmargin=15pt,
  innerbottommargin=15pt}
]{pro}
    % \declaretheorem[style=pro,numbered=no]{Proof}

\declaretheoremstyle[
name= \quad \underline{\textcolor{foreground}{Example}},
     headfont = \bfseries\sffamily,
     postheadspace = \newline,
     % notebraces = \bfseries{(}{)a},
     headpunct = {},
     bodyfont = \normalfont\color{foreground},
     postheadhook={\textcolor{foreground}{\rule[0.4ex]{\linewidth}{0pt}}\\},
     % spacebelow = 10pt,
    mdframed={
  backgroundcolor = background,
  linecolor = foreground,
  linewidth = 1pt,
  skipabove=10pt,
  skipbelow=10pt,
  rightline = false,
  topline = false,
  leftline = false,
  bottomline = false,
  innerleftmargin=15pt,
  innertopmargin=15pt,
  innerrightmargin=15pt,
  innerbottommargin=15pt}
]{ex}
\declaretheorem[style=ex,numbered=no]{example}

\declaretheoremstyle[
     name=,
     headfont = \bfseries\sffamily,
     notebraces = \bfseries{},
     headpunct = { -},
     bodyfont = \color{foreground}\normalfont,
     % postheadhook={\textcolor{black}{\rule[.4ex]{\linewidth}{0.2pt}}\\},
    % spacebelow = 10pt,
    mdframed={
  backgroundcolor = background,
  linecolor = foreground,
  linewidth = 1pt,
  skipabove=0pt,
  skipbelow=0pt,
  innerleftmargin=10pt,
  innertopmargin=10pt,
  innerrightmargin=10pt,
  innerbottommargin=10pt,
  rightline = false,
  topline = false,
  leftline = false,
  bottomline = true}
]{subexercise}
% \declaretheorem[style=subexercise,numbered=no]{subexercise}

\declaretheoremstyle[
     name= \color{losning}Løsning,
     headfont = \bfseries\sffamily,
     notebraces = \bfseries{},
     postheadspace = \newline,
     headpunct = {:},
     bodyfont = \normalfont,
     % qed = ,
     % postheadhook={\textcolor{black}{\rule[.4ex]{\linewidth}{0.2pt}}\\},
    % spacebelow = 10pt,
    mdframed={
  backgroundcolor = background,
  linecolor = losning!75,
  linewidth = 1pt,
  skipabove=0pt,
  skipbelow=10pt,
  innerleftmargin=10pt,
  innertopmargin=10pt,
  innerrightmargin=10pt,
  innerbottommargin=10pt,
  leftline = false,
  rightline = true,
  topline = false,
  bottomline = true}
]{solution}

\newenvironment{SimpleBox}[1]{%
  \begin{mdframed}%
    \noindent\textbf{#1}\\[1ex]
}{%
  \end{mdframed}%
}


\begin{document}
\let\cleardoublepage\clearpage
\thispagestyle{fancy}
\chapter{8 - Scalar Fields and Causality}

Let us consider an expectation value of two fields in the vacuum
\begin{align*}
D(x, y) = \langle 0 | \phi(x) \phi(y) | 0 \rangle,
\end{align*}
where $x$ and $y$ are four-vectors (we suppress the space-time indices for simplicity). We use the field expansion for a real field, $\phi$, which has the form
\begin{align*}
\phi = \int \frac{d^3 \mathbf{p}}{(2\pi)^3} \frac{1}{\sqrt{2 \omega_\mathbf{p}}} \left[ e^{-ipx} a_\mathbf{p} + e^{ipx} a_\mathbf{p}^\dagger \right],
\end{align*}
where $\omega_\mathbf{p} = \sqrt{\mathbf{p}^2 + m^2}$.

\begin{exercise}[1]
Show that
\begin{align*}
    D(x, y) = D(x - y) = \int \frac{d^3 \mathbf{p}}{(2\pi)^3} \frac{1}{ 2 \omega_\mathbf{p}} e^{-ip(x - y)},
\end{align*}
and argue that this quantity is a Lorentz scalar, i.e., it is invariant under Lorentz transformations.
\end{exercise}
Lets plug it in and see what happens. We can pull the integrals and the constants out,
\begin{align*}
    D\left( x,y \right) &= \int \frac{d^3\mathbf{p}}{\left( 2\pi \right) ^{3}} \frac{1}{\sqrt{ 2\omega_{\mathbf{p}}}} \int \frac{d^3\mathbf{q}}{\left( 2\pi \right) ^{3}} \frac{1}{\sqrt{ 2\omega_{\mathbf{q}}}} \left<0 \right|  \left( e^{-ipx}a_{\mathbf{p}}+e^{ipx}a_{\mathbf{p}}^\dagger \right) \left(  e^{-iqy}a_{\mathbf{q}}+e^{iqy }a_{\mathbf{q}}^\dagger \right)   \left|0 \right>
.\end{align*}
Recall that 
\[
    \left<0 \right|a_{\mathbf{p} }^\dagger = 0 \quad, \quad a_{\mathbf{p}} \left|0 \right = 0 
.\] 
So mulitplying the parenthesis out, we only have a single term that survives
\begin{align*}
    D\left( x,y \right) &= \int \frac{d^3\mathbf{p}}{\left( 2\pi \right) ^{3}} \frac{1}{\sqrt{2\omega_{\mathbf{p}}} } \int \frac{d^3\mathbf{q}}{\left( 2\pi \right) ^{3}} \frac{1}{\sqrt{2\omega_{\mathbf{q}}} } \left<0 \right| a_{\mathbf{p}}a_{\mathbf{q}}^\dagger e^{-i\left( px - qy \right) } \left|0 \right> \\
                        &= \int \frac{d^3\mathbf{p}}{\left( 2\pi \right) ^{3}} \frac{1}{\sqrt{2\omega_{\mathbf{p}}}} \int \frac{d^3\mathbf{q}}{\left( 2\pi \right) ^{3}} \frac{1}{\sqrt{2\omega_{\mathbf{q}}}} \left<0 \right| a_{\mathbf{q}}^\dagger a_{\mathbf{p}} + \delta^{3}\left( p - q \right)   \left|0 \right> e^{-i\left( px - qy \right) }\\
                        &= \int \frac{d^3\mathbf{p}}{\left( 2\pi \right) ^{3}} \frac{1}{\sqrt{ 2\omega_{\mathbf{p}}}} \int \frac{d^3\mathbf{q}}{\left( 2\pi \right) ^{3}} \frac{1}{\sqrt{ 2\omega_{\mathbf{q}}}}  \delta^{3}\left( p - q \right) e^{-i\left( px - qy \right) }\\
 &= \int \frac{d^3\mathbf{p}}{\left( 2\pi \right) ^{3}} \frac{1}{2\omega_{\mathbf{p}}}  e^{-ip\left(x - y \right)}
.\end{align*}
\newpage
\begin{exercise}[2]
Show that
\begin{align*}
\left[ \phi(x), \phi(y) \right] = D(x - y) - D(y - x).
\end{align*}
\end{exercise}
This is just computation, so lets go ahead and do that.
\begin{align*}
    \left[ \phi \left( x \right) ,\phi \left( y \right)  \right] &= 
    \int \frac{d^3\mathbf{p}}{\left( 2\pi \right) ^{3}} \frac{1}{\sqrt{ 2\omega_{\mathbf{p}}}} \int \frac{d^3\mathbf{q}}{\left( 2\pi \right) ^{3}} \frac{1}{\sqrt{ 2\omega_{\mathbf{q}}}} \left[ e^{-ipx}a_{\mathbf{p}}+e^{ipx}a_{\mathbf{p}}^\dagger  ,   e^{-iqy}a_{\mathbf{q}}+e^{iqy }a_{\mathbf{q}}^\dagger \right]\\
    &=\int \frac{d^3\mathbf{p}}{\left( 2\pi \right) ^{3}} \frac{1}{\sqrt{ 2\omega_{\mathbf{p}}}} \int \frac{d^3\mathbf{q}}{\left( 2\pi \right) ^{3}} \frac{1}{\sqrt{ 2\omega_{\mathbf{q}}}} \left(   \left[ e^{-ipx}a_{\mathbf{p}}  ,   e^{iqy }a_{\mathbf{q}}^\dagger \right] + \left[   e^{ipx}a_{\mathbf{p}}^\dagger,e^{-iqy}a_{\mathbf{q}} \right]\right) \\
    &=\int \frac{d^3\mathbf{p}}{\left( 2\pi \right) ^{3}} \frac{1}{\sqrt{ 2\omega_{\mathbf{p}}}} \int \frac{d^3\mathbf{q}}{\left( 2\pi \right) ^{3}} \frac{1}{\sqrt{ 2\omega_{\mathbf{q}}}} \left(   \left[ a_{\mathbf{p}}  ,   a_{\mathbf{q}}^\dagger \right]e^{-i(px -qy)} + \left[   a_{\mathbf{p}}^\dagger,a_{\mathbf{q}} \right]e^{-i(qy-px)}\right)\\ 
    &=\int \frac{d^3\mathbf{p}}{\left( 2\pi \right) ^{3}} \frac{1}{ 2\omega_{\mathbf{p}}} \left(e^{-ip(x -y)} +e^{-ip(y-x)}\right) 
.\end{align*}
\begin{exercise}[3]
Show that if $(x - y)^2 < 0$ then $\left[ \phi(x), \phi(y) \right] = 0$. (Hint: First argue that in this case a Lorentz transformation can be found that takes $(x_0 - y_0)$ to $-(x_0 - y_0) = (y_0 - x_0$).
\end{exercise}
When we spatial separation we can find a reference where the events happen simultaneously. Lets call this transform $L_{0}$. Applying this to the four vector $x-y$, we get
 \[
L_{0}\left( x-y \right) =
\begin{bmatrix}
    0 \\
    \vec{x}' - \vec{y}' \\
\end{bmatrix} = x' -y'
.\] 
Since the lorentz transform is linear,
\[
L_0 \left( y-x \right) = -L_0\left( x- y \right) =
\begin{bmatrix}
    0 \\
    \vec{y}' - \vec{x}' \\
\end{bmatrix} = y' - x'
.\] 
Now, since $D$ is lorentz invariant we have that,
\[
D\left( x-y \right) = D\left( x'-y' \right) = \int \frac{d^3\mathbf{p}}{\left( 2\pi \right) ^3} \frac{1}{2\omega_{\mathbf{p}}}e^{-i\mathbf{p}\left( \vec{x}' - \vec{y}' \right) }
.\] 
Now lets a look at the commutator in the transformed frame,
\begin{align*}
 D\left( x'-y' \right) - D\left( y' - x'\right) = \int \frac{d^3\mathbf{p}}{\left( 2\pi \right) ^3} \frac{1}{2\omega_{\mathbf{p}}}e^{-i\mathbf{p}\left( \vec{x}' - \vec{y}' \right) } - \int \frac{d^3\mathbf{q}}{\left( 2\pi \right) ^3} \frac{1}{2\omega_{\mathbf{q}}}e^{-i\mathbf{q}\left( \vec{y}' - \vec{x}' \right) }
.\end{align*}
We can do a substitution $\mathbf{q} \to \mathbf{-p}$, and noting that $\omega_{-\mathbf{p}} = \omega_{\mathbf{p}}$, we end up with the same integrals with the signs flipped,
\[
 = \int \frac{d^3\mathbf{p}}{\left( 2\pi \right) ^3} \frac{1}{2\omega_{\mathbf{p}}}e^{-i\mathbf{p}\left( \vec{x}' - \vec{y}' \right) } - \int \frac{d^3\mathbf{p}}{\left( 2\pi \right) ^3} \frac{1}{2\omega_{\mathbf{p}}}e^{-i\mathbf{p}\left( \vec{x}' - \vec{y}' \right) }=0
.\] 
\begin{exercise}[4]
Argue that you have now shown that causality is preserved for scalar fields.
\end{exercise}
\end{document}
