\documentclass[working, oneside]{../../../Preambles/tuftebook}
% Import xcolor and define some colors
\usepackage{{xcolor}}
\definecolor{{background}}{{HTML}}{{{background}}}
\definecolor{{foreground}}{{HTML}}{{{foreground}}}
\definecolor{{math}}{{HTML}}{{{color6}}}

%%%%%%%%%%%%%%%%%%%%%%%%%%%%%%%%%%%%%%%% IMPORTS %%%%%%%%%%%%%%%%%%%%%%%%%%%%%%%%%%%%%%%%
\documentclass[11pt,onesize,a4paper,titlepage]{article}

%%%%%%%%%%%%%%% Formatting %%%%%%%%%%%%%%% 
\usepackage[english]{babel}
\usepackage[utf8]{inputenc}
\usepackage{adjustbox}
\usepackage{geometry} % Margins
\usepackage{sectsty} % Custom Sections

%%%%%%%%%%%%%%% Font %%%%%%%%%%%%%%% 
\usepackage{Archivo}
\usepackage[T1]{fontenc}
\sffamily

%%%%%%%%%%%%%%% Graphics %%%%%%%%%%%%%%% 
\usepackage{fontawesome5} % Icons
\usepackage{graphicx} % Images
\usepackage[most]{tcolorbox} % Color Box
\usepackage{xcolor} % Colors
\usepackage{tikz} % For Drawing Shapes
%%%\usepackage{emoji} % For flags
\tcbuselibrary{breakable}
%%%\usepackage{academicons}

%%%%%%%%%%%%%%% Miscelanous %%%%%%%%%%%%%%% 
\usepackage{lipsum} % Lorem Ipsum
\usepackage{hyperref} % For Hyperlinks

%%%%%%%%%%%%%%% Colors %%%%%%%%%%%%%%% 
\definecolor{title}{HTML}{b5bff5} % Color of the title
\definecolor{bars}{HTML}{889af0} % Color of the title
\definecolor{backdrop}{HTML}{f2f2f2} % Color of the side column
\definecolor{lightgray}{HTML}{dfdfdf} % Color for the skill bars

%%% TU green: #639a00
%%% TU gray: #e6e6e6
%\definecolor{title}{HTML}{639a00} % Color of the title TU
%\definecolor{bars}{HTML}{889af0} % Color of the title TU

% \definecolor{backdrop}{HTML}{f2f2f2} % Color of the side column
\definecolor{backdrop}{HTML}{e6e6e6} % Color of the side column

\definecolor{subtitle}{HTML}{606060} % 


%%%%%%%%%%%%%%% Section Format %%%%%%%%%%%%%%% 
\sectionfont{                     
    \LARGE % Font size
    \sectionrule{0pt}{0pt}{-8pt}{1pt} % Rule under Section name
}

\subsectionfont{
    \Large % Font size
    \fontfamily{phv}\selectfont % Font family
    %\sectionrule{0pt}{0pt}{-8pt}{1pt} % Rule under Subsection name
    \sectionrule{5pt}{0pt}{0pt}{0pt} % Rule under Subsection name
}

%%%%%%%%%%%%%%% Margins and Headers %%%%%%%%%%%%%%%
\geometry{
  a4paper,
  left=7mm,
  right=7mm,
  bottom=10mm,
  top=10mm
}

\pagestyle{empty} % Empty Headers

\usepackage{marvosym}

% \renewcommand\qedsymbol{\CoffeeCup}

\usepackage{changepage}

\newenvironment{subexercise}[1]{%
    \begin{mdframed}[linewidth=0.5pt, linecolor=foreground, backgroundcolor=background, leftmargin=0cm, innerleftmargin=1em, innertopmargin=0pt, innerbottommargin=0pt, innerrightmargin=0pt, topline=false, rightline=false, bottomline=false]
    \par\noindent\textcolor{foreground}{\textbf{#1.}}\hspace{1em}\ignorespaces
}{%
    \par\addvspace{\baselineskip}\end{mdframed}\ignorespacesafterend
}
\newenvironment{solution}{%
    % \par\addvspace{\baselineskip}\noindent\makebox[\textwidth]{\textcolor{foreground}{\textbullet\hspace{1em}\textbullet\hspace{1em}\textbullet}}\par\addvspace{\baselineskip}
    \begin{mdframed}[linewidth=0.5pt, linecolor=foreground, backgroundcolor=background, rightmargin=0cm, innerleftmargin=0cm, innertopmargin=0pt, innerbottommargin=0pt, innerrightmargin=1em, topline=false, leftline=false, bottomline=false]
    \par\noindent\textcolor{foreground}{\textit{Solution.}}\hspace{1em}\ignorespaces
}{%
    \par\addvspace{\baselineskip}\noindent\hfill\textcolor{foreground}{\Coffeecup}\par\addvspace{\baselineskip}\end{mdframed}\ignorespacesafterend
}
% Exercise environment

\declaretheoremstyle[
    name= \textcolor{foreground}{Exercise},
    postheadspace = \newline,
    bodyfont = \normalfont\color{foreground},
    postheadhook={\textcolor{math}{\rule[.4ex]{\linewidth}{0.5pt}}\\},
    % numberwithin=chapter,
    mdframed={
        backgroundcolor = background,
        linecolor = foreground,
        linewidth = 0.5pt,
        rightline =  true,
        topline = true,
        bottomline = true,
        skipabove=20pt,
        skipbelow=20pt,
        innerleftmargin=15pt,
        innertopmargin=10pt,
        innerrightmargin=15pt,
        innerbottommargin=10pt}
    ]{exercise}
\declaretheorem[style=exercise,numbered=no]{exercise}

% \etocsetlevel{exercise}{2}

% \AtEndEnvironment{exercise}{%
%   \etoctoccontentsline{exercise}{\protect\numberline{\theexercise}}%
% }%
% \etocsetstyle{exercise}
% {}
% {}
% % this will be rendered like a non-numbered section, but we could have used
% % \numberline here also
% {\etocsavedsectiontocline{Exercise \etocnumber}{\etocpage}}
%     {}

% theorem environment

\declaretheoremstyle[
    name= \textcolor{foreground}{Theorem},
    postheadspace = \newline,
    bodyfont = \normalfont\color{foreground},
    postheadhook={\textcolor{math}{\rule[.4ex]{\linewidth}{1pt}}\\},
    mdframed={
        backgroundcolor = background,
        linecolor = foreground,
        linewidth = 1pt,
        rightline =  true,
        topline = true,
        bottomline = true,
        skipabove=20pt,
        skipbelow=20pt,
        innerleftmargin=15pt,
        innertopmargin=10pt,
        innerrightmargin=15pt,
        innerbottommargin=10pt}
    ]{theorem}
\declaretheorem[style=theorem,numbered=yes]{theorem}

\declaretheoremstyle[
    name= \textcolor{foreground}{Definition},
    postheadspace = \newline,
    bodyfont = \normalfont\color{foreground},
    postheadhook={\textcolor{math}{\rule[.4ex]{\linewidth}{1pt}}\\},
    mdframed={
        backgroundcolor = background,
        linecolor = foreground,
        linewidth = 1pt,
        rightline =  true,
        topline = true,
        bottomline = true,
        skipabove=20pt,
        skipbelow=20pt,
        innerleftmargin=15pt,
        innertopmargin=10pt,
        innerrightmargin=15pt,
        innerbottommargin=10pt}
    ]{definition}
\declaretheorem[style=definition,numbered=yes]{definition}
% Example environment

\declaretheoremstyle[
name= \quad \underline{Proof:},
     headfont = \bfseries\sffamily,
     postheadspace = \newline,
     % notebraces = \bfseries{(}{)a},
     headpunct = {},
     bodyfont = ,
     postheadhook={\textcolor{foreground}{\rule[0.4ex]{\linewidth}{0pt}}\\},
     qed=\qedsymbol,
    % spacebelow = 10pt,
    mdframed={
  backgroundcolor = background,
  linecolor = foreground,
  linewidth = 1pt,
  skipabove=10pt,
  skipbelow=10pt,
  rightline = false,
  topline = false,
  leftline = false,
  bottomline = false,
  innerleftmargin=15pt,
  innertopmargin=15pt,
  innerrightmargin=15pt,
  innerbottommargin=15pt}
]{pro}
    % \declaretheorem[style=pro,numbered=no]{Proof}

\declaretheoremstyle[
name= \quad \underline{\textcolor{foreground}{Example}},
     headfont = \bfseries\sffamily,
     postheadspace = \newline,
     % notebraces = \bfseries{(}{)a},
     headpunct = {},
     bodyfont = \normalfont\color{foreground},
     postheadhook={\textcolor{foreground}{\rule[0.4ex]{\linewidth}{0pt}}\\},
     % spacebelow = 10pt,
    mdframed={
  backgroundcolor = background,
  linecolor = foreground,
  linewidth = 1pt,
  skipabove=10pt,
  skipbelow=10pt,
  rightline = false,
  topline = false,
  leftline = false,
  bottomline = false,
  innerleftmargin=15pt,
  innertopmargin=15pt,
  innerrightmargin=15pt,
  innerbottommargin=15pt}
]{ex}
\declaretheorem[style=ex,numbered=no]{example}

\declaretheoremstyle[
     name=,
     headfont = \bfseries\sffamily,
     notebraces = \bfseries{},
     headpunct = { -},
     bodyfont = \color{foreground}\normalfont,
     % postheadhook={\textcolor{black}{\rule[.4ex]{\linewidth}{0.2pt}}\\},
    % spacebelow = 10pt,
    mdframed={
  backgroundcolor = background,
  linecolor = foreground,
  linewidth = 1pt,
  skipabove=0pt,
  skipbelow=0pt,
  innerleftmargin=10pt,
  innertopmargin=10pt,
  innerrightmargin=10pt,
  innerbottommargin=10pt,
  rightline = false,
  topline = false,
  leftline = false,
  bottomline = true}
]{subexercise}
% \declaretheorem[style=subexercise,numbered=no]{subexercise}

\declaretheoremstyle[
     name= \color{losning}Løsning,
     headfont = \bfseries\sffamily,
     notebraces = \bfseries{},
     postheadspace = \newline,
     headpunct = {:},
     bodyfont = \normalfont,
     % qed = ,
     % postheadhook={\textcolor{black}{\rule[.4ex]{\linewidth}{0.2pt}}\\},
    % spacebelow = 10pt,
    mdframed={
  backgroundcolor = background,
  linecolor = losning!75,
  linewidth = 1pt,
  skipabove=0pt,
  skipbelow=10pt,
  innerleftmargin=10pt,
  innertopmargin=10pt,
  innerrightmargin=10pt,
  innerbottommargin=10pt,
  leftline = false,
  rightline = true,
  topline = false,
  bottomline = true}
]{solution}

\newenvironment{SimpleBox}[1]{%
  \begin{mdframed}%
    \noindent\textbf{#1}\\[1ex]
}{%
  \end{mdframed}%
}


\begin{document}
\let\cleardoublepage\clearpage
\thispagestyle{fancy}
\chapter{Free gauge fields Lagrangian}
In this exercise we wish to show that the free Lagrangian of the gauge fields \( \mathbf{A}_\mu \),
\begin{align*}
\mathcal{L}_A &= -\frac{1}{16\pi} \mathbf{F}^{\mu\nu} \cdot \mathbf{F}_{\mu\nu}, \tag{9}
\end{align*}
is invariant under a local gauge transformation. Note that the field tensor in Yang-Mills theory takes the form
\begin{align*}
\mathbf{F}^{\mu\nu} &= \partial^\mu \mathbf{A}^\nu - \partial^\nu \mathbf{A}^\mu + 2g (\mathbf{A}^\mu \times \mathbf{A}^\nu). \tag{10}
\end{align*}

\begin{exercise}[1]
Show that the commutator of the covariant derivative is
\begin{align*}
[D_\mu, D_\nu] &= -ig \mathbf{F}_{\mu\nu} \cdot \boldsymbol{\sigma}. \tag{11}
\end{align*}
Hint: You might find the following identity for the Pauli matrices useful: \( (\boldsymbol{\sigma} \cdot \mathbf{a})(\boldsymbol{\sigma} \cdot \mathbf{b}) = \mathbf{a} \cdot \mathbf{b} + i \boldsymbol{\sigma} \cdot (\mathbf{a} \times \mathbf{b}) \), where \( \mathbf{a} \) and \( \mathbf{b} \) are vectors.
\end{exercise}

Lets start by recalling the covariant derivative,
\[
D_{\mu } = \partial_{\mu } - i g\mathbf{A}_{\mu }\cdot \mathbf{\sigma}
.\] 
The following commutator will be useful,
\begin{align*}
    \left[ \partial_\mu , \mathbf{A_v} \right] 
.\end{align*}
Let's see how it acts on a test function $\Psi$,
\begin{align*}
\left[ \partial_\mu, \mathbf{A}_{v}  \right] \Psi &= \partial_\mu \left( \mathbf{A}_v \Psi \right) - \mathbf{A}_v \left( \partial_\mu \Psi \right)  \\
 &= \left(   \partial_\mu \mathbf{A}_v \right)\Psi +\mathbf{A_v}\partial_\mu \Psi - \mathbf{A}_v \left( \partial_\mu \Psi \right)  \\
 &= \left( \partial_\mu \mathbf{A}_v \right) \Psi 
.\end{align*}
Therefore,
\[
\left[ \partial_\mu , \mathbf{A}_v \right] = \partial_{\mu }\mathbf{A}_v
.\] 
Using this, we can calculate the commutator of the covariant derivative,
\begin{align*}
    \left[ D_\mu , D_v \right] &= \left[ \partial_\mu  - ig \mathbf{A}_{\mu }\cdot\mathbf{\sigma}, \partial_v  - ig \mathbf{A}_{v}\cdot\mathbf{\sigma} \right] \\
    &= \left[ \partial_\mu , \partial_v \right] + \left[ \partial_\mu , -ig\mathbf{A_v}\cdot \mathbf{\sigma} \right] + \left[ -ig\mathbf{A_\mu }\cdot \mathbf{\sigma} , \partial_v\right]  + \left[ -ig\mathbf{A}_{\mu }\cdot \mathbf{\sigma}, -ig\mathbf{A}_v\cdot \mathbf{\sigma} \right] \\
    &=  -ig\left[ \partial_\mu , \mathbf{A_v}\right] \cdot \mathbf{\sigma} -ig\left[ \mathbf{A_\mu }, \partial_v \right] \cdot \mathbf{\sigma} -g^2 \left[ \mathbf{A}_{\mu }\cdot \mathbf{\sigma}, \mathbf{A}_v\cdot \mathbf{\sigma} \right] \\
    &=  -ig\left(\partial_\mu \mathbf{A_v}\right) \cdot \mathbf{\sigma} + ig\left(  \partial_v\mathbf{A_\mu } \right) \cdot \mathbf{\sigma} -g^2 \left[ \mathbf{A}_{\mu }\cdot \mathbf{\sigma}, \mathbf{A}_v\cdot \mathbf{\sigma} \right] \\
    &=  -ig\left(\partial_\mu \mathbf{A_v}-  \partial_v\mathbf{A_\mu } \right) \cdot \mathbf{\sigma} -g^2 \left[ \mathbf{A}_{\mu }\cdot \mathbf{\sigma}, \mathbf{A}_v\cdot \mathbf{\sigma} \right] \\
.\end{align*}
Now we just need to calculate the final commutator,
\begin{align*}
    \left[ \mathbf{A}_{\mu }\cdot \mathbf{\sigma}, \mathbf{A}_v \cdot \mathbf{\sigma} \right] &= 
    \left( \mathbf{A}_{\mu }\cdot \mathbf{\sigma}\right)\cdot \left(    \mathbf{A}_v \cdot \mathbf{\sigma} \right) -
     \left(    \mathbf{A}_v \cdot \mathbf{\sigma} \right)\cdot\left( \mathbf{A}_{\mu }\cdot \mathbf{\sigma}\right) \\
     &= \mathbf{A_\mu }\cdot \mathbf{A_v} + \left( \mathbf{A_\mu }\times \mathbf{A_v} \right) \cdot i\mathbf{\sigma}  -
\left(    \mathbf{A }_v\cdot \mathbf{A}_{\mu } + \left( \mathbf{A_v }\times \mathbf{A_\mu } \right) \cdot i\mathbf{\sigma}  \right)\\
     &=  \left( \mathbf{A_\mu }\times \mathbf{A_v} \right) \cdot i\mathbf{\sigma}  -
\left(   \left( \mathbf{A_v }\times \mathbf{A_\mu } \right) \cdot i\mathbf{\sigma}  \right)\\
     &=2\left( \mathbf{A}_{\mu }\times \mathbf{A}_{v} \right) \cdot i\mathbf{\sigma}
.\end{align*}
we can insert this,
\begin{align*}
     \left[ D_\mu , D_v \right] &=  
     -ig\left(\partial_\mu \mathbf{A_v}-  \partial_v\mathbf{A_\mu } \right) \cdot \mathbf{\sigma} -g^2 \left[ \mathbf{A}_{\mu }\cdot \mathbf{\sigma}, \mathbf{A}_v\cdot \mathbf{\sigma} \right] \\
    &=  -ig\left(\partial_\mu \mathbf{A_v}-  \partial_v\mathbf{A_\mu } \right) \cdot \mathbf{\sigma} -g^2 2\left( \mathbf{A}_{\mu }\times \mathbf{A}_{v} \right) \cdot i\mathbf{\sigma} \\
    &=  -ig\left(\partial_\mu \mathbf{A_v}-  \partial_v\mathbf{A_\mu }  +2g\left( \mathbf{A}_{\mu }\times \mathbf{A}_{v} \right) \right)\cdot \mathbf{\sigma} = -ig\mathbf{F}_{\mu v}\cdot \mathbf{\sigma}
.\end{align*}
\begin{exercise}[2]
Argue that the transformation law of the covariant derivative in eq. (7) implies that
\begin{align*}
[D_\mu, D_\nu]\psi &\rightarrow V(x)[D_\mu, D_\nu]\psi, \tag{12}
\end{align*}
and show that this implies that
\begin{align*}
\mathbf{F}_{\mu\nu} \cdot \boldsymbol{\sigma} &\rightarrow V(x) \mathbf{F}_{\mu\nu} \cdot \boldsymbol{\sigma} V^\dagger(x). \tag{13}
\end{align*}
\end{exercise}
We have the following transformation law,
\[
D_\mu\Psi \to V\left( x \right) \left( D_\mu \Psi \right) 
.\] 
Applying this this to commutator,
\begin{align*}
    \left[ D_\mu , D_v \right] \Psi = D_\mu \left( D_v \Psi \right) - D_v\left( D_\mu \Psi \right)\to   D_\mu \left(V\left( x \right)  D_v \Psi \right) - D_v\left( V\left( x \right) D_\mu \Psi \right)
    &=  V\left( x \right) \left[ D_\mu , D_v \right] \Psi
.\end{align*} 
Lets show that this implies (13). From the previous exercise we know,
\[
\left[ D_\mu , D_v \right] = -ig \mathbf{F}_{\mu v}\cdot \mathbf{\sigma}
.\] 
And therefore,
\[
\left[ D_\mu , D_v \right]' = -ig \mathbf{F}_{\mu v}'\cdot \mathbf{\sigma}
.\] 
We can insert this,
\begin{align*}
    \left( -ig\mathbf{F}_{\mu  v}' \cdot  \mathbf{\sigma} \right) \Psi' &= V\left( x \right) \left( -ig \mathbf{F}_{\mu  v}\cdot \mathbf{\sigma} \right) \Psi\\
    \left( \mathbf{F}_{\mu  v}' \cdot  \mathbf{\sigma} \right) \Psi' &= V\left( x \right) \left( \mathbf{F}_{\mu  v}\cdot \mathbf{\sigma} \right) \Psi \\
    \left( \mathbf{F}_{\mu  v}' \cdot  \mathbf{\sigma} \right) V\left( x \right) \Psi &= V\left( x \right) \left( \mathbf{F}_{\mu  v}\cdot \mathbf{\sigma} \right) \Psi
.\end{align*}
This will have to hold for any field $\Psi$, so we can equate the operators,
\begin{align*}
    \left( \mathbf{F}_{\mu  v}' \cdot  \mathbf{\sigma} \right) V\left( x \right)  &= V\left( x \right) \left( \mathbf{F}_{\mu  v}\cdot \mathbf{\sigma} \right) \\
    \left( \mathbf{F}_{\mu  v}' \cdot  \mathbf{\sigma} \right) V\left( x \right) V^{-1}\left( x \right)  &= V\left( x \right) \left( \mathbf{F}_{\mu  v}\cdot \mathbf{\sigma} \right) V^{-1}\left( x \right) \\
    \left( \mathbf{F}_{\mu  v}' \cdot  \mathbf{\sigma} \right) &= V\left( x \right) \left( \mathbf{F}_{\mu  v}\cdot \mathbf{\sigma} \right) V\left( x \right)^{\dagger} \\
.\end{align*}
\begin{exercise}[3]
Using the transformation of eq. (13) to show that
\begin{align*}
\text{Tr} \left[ (\mathbf{F}^{\mu\nu} \cdot \boldsymbol{\sigma}) (\mathbf{F}_{\mu\nu} \cdot \boldsymbol{\sigma}) \right], \tag{14}
\end{align*}
is invariant.
\end{exercise}
We show this by applying the transformation, and asserting that the resulting is unchanged,
\begin{align*}
    \text{Tr}\left[\left( \mathbf{F^{\mu  v}}\cdot \mathbf{\sigma} \right) \left( \mathbf{F_{\mu v}}\cdot \mathbf{\sigma} \right) \right]' &=
    \text{Tr}\left[\left( \mathbf{F^{\mu  v}}'\cdot \mathbf{\sigma} \right) \left( \mathbf{F_{\mu v}}'\cdot \mathbf{\sigma} \right) \right] \\
&= \text{Tr}\left[\left(V\left( x \right)  \mathbf{F^{\mu  v}}\cdot \mathbf{\sigma}V\left( x \right)^{\dagger}  \right) \left(V\left( x \right)  \mathbf{F_{\mu v}}'\cdot \mathbf{\sigma}V\left( x \right) ^{\dagger} \right) \right] \\
&= \text{Tr}\left[\left(V\left( x \right)  \mathbf{F^{\mu  v}}\cdot \mathbf{\sigma}  \right)   \mathbf{F_{\mu v}}'\cdot \mathbf{\sigma}V\left( x \right) ^{\dagger}  \right] \\
.\end{align*}
The trace is invariant under circular shifts,
\begin{align*}
 \text{Tr}\left[\left(V\left( x \right)  \mathbf{F^{\mu  v}}\cdot \mathbf{\sigma}  \right)   \mathbf{F_{\mu v}}'\cdot \mathbf{\sigma}V\left( x \right) ^{\dagger}  \right] =
 \text{Tr}\left[  \mathbf{F^{\mu  v}}\cdot \mathbf{\sigma}     \mathbf{F_{\mu v}}'\cdot \mathbf{\sigma}V\left( x \right) ^{\dagger}  V\left( x \right)\right] =
 \text{Tr}\left[  \mathbf{F^{\mu  v}}\cdot \mathbf{\sigma}     \mathbf{F_{\mu v}}'\cdot \mathbf{\sigma}\right] 
.\end{align*}
\begin{exercise}[4]
Show that the trace in eq. (14) is equal to \( 2 \mathbf{F}^{\mu\nu} \cdot \mathbf{F}_{\mu\nu} \).
\end{exercise}
We do this by expanding the dot-product within the trace.
\[
\mathbf{F}^{\mu  v}\cdot \sigma =  \sum_\alpha \mathbf{F}^{\mu v}_\alpha \mathbf{\sigma}_\alpha
.\] 
We can insert these,
\[
\text{Tr}\left[\left( \mathbf{F}^{\mu  v}\cdot  \mathbf{ \sigma} \right) \left( \mathbf{F}_{\mu  v}\cdot \mathbf{\sigma} \right) \right] = \text{Tr}\left[\left( \mathbf{F}^{\mu v}\cdot \mathbf{F}_{\mu v} \right)\mathbb{I}_{2\times 2} + i\mathbf{\sigma} \cdot \left( \mathbf{F}^{\mu v}\times \mathbf{F}_{\mu  v} \right)  \right]
.\] 
Since everything here is index-wise, and we are implicitly performing sums, we can move the the field-strength tensors outside of the trace,
\[
= \left( \mathbf{F}^{\mu v}\cdot \mathbf{F}_{\mu  v} \right) \text{Tr}\left[\mathbb{I}_{2\times 2}\right] + i  \sum_k\text{Tr}\left[\mathbf{\sigma}_{k}\right]\left( \mathbf{F^{\mu v}}\times \mathbf{F}_{\mu v} \right) _k = 2\mathbf{F}^{\mu v}\cdot \mathbf{F}_{\mu v}
.\] 
\begin{exercise}[5]
Combine 3) and 4) to show that the Lagrangian in eq. (9) is invariant.
\end{exercise}
\end{document}
