\documentclass[working, oneside]{../../../Preambles/tuftebook}
% Import xcolor and define some colors
\usepackage{xcolor}
\definecolor{background}{HTML}{ffffff}
\definecolor{foreground}{HTML}{000000}
\definecolor{math}{HTML}{000000}

%%%%%%%%%%%%%%%%%%%%
%% SUPER PREAMBLE %%
%%%%%%%%%%%%%%%%%%%%

% \usepackage[utf8]{inputenc}
\usepackage[T1]{fontenc} % Fonts and stuff
\usepackage{amsmath, amsfonts, mathtools, amsthm, amssymb} % math

\usepackage{fancyhdr} % Header, Footer etc.
\usepackage{adforn}
\usepackage{efbox}
\usepackage{lastpage}
\usepackage{marvosym}
\usepackage{pict2e}
\usepackage{caption}
\usepackage{wrapfig}
\usepackage{graphicx}
\usepackage{sidecap}
% \usepackage{mathpazo} 
% \usepackage{cmbright}
\usepackage{mathptmx}

\usepackage[
    sorting=nyt,
    style=alphabetic
]{biblatex}
\addbibresource{references.bib}
\usepackage[noabbrev]{cleveref}

\pagestyle{fancy}
\fancyhead[R]{}
\fancyhead[L]{}
\fancyfoot[C]{\efbox[margin = 10pt,
                    topline = false,
                    leftline = false,
                    rightline = false,
                    backgroundcolor = background,
                    linewidth = 1pt,
                    linecolor = foreground]{\thepage\ of \pageref{LastPage}}}
% \fancyfoot[C]{\color{foreground} \thepage}

% \renewcommand{\headrule}{%
% 	\hrulefill
% }
% \renewcommand{\footrulewidth}{0pt}
\renewcommand{\headrulewidth}{0pt}

% \setlength{\headheight}{15pt}
% \setlength{\footheight}{15pt}

%% Margin Control %%

% \def\changemargin#1#2{\list{}{\rightmargin#2\leftmargin#1}\item[]}
% \let\endchangemargin=\endlist


%%%%%%%%%%%%%%%%%%%%%%%%%%%%%%%%%%%%%%%%%%%%%%%%%%%%%%%%%%

% figure support

\usepackage{import}
\usepackage{transparent}


\newcommand{\incfig}[2][1]{%
    \def\svgwidth{#1\columnwidth}
    \import{figures/}{#2.pdf_tex}
}

% \pdfsuppresswarningpagegroup=1

%%%%%%%%%%%%%%%%%%%%%%%%%%%%%%%%%%%%%%%%%%%%%%%%%%%%%%%%%%

\usepackage{tikzsymbols} % Symbols
\usepackage[framemethod=TikZ]{mdframed} % Boxes around theorem environments
\usepackage{thmtools}




% \everymath{\color{math}}
% \everydisplay{\color{math}}
% \def\m@th{\normalcolor\mathsurround\z@}

\color{foreground}

	% \end{changemargin}
	% } 

 % \newenvironment{subexercise}[1]
 % {\noindent
	 % \textbf{(#1)} \quad \adforn{10} \quad \em
 % }{}

% Mathematical typesetting stuff.

 % \newcommand{\dd}{\mathrm{\textbf{d}}}

 % Change font

% \usepackage{tgadventor}
% \usepackage{cmbright}
% \usepackage{bm}

% \usepackage{microtype} % Microtypography
% \usepackage{fontspec}% Hyperlinks
% \usepackage{fouriernc}

% \def\MT@set@inh@list#1#2{%
%   \MT@ifempty\MT@inh@feat{%
%     \MT@map@clist@c\MT@features{\begingroup % <--
%       \MT@ifstreq{##1}{tr}\relax{\MT@declare@char@inh{##1}{#1}{#2}}%
%     \endgroup}% <--
%   }{%
%     \MT@map@clist@c\MT@inh@feat{\begingroup % <--
%       \KV@@sp@def\@tempa{##1}%
%       \MT@ifempty\@tempa\relax{%
%         \edef\@tempa{\csname MT@rbba@\@tempa\endcsname}%
%         \MT@ifstreq\@tempa{tr}\relax{%
%           \MT@exp@one@n\MT@declare@char@inh{\@tempa}{#1}{#2}}}%
%     \endgroup}% <--
%   }%      
% \DeclareCaptionFormat{custom}{\bfseries#1#2\itshape#3}%   \MT@end@catcodes
\DeclareCaptionFormat{custom}{\bfseries\itshape#1#2\normalfont\small#3}
\captionsetup{
    format=custom,
    labelsep=space,
    width=\textwidth, % Set the caption width to be 80% of the text width
    % justification=jusitified, % Center-align the caption
    % font=it % Italicize the caption text
}
% }
\usepackage{setspace}
\DeclareCaptionFormat{margin}{\small\bfseries#1#2#3}

\usepackage{xparse} % For advanced command definitions with optional arguments

\NewDocumentCommand{\marginfig}{O{0cm} m m m}{%
  % #1 = optional padding (default 0cm), #2 = filename, #3 = label, #4 = caption
  \marginpar{%
    \includegraphics[width=\marginparwidth]{#2}%
    \captionsetup{format=custom, labelsep=space, width=\marginparwidth, justification=raggedright, font=small}
    \captionof{figure}{#4}%
    \label{fig:#3}%
    % \rule{\marginparwidth}{0.4pt} % Adds a line below the caption
    \vspace{#1} % Adds the specified padding below the caption
  }%
}

\NewDocumentCommand{\maintextfig}{O{0cm} m m m}{
  % #1 = optional vertical adjustment for the caption (default 0cm)
  % #2 = filename for the figure
  % #3 = label for the figure
  % #4 = caption text

  % Place the figure in the text
  \begin{figure}[htbp]
    \centering
    \includegraphics[width=\textwidth]{#2}
    \marginnote{\captionsetup{format=custom, labelsep=space, width=\marginparwidth, justification=fill, font={stretch=1}}
        \captionof{figure}[#4]{#4}\label{fig:#3}}[#1]
    \label{fig:#3}
  \end{figure}

  % Place the caption in the margin
  % \marginnote{\captionsetup{format=custom, labelsep=space, width=\marginparwidth, justification=raggedright, font={stretch=1}}
  %   \captionof{figure}[#4]{#4}\label{fig:#3}}[#1]
}
% \NewDocumentCommand{\marginfig}{m m m}{
%   % #1 = filename, #2 = label, #3 = caption
%   \begin{wrapfigure}{r}{5cm} % "r" for right side, and "5cm" for the width of the figure
%       \centering
%       \includegraphics[width=5cm]{#1}
%      \captionsetup{format=custom, labelsep=space, width=6cm, justification=raggedright, font={stretch=1}}
%       \captionof{figure}{#3}%
%       \label{fig:#2}
%   \end{wrapfigure}
% }{}
% \newcommand{\marginfig}[3]{%
%   \marginpar{%
%     \includegraphics[width=\marginparwidth]{#1}%
%     \captionsetup{format=custom, labelsep=space, width=\marginparwidth, justification=raggedright, font={stretch=1}}
%     \captionof{figure}{\fontsize{11pt}{11pt}\selectfont #3}%
%     \label{fig:#2}%
%     % \rule{\marginparwidth}{0.4pt}
%   }%
% }
% \newcommand{\marginfig}[4][0pt]{%
%   \marginpar{%
%     \raisebox{#1}{%
%       \includegraphics[width=\marginparwidth]{#2}%
%       \captionsetup{format=margin, labelsep=space, justification=raggedright}
%       \captionof{figure}{#4}%
%       \label{fig:#3}%
%     }%
%   }%
% }
% \newcommand{\marginfig}[2][0pt]{%
%   \marginpar{\raisebox{#1}{%
%       \includegraphics[width=1.0\marginparwidth]{#2}
%       % \label{fig:#3}
%       % \caption{#4}
%       % \parbox{\marginparwidth}{\smaller \textbf{figure:} #3}%
%   }}%
% }
\newcommand{\margintext}[2][0pt]{%
  \marginpar{\raisebox{#1}{%
    \parbox{\marginparwidth}{\smaller \textbf{figure:} #2}%
    }}%
}

\newcommand{\marginmath}[2][0pt]{%
  \marginpar{\raisebox{#1}{%
    \parbox{1.2\marginparwidth}{#2}%
    }}%
}
\pagecolor{background}
\usepackage{listings}
\definecolor{commentsColor}{rgb}{0.497495, 0.497587, 0.497464}
\definecolor{keywordsColor}{rgb}{0.000000, 0.000000, 0.635294}
\definecolor{stringColor}{rgb}{0.558215, 0.000000, 0.135316}
\renewcommand*\ttdefault{txtt}
\lstset{
  basicstyle=\ttfamily\small,                   % the size of the fonts that are used for the code
  breakatwhitespace=false,                      % sets if automatic breaks should only happen at whitespace
  breaklines=true,                              % sets automatic line breaking
  frame=tb,                                     % adds a frame around the code
  commentstyle=\color{commentsColor}\textit,    % comment style
  keywordstyle=\color{keywordsColor}\bfseries,  % keyword style
  stringstyle=\color{stringColor},              % string literal style
  numbers=left,                                 % where to put the line-numbers; possible values are (none, left, right)
  numbersep=5pt,                                % how far the line-numbers are from the code
  numberstyle=\tiny\color{commentsColor},       % the style that is used for the line-numbers
  showstringspaces=false,                       % underline spaces within strings only
  tabsize=2,                                    % sets default tabsize to 2 spaces
  language=Scala
}


\usepackage{marvosym}

% \renewcommand\qedsymbol{\CoffeeCup}

\usepackage{changepage}

\newenvironment{subexercise}[1]{%
    \begin{mdframed}[linewidth=0.5pt, linecolor=foreground, backgroundcolor=background, leftmargin=0cm, innerleftmargin=1em, innertopmargin=0pt, innerbottommargin=0pt, innerrightmargin=0pt, topline=false, rightline=false, bottomline=false]
    \par\noindent\textcolor{foreground}{\textbf{#1.}}\hspace{1em}\ignorespaces
}{%
    \par\addvspace{\baselineskip}\end{mdframed}\ignorespacesafterend
}
\newenvironment{solution}{%
    % \par\addvspace{\baselineskip}\noindent\makebox[\textwidth]{\textcolor{foreground}{\textbullet\hspace{1em}\textbullet\hspace{1em}\textbullet}}\par\addvspace{\baselineskip}
    \begin{mdframed}[linewidth=0.5pt, linecolor=foreground, backgroundcolor=background, rightmargin=0cm, innerleftmargin=0cm, innertopmargin=0pt, innerbottommargin=0pt, innerrightmargin=1em, topline=false, leftline=false, bottomline=false]
    \par\noindent\textcolor{foreground}{\textit{Solution.}}\hspace{1em}\ignorespaces
}{%
    \par\addvspace{\baselineskip}\noindent\hfill\textcolor{foreground}{\Coffeecup}\par\addvspace{\baselineskip}\end{mdframed}\ignorespacesafterend
}
% Exercise environment

\declaretheoremstyle[
    name= \textcolor{foreground}{Exercise},
    postheadspace = \newline,
    bodyfont = \normalfont\color{foreground},
    postheadhook={\textcolor{math}{\rule[.4ex]{\linewidth}{0.5pt}}\\},
    % numberwithin=chapter,
    mdframed={
        backgroundcolor = background,
        linecolor = foreground,
        linewidth = 0.5pt,
        rightline =  true,
        topline = true,
        bottomline = true,
        skipabove=20pt,
        skipbelow=20pt,
        innerleftmargin=15pt,
        innertopmargin=10pt,
        innerrightmargin=15pt,
        innerbottommargin=10pt}
    ]{exercise}
\declaretheorem[style=exercise,numbered=no]{exercise}

% \etocsetlevel{exercise}{2}

% \AtEndEnvironment{exercise}{%
%   \etoctoccontentsline{exercise}{\protect\numberline{\theexercise}}%
% }%
% \etocsetstyle{exercise}
% {}
% {}
% % this will be rendered like a non-numbered section, but we could have used
% % \numberline here also
% {\etocsavedsectiontocline{Exercise \etocnumber}{\etocpage}}
%     {}

% theorem environment

\declaretheoremstyle[
    name= \textcolor{foreground}{Theorem},
    postheadspace = \newline,
    bodyfont = \normalfont\color{foreground},
    postheadhook={\textcolor{math}{\rule[.4ex]{\linewidth}{1pt}}\\},
    mdframed={
        backgroundcolor = background,
        linecolor = foreground,
        linewidth = 1pt,
        rightline =  true,
        topline = true,
        bottomline = true,
        skipabove=20pt,
        skipbelow=20pt,
        innerleftmargin=15pt,
        innertopmargin=10pt,
        innerrightmargin=15pt,
        innerbottommargin=10pt}
    ]{theorem}
\declaretheorem[style=theorem,numbered=yes]{theorem}

\declaretheoremstyle[
    name= \textcolor{foreground}{Definition},
    postheadspace = \newline,
    bodyfont = \normalfont\color{foreground},
    postheadhook={\textcolor{math}{\rule[.4ex]{\linewidth}{1pt}}\\},
    mdframed={
        backgroundcolor = background,
        linecolor = foreground,
        linewidth = 1pt,
        rightline =  true,
        topline = true,
        bottomline = true,
        skipabove=20pt,
        skipbelow=20pt,
        innerleftmargin=15pt,
        innertopmargin=10pt,
        innerrightmargin=15pt,
        innerbottommargin=10pt}
    ]{definition}
\declaretheorem[style=definition,numbered=yes]{definition}
% Example environment

\declaretheoremstyle[
name= \quad \underline{Proof:},
     headfont = \bfseries\sffamily,
     postheadspace = \newline,
     % notebraces = \bfseries{(}{)a},
     headpunct = {},
     bodyfont = ,
     postheadhook={\textcolor{foreground}{\rule[0.4ex]{\linewidth}{0pt}}\\},
     qed=\qedsymbol,
    % spacebelow = 10pt,
    mdframed={
  backgroundcolor = background,
  linecolor = foreground,
  linewidth = 1pt,
  skipabove=10pt,
  skipbelow=10pt,
  rightline = false,
  topline = false,
  leftline = false,
  bottomline = false,
  innerleftmargin=15pt,
  innertopmargin=15pt,
  innerrightmargin=15pt,
  innerbottommargin=15pt}
]{pro}
    % \declaretheorem[style=pro,numbered=no]{Proof}

\declaretheoremstyle[
name= \quad \underline{\textcolor{foreground}{Example}},
     headfont = \bfseries\sffamily,
     postheadspace = \newline,
     % notebraces = \bfseries{(}{)a},
     headpunct = {},
     bodyfont = \normalfont\color{foreground},
     postheadhook={\textcolor{foreground}{\rule[0.4ex]{\linewidth}{0pt}}\\},
     % spacebelow = 10pt,
    mdframed={
  backgroundcolor = background,
  linecolor = foreground,
  linewidth = 1pt,
  skipabove=10pt,
  skipbelow=10pt,
  rightline = false,
  topline = false,
  leftline = false,
  bottomline = false,
  innerleftmargin=15pt,
  innertopmargin=15pt,
  innerrightmargin=15pt,
  innerbottommargin=15pt}
]{ex}
\declaretheorem[style=ex,numbered=no]{example}

\declaretheoremstyle[
     name=,
     headfont = \bfseries\sffamily,
     notebraces = \bfseries{},
     headpunct = { -},
     bodyfont = \color{foreground}\normalfont,
     % postheadhook={\textcolor{black}{\rule[.4ex]{\linewidth}{0.2pt}}\\},
    % spacebelow = 10pt,
    mdframed={
  backgroundcolor = background,
  linecolor = foreground,
  linewidth = 1pt,
  skipabove=0pt,
  skipbelow=0pt,
  innerleftmargin=10pt,
  innertopmargin=10pt,
  innerrightmargin=10pt,
  innerbottommargin=10pt,
  rightline = false,
  topline = false,
  leftline = false,
  bottomline = true}
]{subexercise}
% \declaretheorem[style=subexercise,numbered=no]{subexercise}

\declaretheoremstyle[
     name= \color{losning}Løsning,
     headfont = \bfseries\sffamily,
     notebraces = \bfseries{},
     postheadspace = \newline,
     headpunct = {:},
     bodyfont = \normalfont,
     % qed = ,
     % postheadhook={\textcolor{black}{\rule[.4ex]{\linewidth}{0.2pt}}\\},
    % spacebelow = 10pt,
    mdframed={
  backgroundcolor = background,
  linecolor = losning!75,
  linewidth = 1pt,
  skipabove=0pt,
  skipbelow=10pt,
  innerleftmargin=10pt,
  innertopmargin=10pt,
  innerrightmargin=10pt,
  innerbottommargin=10pt,
  leftline = false,
  rightline = true,
  topline = false,
  bottomline = true}
]{solution}

\newenvironment{SimpleBox}[1]{%
  \begin{mdframed}%
    \noindent\textbf{#1}\\[1ex]
}{%
  \end{mdframed}%
}


\begin{document}
\let\cleardoublepage\clearpage
\thispagestyle{fancy}
\chapter{Free gauge fields Lagrangian}
In this exercise we wish to show that the free Lagrangian of the gauge fields \( \mathbf{A}_\mu \),
\begin{align*}
\mathcal{L}_A &= -\frac{1}{16\pi} \mathbf{F}^{\mu\nu} \cdot \mathbf{F}_{\mu\nu}, \tag{9}
\end{align*}
is invariant under a local gauge transformation. Note that the field tensor in Yang-Mills theory takes the form
\begin{align*}
\mathbf{F}^{\mu\nu} &= \partial^\mu \mathbf{A}^\nu - \partial^\nu \mathbf{A}^\mu + 2g (\mathbf{A}^\mu \times \mathbf{A}^\nu). \tag{10}
\end{align*}

\begin{exercise}[1]
Show that the commutator of the covariant derivative is
\begin{align*}
[D_\mu, D_\nu] &= -ig \mathbf{F}_{\mu\nu} \cdot \boldsymbol{\sigma}. \tag{11}
\end{align*}
Hint: You might find the following identity for the Pauli matrices useful: \( (\boldsymbol{\sigma} \cdot \mathbf{a})(\boldsymbol{\sigma} \cdot \mathbf{b}) = \mathbf{a} \cdot \mathbf{b} + i \boldsymbol{\sigma} \cdot (\mathbf{a} \times \mathbf{b}) \), where \( \mathbf{a} \) and \( \mathbf{b} \) are vectors.
\end{exercise}

Lets start by recalling the covariant derivative,
\[
D_{\mu } = \partial_{\mu } - i g\mathbf{A}_{\mu }\cdot \mathbf{\sigma}
.\] 
The following commutator will be useful,
\begin{align*}
    \left[ \partial_\mu , \mathbf{A_v} \right] 
.\end{align*}
Let's see how it acts on a test function $\Psi$,
\begin{align*}
\left[ \partial_\mu, \mathbf{A}_{v}  \right] \Psi &= \partial_\mu \left( \mathbf{A}_v \Psi \right) - \mathbf{A}_v \left( \partial_\mu \Psi \right)  \\
 &= \left(   \partial_\mu \mathbf{A}_v \right)\Psi +\mathbf{A_v}\partial_\mu \Psi - \mathbf{A}_v \left( \partial_\mu \Psi \right)  \\
 &= \left( \partial_\mu \mathbf{A}_v \right) \Psi 
.\end{align*}
Therefore,
\[
\left[ \partial_\mu , \mathbf{A}_v \right] = \partial_{\mu }\mathbf{A}_v
.\] 
Using this, we can calculate the commutator of the covariant derivative,
\begin{align*}
    \left[ D_\mu , D_v \right] &= \left[ \partial_\mu  - ig \mathbf{A}_{\mu }\cdot\mathbf{\sigma}, \partial_v  - ig \mathbf{A}_{v}\cdot\mathbf{\sigma} \right] \\
    &= \left[ \partial_\mu , \partial_v \right] + \left[ \partial_\mu , -ig\mathbf{A_v}\cdot \mathbf{\sigma} \right] + \left[ -ig\mathbf{A_\mu }\cdot \mathbf{\sigma} , \partial_v\right]  + \left[ -ig\mathbf{A}_{\mu }\cdot \mathbf{\sigma}, -ig\mathbf{A}_v\cdot \mathbf{\sigma} \right] \\
    &=  -ig\left[ \partial_\mu , \mathbf{A_v}\right] \cdot \mathbf{\sigma} -ig\left[ \mathbf{A_\mu }, \partial_v \right] \cdot \mathbf{\sigma} -g^2 \left[ \mathbf{A}_{\mu }\cdot \mathbf{\sigma}, \mathbf{A}_v\cdot \mathbf{\sigma} \right] \\
    &=  -ig\left(\partial_\mu \mathbf{A_v}\right) \cdot \mathbf{\sigma} + ig\left(  \partial_v\mathbf{A_\mu } \right) \cdot \mathbf{\sigma} -g^2 \left[ \mathbf{A}_{\mu }\cdot \mathbf{\sigma}, \mathbf{A}_v\cdot \mathbf{\sigma} \right] \\
    &=  -ig\left(\partial_\mu \mathbf{A_v}-  \partial_v\mathbf{A_\mu } \right) \cdot \mathbf{\sigma} -g^2 \left[ \mathbf{A}_{\mu }\cdot \mathbf{\sigma}, \mathbf{A}_v\cdot \mathbf{\sigma} \right] \\
.\end{align*}
Now we just need to calculate the final commutator,
\begin{align*}
    \left[ \mathbf{A}_{\mu }\cdot \mathbf{\sigma}, \mathbf{A}_v \cdot \mathbf{\sigma} \right] &= 
    \left( \mathbf{A}_{\mu }\cdot \mathbf{\sigma}\right)\cdot \left(    \mathbf{A}_v \cdot \mathbf{\sigma} \right) -
     \left(    \mathbf{A}_v \cdot \mathbf{\sigma} \right)\cdot\left( \mathbf{A}_{\mu }\cdot \mathbf{\sigma}\right) \\
     &= \mathbf{A_\mu }\cdot \mathbf{A_v} + \left( \mathbf{A_\mu }\times \mathbf{A_v} \right) \cdot i\mathbf{\sigma}  -
\left(    \mathbf{A }_v\cdot \mathbf{A}_{\mu } + \left( \mathbf{A_v }\times \mathbf{A_\mu } \right) \cdot i\mathbf{\sigma}  \right)\\
     &=  \left( \mathbf{A_\mu }\times \mathbf{A_v} \right) \cdot i\mathbf{\sigma}  -
\left(   \left( \mathbf{A_v }\times \mathbf{A_\mu } \right) \cdot i\mathbf{\sigma}  \right)\\
     &=2\left( \mathbf{A}_{\mu }\times \mathbf{A}_{v} \right) \cdot i\mathbf{\sigma}
.\end{align*}
we can insert this,
\begin{align*}
     \left[ D_\mu , D_v \right] &=  
     -ig\left(\partial_\mu \mathbf{A_v}-  \partial_v\mathbf{A_\mu } \right) \cdot \mathbf{\sigma} -g^2 \left[ \mathbf{A}_{\mu }\cdot \mathbf{\sigma}, \mathbf{A}_v\cdot \mathbf{\sigma} \right] \\
    &=  -ig\left(\partial_\mu \mathbf{A_v}-  \partial_v\mathbf{A_\mu } \right) \cdot \mathbf{\sigma} -g^2 2\left( \mathbf{A}_{\mu }\times \mathbf{A}_{v} \right) \cdot i\mathbf{\sigma} \\
    &=  -ig\left(\partial_\mu \mathbf{A_v}-  \partial_v\mathbf{A_\mu }  +2g\left( \mathbf{A}_{\mu }\times \mathbf{A}_{v} \right) \right)\cdot \mathbf{\sigma} = -ig\mathbf{F}_{\mu v}\cdot \mathbf{\sigma}
.\end{align*}
\begin{exercise}[2]
Argue that the transformation law of the covariant derivative in eq. (7) implies that
\begin{align*}
[D_\mu, D_\nu]\psi &\rightarrow V(x)[D_\mu, D_\nu]\psi, \tag{12}
\end{align*}
and show that this implies that
\begin{align*}
\mathbf{F}_{\mu\nu} \cdot \boldsymbol{\sigma} &\rightarrow V(x) \mathbf{F}_{\mu\nu} \cdot \boldsymbol{\sigma} V^\dagger(x). \tag{13}
\end{align*}
\end{exercise}
We have the following transformation law,
\[
D_\mu\Psi \to V\left( x \right) \left( D_\mu \Psi \right) 
.\] 
Applying this this to commutator,
\begin{align*}
    \left[ D_\mu , D_v \right] \Psi = D_\mu \left( D_v \Psi \right) - D_v\left( D_\mu \Psi \right)\to   D_\mu \left(V\left( x \right)  D_v \Psi \right) - D_v\left( V\left( x \right) D_\mu \Psi \right)
    &=  V\left( x \right) \left[ D_\mu , D_v \right] \Psi
.\end{align*} 
Lets show that this implies (13). From the previous exercise we know,
\[
\left[ D_\mu , D_v \right] = -ig \mathbf{F}_{\mu v}\cdot \mathbf{\sigma}
.\] 
And therefore,
\[
\left[ D_\mu , D_v \right]' = -ig \mathbf{F}_{\mu v}'\cdot \mathbf{\sigma}
.\] 
We can insert this,
\begin{align*}
    \left( -ig\mathbf{F}_{\mu  v}' \cdot  \mathbf{\sigma} \right) \Psi' &= V\left( x \right) \left( -ig \mathbf{F}_{\mu  v}\cdot \mathbf{\sigma} \right) \Psi\\
    \left( \mathbf{F}_{\mu  v}' \cdot  \mathbf{\sigma} \right) \Psi' &= V\left( x \right) \left( \mathbf{F}_{\mu  v}\cdot \mathbf{\sigma} \right) \Psi \\
    \left( \mathbf{F}_{\mu  v}' \cdot  \mathbf{\sigma} \right) V\left( x \right) \Psi &= V\left( x \right) \left( \mathbf{F}_{\mu  v}\cdot \mathbf{\sigma} \right) \Psi
.\end{align*}
This will have to hold for any field $\Psi$, so we can equate the operators,
\begin{align*}
    \left( \mathbf{F}_{\mu  v}' \cdot  \mathbf{\sigma} \right) V\left( x \right)  &= V\left( x \right) \left( \mathbf{F}_{\mu  v}\cdot \mathbf{\sigma} \right) \\
    \left( \mathbf{F}_{\mu  v}' \cdot  \mathbf{\sigma} \right) V\left( x \right) V^{-1}\left( x \right)  &= V\left( x \right) \left( \mathbf{F}_{\mu  v}\cdot \mathbf{\sigma} \right) V^{-1}\left( x \right) \\
    \left( \mathbf{F}_{\mu  v}' \cdot  \mathbf{\sigma} \right) &= V\left( x \right) \left( \mathbf{F}_{\mu  v}\cdot \mathbf{\sigma} \right) V\left( x \right)^{\dagger} \\
.\end{align*}
\begin{exercise}[3]
Using the transformation of eq. (13) to show that
\begin{align*}
\text{Tr} \left[ (\mathbf{F}^{\mu\nu} \cdot \boldsymbol{\sigma}) (\mathbf{F}_{\mu\nu} \cdot \boldsymbol{\sigma}) \right], \tag{14}
\end{align*}
is invariant.
\end{exercise}
We show this by applying the transformation, and asserting that the resulting is unchanged,
\begin{align*}
    \text{Tr}\left[\left( \mathbf{F^{\mu  v}}\cdot \mathbf{\sigma} \right) \left( \mathbf{F_{\mu v}}\cdot \mathbf{\sigma} \right) \right]' &=
    \text{Tr}\left[\left( \mathbf{F^{\mu  v}}'\cdot \mathbf{\sigma} \right) \left( \mathbf{F_{\mu v}}'\cdot \mathbf{\sigma} \right) \right] \\
&= \text{Tr}\left[\left(V\left( x \right)  \mathbf{F^{\mu  v}}\cdot \mathbf{\sigma}V\left( x \right)^{\dagger}  \right) \left(V\left( x \right)  \mathbf{F_{\mu v}}'\cdot \mathbf{\sigma}V\left( x \right) ^{\dagger} \right) \right] \\
&= \text{Tr}\left[\left(V\left( x \right)  \mathbf{F^{\mu  v}}\cdot \mathbf{\sigma}  \right)   \mathbf{F_{\mu v}}'\cdot \mathbf{\sigma}V\left( x \right) ^{\dagger}  \right] \\
.\end{align*}
The trace is invariant under circular shifts,
\begin{align*}
 \text{Tr}\left[\left(V\left( x \right)  \mathbf{F^{\mu  v}}\cdot \mathbf{\sigma}  \right)   \mathbf{F_{\mu v}}'\cdot \mathbf{\sigma}V\left( x \right) ^{\dagger}  \right] =
 \text{Tr}\left[  \mathbf{F^{\mu  v}}\cdot \mathbf{\sigma}     \mathbf{F_{\mu v}}'\cdot \mathbf{\sigma}V\left( x \right) ^{\dagger}  V\left( x \right)\right] =
 \text{Tr}\left[  \mathbf{F^{\mu  v}}\cdot \mathbf{\sigma}     \mathbf{F_{\mu v}}'\cdot \mathbf{\sigma}\right] 
.\end{align*}
\begin{exercise}[4]
Show that the trace in eq. (14) is equal to \( 2 \mathbf{F}^{\mu\nu} \cdot \mathbf{F}_{\mu\nu} \).
\end{exercise}
We do this by expanding the dot-product within the trace.
\[
\mathbf{F}^{\mu  v}\cdot \sigma =  \sum_\alpha \mathbf{F}^{\mu v}_\alpha \mathbf{\sigma}_\alpha
.\] 
We can insert these,
\[
\text{Tr}\left[\left( \mathbf{F}^{\mu  v}\cdot  \mathbf{ \sigma} \right) \left( \mathbf{F}_{\mu  v}\cdot \mathbf{\sigma} \right) \right] = \text{Tr}\left[\left( \mathbf{F}^{\mu v}\cdot \mathbf{F}_{\mu v} \right)\mathbb{I}_{2\times 2} + i\mathbf{\sigma} \cdot \left( \mathbf{F}^{\mu v}\times \mathbf{F}_{\mu  v} \right)  \right]
.\] 
Since everything here is index-wise, and we are implicitly performing sums, we can move the the field-strength tensors outside of the trace,
\[
= \left( \mathbf{F}^{\mu v}\cdot \mathbf{F}_{\mu  v} \right) \text{Tr}\left[\mathbb{I}_{2\times 2}\right] + i  \sum_k\text{Tr}\left[\mathbf{\sigma}_{k}\right]\left( \mathbf{F^{\mu v}}\times \mathbf{F}_{\mu v} \right) _k = 2\mathbf{F}^{\mu v}\cdot \mathbf{F}_{\mu v}
.\] 
\begin{exercise}[5]
Combine 3) and 4) to show that the Lagrangian in eq. (9) is invariant.
\end{exercise}
\end{document}
