\documentclass[working, oneside]{../../Preambles/tuftebook}
% Import xcolor and define some colors
\usepackage{{xcolor}}
\definecolor{{background}}{{HTML}}{{{background}}}
\definecolor{{foreground}}{{HTML}}{{{foreground}}}
\definecolor{{math}}{{HTML}}{{{color6}}}

%%%%%%%%%%%%%%%%%%%%%%%%%%%%%%%%%%%%%%%% IMPORTS %%%%%%%%%%%%%%%%%%%%%%%%%%%%%%%%%%%%%%%%
\documentclass[11pt,onesize,a4paper,titlepage]{article}

%%%%%%%%%%%%%%% Formatting %%%%%%%%%%%%%%% 
\usepackage[english]{babel}
\usepackage[utf8]{inputenc}
\usepackage{adjustbox}
\usepackage{geometry} % Margins
\usepackage{sectsty} % Custom Sections

%%%%%%%%%%%%%%% Font %%%%%%%%%%%%%%% 
\usepackage{Archivo}
\usepackage[T1]{fontenc}
\sffamily

%%%%%%%%%%%%%%% Graphics %%%%%%%%%%%%%%% 
\usepackage{fontawesome5} % Icons
\usepackage{graphicx} % Images
\usepackage[most]{tcolorbox} % Color Box
\usepackage{xcolor} % Colors
\usepackage{tikz} % For Drawing Shapes
%%%\usepackage{emoji} % For flags
\tcbuselibrary{breakable}
%%%\usepackage{academicons}

%%%%%%%%%%%%%%% Miscelanous %%%%%%%%%%%%%%% 
\usepackage{lipsum} % Lorem Ipsum
\usepackage{hyperref} % For Hyperlinks

%%%%%%%%%%%%%%% Colors %%%%%%%%%%%%%%% 
\definecolor{title}{HTML}{b5bff5} % Color of the title
\definecolor{bars}{HTML}{889af0} % Color of the title
\definecolor{backdrop}{HTML}{f2f2f2} % Color of the side column
\definecolor{lightgray}{HTML}{dfdfdf} % Color for the skill bars

%%% TU green: #639a00
%%% TU gray: #e6e6e6
%\definecolor{title}{HTML}{639a00} % Color of the title TU
%\definecolor{bars}{HTML}{889af0} % Color of the title TU

% \definecolor{backdrop}{HTML}{f2f2f2} % Color of the side column
\definecolor{backdrop}{HTML}{e6e6e6} % Color of the side column

\definecolor{subtitle}{HTML}{606060} % 


%%%%%%%%%%%%%%% Section Format %%%%%%%%%%%%%%% 
\sectionfont{                     
    \LARGE % Font size
    \sectionrule{0pt}{0pt}{-8pt}{1pt} % Rule under Section name
}

\subsectionfont{
    \Large % Font size
    \fontfamily{phv}\selectfont % Font family
    %\sectionrule{0pt}{0pt}{-8pt}{1pt} % Rule under Subsection name
    \sectionrule{5pt}{0pt}{0pt}{0pt} % Rule under Subsection name
}

%%%%%%%%%%%%%%% Margins and Headers %%%%%%%%%%%%%%%
\geometry{
  a4paper,
  left=7mm,
  right=7mm,
  bottom=10mm,
  top=10mm
}

\pagestyle{empty} % Empty Headers

\usepackage{marvosym}

% \renewcommand\qedsymbol{\CoffeeCup}

\usepackage{changepage}

\newenvironment{subexercise}[1]{%
    \begin{mdframed}[linewidth=0.5pt, linecolor=foreground, backgroundcolor=background, leftmargin=0cm, innerleftmargin=1em, innertopmargin=0pt, innerbottommargin=0pt, innerrightmargin=0pt, topline=false, rightline=false, bottomline=false]
    \par\noindent\textcolor{foreground}{\textbf{#1.}}\hspace{1em}\ignorespaces
}{%
    \par\addvspace{\baselineskip}\end{mdframed}\ignorespacesafterend
}
\newenvironment{solution}{%
    % \par\addvspace{\baselineskip}\noindent\makebox[\textwidth]{\textcolor{foreground}{\textbullet\hspace{1em}\textbullet\hspace{1em}\textbullet}}\par\addvspace{\baselineskip}
    \begin{mdframed}[linewidth=0.5pt, linecolor=foreground, backgroundcolor=background, rightmargin=0cm, innerleftmargin=0cm, innertopmargin=0pt, innerbottommargin=0pt, innerrightmargin=1em, topline=false, leftline=false, bottomline=false]
    \par\noindent\textcolor{foreground}{\textit{Solution.}}\hspace{1em}\ignorespaces
}{%
    \par\addvspace{\baselineskip}\noindent\hfill\textcolor{foreground}{\Coffeecup}\par\addvspace{\baselineskip}\end{mdframed}\ignorespacesafterend
}
% Exercise environment

\declaretheoremstyle[
    name= \textcolor{foreground}{Exercise},
    postheadspace = \newline,
    bodyfont = \normalfont\color{foreground},
    postheadhook={\textcolor{math}{\rule[.4ex]{\linewidth}{0.5pt}}\\},
    % numberwithin=chapter,
    mdframed={
        backgroundcolor = background,
        linecolor = foreground,
        linewidth = 0.5pt,
        rightline =  true,
        topline = true,
        bottomline = true,
        skipabove=20pt,
        skipbelow=20pt,
        innerleftmargin=15pt,
        innertopmargin=10pt,
        innerrightmargin=15pt,
        innerbottommargin=10pt}
    ]{exercise}
\declaretheorem[style=exercise,numbered=no]{exercise}

% \etocsetlevel{exercise}{2}

% \AtEndEnvironment{exercise}{%
%   \etoctoccontentsline{exercise}{\protect\numberline{\theexercise}}%
% }%
% \etocsetstyle{exercise}
% {}
% {}
% % this will be rendered like a non-numbered section, but we could have used
% % \numberline here also
% {\etocsavedsectiontocline{Exercise \etocnumber}{\etocpage}}
%     {}

% theorem environment

\declaretheoremstyle[
    name= \textcolor{foreground}{Theorem},
    postheadspace = \newline,
    bodyfont = \normalfont\color{foreground},
    postheadhook={\textcolor{math}{\rule[.4ex]{\linewidth}{1pt}}\\},
    mdframed={
        backgroundcolor = background,
        linecolor = foreground,
        linewidth = 1pt,
        rightline =  true,
        topline = true,
        bottomline = true,
        skipabove=20pt,
        skipbelow=20pt,
        innerleftmargin=15pt,
        innertopmargin=10pt,
        innerrightmargin=15pt,
        innerbottommargin=10pt}
    ]{theorem}
\declaretheorem[style=theorem,numbered=yes]{theorem}

\declaretheoremstyle[
    name= \textcolor{foreground}{Definition},
    postheadspace = \newline,
    bodyfont = \normalfont\color{foreground},
    postheadhook={\textcolor{math}{\rule[.4ex]{\linewidth}{1pt}}\\},
    mdframed={
        backgroundcolor = background,
        linecolor = foreground,
        linewidth = 1pt,
        rightline =  true,
        topline = true,
        bottomline = true,
        skipabove=20pt,
        skipbelow=20pt,
        innerleftmargin=15pt,
        innertopmargin=10pt,
        innerrightmargin=15pt,
        innerbottommargin=10pt}
    ]{definition}
\declaretheorem[style=definition,numbered=yes]{definition}
% Example environment

\declaretheoremstyle[
name= \quad \underline{Proof:},
     headfont = \bfseries\sffamily,
     postheadspace = \newline,
     % notebraces = \bfseries{(}{)a},
     headpunct = {},
     bodyfont = ,
     postheadhook={\textcolor{foreground}{\rule[0.4ex]{\linewidth}{0pt}}\\},
     qed=\qedsymbol,
    % spacebelow = 10pt,
    mdframed={
  backgroundcolor = background,
  linecolor = foreground,
  linewidth = 1pt,
  skipabove=10pt,
  skipbelow=10pt,
  rightline = false,
  topline = false,
  leftline = false,
  bottomline = false,
  innerleftmargin=15pt,
  innertopmargin=15pt,
  innerrightmargin=15pt,
  innerbottommargin=15pt}
]{pro}
    % \declaretheorem[style=pro,numbered=no]{Proof}

\declaretheoremstyle[
name= \quad \underline{\textcolor{foreground}{Example}},
     headfont = \bfseries\sffamily,
     postheadspace = \newline,
     % notebraces = \bfseries{(}{)a},
     headpunct = {},
     bodyfont = \normalfont\color{foreground},
     postheadhook={\textcolor{foreground}{\rule[0.4ex]{\linewidth}{0pt}}\\},
     % spacebelow = 10pt,
    mdframed={
  backgroundcolor = background,
  linecolor = foreground,
  linewidth = 1pt,
  skipabove=10pt,
  skipbelow=10pt,
  rightline = false,
  topline = false,
  leftline = false,
  bottomline = false,
  innerleftmargin=15pt,
  innertopmargin=15pt,
  innerrightmargin=15pt,
  innerbottommargin=15pt}
]{ex}
\declaretheorem[style=ex,numbered=no]{example}

\declaretheoremstyle[
     name=,
     headfont = \bfseries\sffamily,
     notebraces = \bfseries{},
     headpunct = { -},
     bodyfont = \color{foreground}\normalfont,
     % postheadhook={\textcolor{black}{\rule[.4ex]{\linewidth}{0.2pt}}\\},
    % spacebelow = 10pt,
    mdframed={
  backgroundcolor = background,
  linecolor = foreground,
  linewidth = 1pt,
  skipabove=0pt,
  skipbelow=0pt,
  innerleftmargin=10pt,
  innertopmargin=10pt,
  innerrightmargin=10pt,
  innerbottommargin=10pt,
  rightline = false,
  topline = false,
  leftline = false,
  bottomline = true}
]{subexercise}
% \declaretheorem[style=subexercise,numbered=no]{subexercise}

\declaretheoremstyle[
     name= \color{losning}Løsning,
     headfont = \bfseries\sffamily,
     notebraces = \bfseries{},
     postheadspace = \newline,
     headpunct = {:},
     bodyfont = \normalfont,
     % qed = ,
     % postheadhook={\textcolor{black}{\rule[.4ex]{\linewidth}{0.2pt}}\\},
    % spacebelow = 10pt,
    mdframed={
  backgroundcolor = background,
  linecolor = losning!75,
  linewidth = 1pt,
  skipabove=0pt,
  skipbelow=10pt,
  innerleftmargin=10pt,
  innertopmargin=10pt,
  innerrightmargin=10pt,
  innerbottommargin=10pt,
  leftline = false,
  rightline = true,
  topline = false,
  bottomline = true}
]{solution}

\newenvironment{SimpleBox}[1]{%
  \begin{mdframed}%
    \noindent\textbf{#1}\\[1ex]
}{%
  \end{mdframed}%
}


\begin{document}
\let\cleardoublepage\clearpage
\thispagestyle{fancy}
\chapter{10 - LSZ reduction formulas}

Calculating $S$-matrix elements is essential to quantum field theory since they can be used to find cross-section and decay rates. In general one can write the transition of an $m$-particle in-state to a $n$ particle out-state as
\begin{align*}
S_{fi}^\dagger = \langle \mathbf{q}_1, \dots, \mathbf{q}_n; \text{out} | \mathbf{p}_1, \dots, \mathbf{p}_m; \text{in} \rangle. \tag{87}
\end{align*}

\begin{exercise}[1]
Argue that this definition is equivalent to the one in Eq. (52). Note that the above definition is the complex conjugate.
\end{exercise}

However, this definition is quite abstract, and not immediately useful for practical calculations. We therefore derive the Lehmann-Symanzik-Zimmermann (LSZ) reduction formulas, which connects the $S$ matrix element to the $n$-particle Green's function (or $n$-point function)
\begin{align*}
\Delta^{(n)}(x_1, \dots, x_n) = \langle 0 | T \left[ \phi(x_1) \dots \phi(x_n) \right] | 0 \rangle. \tag{88}
\end{align*}

In order to simplify the calculations we consider first the transition amplitude between two single particle states ($\mathbf{q}_{\text{out}} | \mathbf{p}_{\text{in}}$) and then generalize to more particles. We use the field expansion of an real scalar field in Eq. (78).

\begin{exercise}[2]
Show that the inversion formulas for the creation and annihilation operators are
\begin{align*}
a_\mathbf{p}^\dagger &= \frac{-i}{\sqrt{2}} \int d^3x e^{-ipx} \left[ \frac{\pi(x)}{\sqrt{\omega_\mathbf{p}}} + i \sqrt{\omega_\mathbf{p}} \phi(x) \right] = \frac{-i}{\sqrt{2}} \int \frac{d^3x}{\sqrt{\omega_\mathbf{p}}} e^{-ipx} \overleftrightarrow{\partial_0} \phi(x), \tag{89a} \\
a_\mathbf{p} &= \frac{+i}{\sqrt{2}} \int d^3x e^{+ipx} \left[ \frac{\pi(x)}{\sqrt{\omega_\mathbf{p}}} - i \sqrt{\omega_\mathbf{p}} \phi(x) \right] = \frac{+i}{\sqrt{2}} \int \frac{d^3x}{\sqrt{\omega_\mathbf{p}}} e^{+ipx} \overleftrightarrow{\partial_0} \phi(x), \tag{89b}
\end{align*}
where we have defined the operator
\begin{align*}
A \overleftrightarrow{\partial_0} B = A (\partial_0 B) - (\partial_0 A) B. \tag{90}
\end{align*}
\end{exercise}

\begin{exercise}[3]
Argue that we can write
\begin{align*}
\langle \mathbf{q}_{\text{out}} | \mathbf{p}_{\text{in}} \rangle = \lim_{x_0 \to -\infty} \left( -i \int d^3x e^{-ipx} \langle \mathbf{q}_{\text{out}} | \left[ \pi(x) + i \omega_\mathbf{p} \phi(x) \right] | 0_{\text{in}} \rangle \right). \tag{91}
\end{align*}
\end{exercise}
\begin{solution}
Lets work on the right-hand side,
\begin{align*}
    &= \lim_{x_0 \to -\infty} \left( -i\int d^{3}x e^{-ipx}\left<q_{out} \right|\left[ \pi\left( x \right) + i\omega_p \phi \left( x \right) \right] \left|0_{in} \right> \right)  \\
    &= \lim_{x_0 \to -\infty} \left( -i\left<q_{out} \right|\left[ \int d^{3}x e^{-ipx}\pi\left( x \right) + i\omega_p \phi \left( x \right) \right] \left|0_{in} \right> \right)  \\
    &= \lim_{x_0 \to -\infty} \left( -i\left<q_{out} \right|i\sqrt{2\omega_p} a^\dagger_{p} \left|0_{in} \right> \right)  = \left<q_{out} \mid p_{in} \right>
.\end{align*}
Where we have used the inversion fomula for the creation operator and the assumption that,
\[
\lim_{x_0 \to -\infty} a_p^\dagger = a_{p_{in}}^\dagger
.\] 
\end{solution}

\begin{exercise}[4]
Using the following identity from the fundamental theorem of calculus
\begin{align*}
\lim_{t \to -\infty} F(t) = \lim_{t \to \infty} F(t) - \int_{-\infty}^{\infty} dt \partial_t F(t), \tag{92}
\end{align*}
show that
\begin{align*}
\langle \mathbf{q}_{\text{out}} | \mathbf{p}_{\text{in}} \rangle = \sqrt{2 \omega_\mathbf{p}} \sqrt{2 \omega_\mathbf{q}} \langle 0_{\text{out}} | a_{\mathbf{q}_{\text{out}}} a_{\mathbf{p}_{\text{out}}}^\dagger | 0_{\text{in}} \rangle + i \int d^4x \partial_0 e^{-ipx} \langle \mathbf{q}_{\text{out}} | \left[ \pi(x) + i \omega_\mathbf{p} \phi(x) \right] | 0_{\text{in}} \rangle. \tag{93}
\end{align*}
\end{exercise}
Let,
\[
F\left( x_0 \right)  =  -i\int d^{3}x e^{-ipx}\left<q_{out} \right|\left[ \pi\left( x \right) + i\omega_p \phi \left( x \right) \right] \left|0_{in} \right> 
.\] 
Lets calculate the two terms,
\begin{align*}
    -\int_{-\infty}^{\infty} dt \partial_0 F\left( x_0 \right) &= -\int_{-\infty}^{\infty} dt \partial_0 -i\int d^{3}x e^{-ipx}\left<q_{out} \right|\left[ \pi\left( x \right) + i\omega_p \phi \left( x \right) \right] \left|0_{in} \right> \\
                                                               &=i\int d^{4}x \partial_0e^{-ipx}\left<q_{out} \right|\left[ \pi\left( x \right) + i\omega_p \phi \left( x \right) \right] \left|0_{in} \right>   
.\end{align*}
And the other term,
\begin{align*}
    \lim_{x_0 \to -\infty} F\left( x_0 \right) &= -i \lim_{x_0 \to -\infty} \int d^{3}x e^{-ipx}\left<q_{out} \right|\left[ \pi\left( x \right) + i\omega_p \phi \left( x \right) \right] \left|0_{in} \right>  \\
    &= \left<q_{out} \right| a_{p_{out}}^\dagger \sqrt{2\omega_p} \left|0_{in} \right> \\
    &= \left<0_{out} \right|a_{q_{out}}\sqrt{2\omega_{q}}  a_{p_{out}}^\dagger \sqrt{2\omega_p} \left|0_{in} \right> 
.\end{align*}
\begin{exercise}[5]
We are only interested in interactions. Argue that we can throw away the first term of the above equation.
\end{exercise}

\begin{exercise}[6]
Show that by explicitly performing the time derivative we obtain
\begin{align*}
\langle \mathbf{q}_{\text{out}} | \mathbf{p}_{\text{in}} \rangle = i \int d^4x e^{-ipx} (\partial_\mu \partial^\mu + m^2) \langle \mathbf{q}_{\text{out}} | \phi(x) | 0_{\text{in}} \rangle. \tag{94}
\end{align*}
\end{exercise}
\begin{solution}
lets go ahead and begin,
\begin{align*}
  \left<\mathbf{q}_{out} \mid \mathbf{p}_{in} \right> &= i \int d^4x \partial_0 e^{-ipx} \langle \mathbf{q}_{\text{out}} | \left[ \pi(x) + i \omega_\mathbf{p} \phi(x) \right] | 0_{\text{in}} \rangle.\\
  &=   i \int d^4x \left(   \left( -i\omega_p \right)  e^{-ipx} \langle \mathbf{q}_{\text{out}} | \left[ \partial_0\phi \left( x \right)  + i \omega_\mathbf{p} \phi(x) \right] | 0_{\text{in}} \rangle +  e^{-ipx} \langle \mathbf{q}_{\text{out}} | \left[ \partial^2_{0}\phi \left( x \right)  + i \omega_\mathbf{p} \partial_0\phi(x) \right] | 0_{\text{in}} \rangle \right)
.\end{align*}
Collect the terms, and note that the $\partial_0\phi \left( x \right) $ terms cancel,
\begin{align*}
  &=   i \int d^4x \left( e^{-ipx} \langle \mathbf{q}_{\text{out}}  |\left[\omega_\mathbf{p}^2 \phi(x) +   \partial^2_{0}\phi \left( x \right) \right] | 0_{\text{in}} \rangle \right) \\
  &=   i \int d^4x \left( e^{-ipx} \langle \mathbf{q}_{\text{out}}  |\left[\left( \mathbf{p}^2 + m^2 \right)  \phi(x) +   \partial^2_{0}\phi \left( x \right) \right] | 0_{\text{in}} \rangle \right) \\
  &=   i \int d^4x \left( e^{-ipx} \langle \mathbf{q}_{\text{out}}  |\left[  m^2\phi(x) +   \partial^2_{0}\phi \left( x \right) \right] | 0_{\text{in}} \rangle +   \langle \mathbf{q}_{\text{out}}  |\left[\left( \mathbf{p}^2 e^{-ipx} \right)  \phi(x) \right] | 0_{\text{in}} \rangle  \right) \\
  &=   i \int d^4x \left( e^{-ipx} \langle \mathbf{q}_{\text{out}}  |\left[  m^2\phi(x) +   \partial^2_{0}\phi \left( x \right) \right] | 0_{\text{in}} \rangle +   \langle \mathbf{q}_{\text{out}}  |\left[\left( -\nabla^2  e^{-ipx} \right)  \phi(x) \right] | 0_{\text{in}} \rangle  \right) \\
  &=   i \left(   \int d^4x  e^{-ipx} \langle \mathbf{q}_{\text{out}}  |\left[  m^2\phi(x) +   \partial^2_{0}\phi \left( x \right) \right] | 0_{\text{in}} \rangle +   \langle \mathbf{q}_{\text{out}}  |\left[\int d^4x\left( -\nabla^2  e^{-ipx} \right)  \phi(x) \right] | 0_{\text{in}} \rangle   \right)
.\end{align*}
Lets take a look at integral on the right by itself for a moment, we will be doing integration by parts in three dimensions, to move the laplacian onto the field.
\begin{align*}
    \int d^4x\left( -\nabla^2  e^{-ipx} \right)  \phi(x)  &= \int dt \int d^3x\left( -\nabla^2  e^{-ipx} \right)  \phi(x)  \\
    &= -\int dt \left( \oint_{\Omega} dS \nabla e^{-ipx} \phi \left( x \right) \hat{n}- \int d^{3}x \nabla e^{-ipx}\cdot \nabla \phi \left( x \right) \right) \\
    &= \int dt \left(  \int d^{3}x \nabla e^{-ipx}\cdot \nabla \phi \left( x \right) \right) \\
    &= \int dt \left( \oint_{\Omega} dS  e^{-ipx} \nabla \phi \left( x \right) \hat{n}- \int d^{3}x e^{-ipx}\cdot \nabla^2 \phi \left( x \right) \right) \\
    &= -\int dt \left(  \int d^{3}x e^{-ipx}\cdot \nabla^2 \phi \left( x \right) \right) \\
    &= - \int d^{4}x e^{-ipx}\cdot \nabla^2 \phi \left( x \right) 
.\end{align*}
We can insert this, move the integral out again and collect the terms touch $\phi \left( x \right) $,
\begin{align*}
    i  \int d^{4}x \left<\mathbf{q}_{out} \right| \left[ \left( m^2 + \partial_0^2 - \nabla ^2 \right) \phi \left( x \right)  \right] \left|0_{in} \right> =i  \int d^{4}x \left<\mathbf{q}_{out} \right| \left[ \left( m^2 +\partial_{\mu }\partial^{\mu } \right) \phi \left( x \right)  \right] \left|0_{in} \right>
.\end{align*}
\end{solution}
\begin{exercise}[7]
Using the same procedure on the out-state show that we obtain
\begin{align*}
\langle \mathbf{q}_{\text{out}} | \mathbf{p}_{\text{in}} \rangle = i \int d^4x e^{-ipx} (\Box_x + m^2) \\ \left[ \sqrt{2 \omega_\mathbf{q}} \langle 0_{\text{out}} | a_{\mathbf{q}_{\text{in}}} \phi(x) | 0_{\text{in}} \rangle + i \int d^4y \partial_{y_0} e^{iqy} \langle 0_{\text{out}} | \left[ \pi(y) - i \omega_\mathbf{q} \phi(y) \right] \phi(x) | 0_{\text{in}} \rangle \right], \tag{95}
\end{align*}
where we have used $\Box = \partial_\mu \partial^\mu$ with subscript $x$ to denote to which variable the derivation is with respect to.
\end{exercise}
\begin{solution}
We can write,
\begin{align*}
    \left<\mathbf{q}_{out} \mid \mathbf{p}_{in} \right> &= \left<0_{out} \right|\sqrt{2 \omega_{q}} a_{q_{out}} \left|\mathbf{p}_{in} \right> \\
    &= \lim_{y_0 \to \infty} \sqrt{2\omega_q} \int d^{3}y \left< 0_{out} \right| \left( \frac{i}{\sqrt{2} } \right) e^{ipy} \left[ \frac{\pi\left( y \right) }{\sqrt{\omega_p} } - i\sqrt{\omega_p} \phi \left( y \right)  \right] \left|\mathbf{p}_{in} \right> \\
    &= \lim_{y_0 \to \infty}  \int d^{3}y\, i e^{ipy} \left< 0_{out} \right|  \left[ \pi\left( y \right)  - i\omega_p \phi \left( y \right)  \right] \left|\mathbf{p}_{in} \right> \\
.\end{align*}
We will use the identify from above again,
\[
\lim_{x_0 \to \infty} F\left( t \right) = \lim_{x_0 \to -\infty} F\left( t \right) + \int_{-\infty}^{\infty} dt \partial_t F\left( t \right) 
.\] 
And we get,
\begin{align*}
    &= \sqrt{2\omega_{q}} \left<0_{out} \right| a_{\mathbf{q}_{in}} \left|\mathbf{p}_{in} \right>+ i \int d^{4}y\, \partial_{y_{o}}e^{ipy} \left< 0_{out} \right|  \left[ \pi\left( y \right)  - i\omega_p \phi \left( y \right)  \right] \left|\mathbf{p}_{in} \right> \\
.\end{align*}
The calculation we did before was essentially just a rewrite of $\left|\mathbf{p}_{in} \right>$, so we can do this rewrite again to obtain the desired expression.
\begin{align*}
\langle \mathbf{q}_{\text{out}} | \mathbf{p}_{\text{in}} \rangle = i \int d^4x e^{-ipx} (\Box_x + m^2) \\ \left[ \sqrt{2 \omega_\mathbf{q}} \langle 0_{\text{out}} | a_{\mathbf{q}_{\text{in}}} \phi(x) | 0_{\text{in}} \rangle + i \int d^4y \partial_{y_0} e^{iqy} \langle 0_{\text{out}} | \left[ \pi(y) - i \omega_\mathbf{q} \phi(y) \right] \phi(x) | 0_{\text{in}} \rangle \right]
.\end{align*}
\end{solution}

\begin{exercise}[8]
Argue that we are free to insert a time-ordering product in both terms and argue that this means that we can drop the first term, which leaves
\begin{align*}
\langle \mathbf{q}_{\text{out}} | \mathbf{p}_{\text{in}} \rangle = i^2 \int d^4x d^4y e^{-ipx} (\Box_x + m^2) \partial_{y_0} e^{iqy} \langle 0_{\text{out}} | T \left[ \left( \pi(y) - i \omega_\mathbf{q} \phi(y) \right) \phi(x) \right] | 0_{\text{in}} \rangle. \tag{96}
\end{align*}
\end{exercise}
\begin{solution}
We want the fields to interact in the right order, so we are free to insert time-orering operators.. Lets first rewrite the left term,
\begin{align*}
    \left<0_{out} \right|a_{\mathbf{q}_{in}} \phi \left( x \right)  \left|0_{in} \right> &= \lim_{y_0 \to \infty} \left<0_{out} \right| a_{\mathbf{q}}\phi \left( x \right) \left|0_{out} \right> \\
    &=\lim_{y_0 \to -\infty} \left<0_{out} \right| T\left[   a_{\mathbf{q}}\phi \left( x \right) \right]\left|0_{out} \right> \\
    &=\lim_{y_0 \to -\infty} \left<0_{out} \right|  \theta \left( y_0 - x_0 \right)   a_{\mathbf{q}}\phi \left( x \right) + \theta \left( x_0 - y_0 \right) \phi \left( x \right)a_{\mathbf{q}}\left|0_{out} \right> \\
    &=\lim_{y_0 \to -\infty} \left<0_{out} \right|  \theta \left( y_0 - x_0 \right)   a_{\mathbf{q}}\phi \left( x \right)\left|0_{out} \right> = 0
.\end{align*}

\end{solution}

\begin{exercise}[9]
Do the derivative with respect to time $y^0$, as in part 6), to arrive at
\begin{align*}
\langle \mathbf{q}_{\text{out}} | \mathbf{p}_{\text{in}} \rangle = i^2 \int d^4x d^4y e^{-i(px - qy)} (\Box_x + m^2) (\Box_y + m^2) \langle 0_{\text{out}} | T \left[ \phi(y) \phi(x) \right] | 0_{\text{in}} \rangle. \tag{97}
\end{align*}
\end{exercise}
\begin{solution}
We start by splitting up the time-ordering operator,
\begin{align*}
    \left<\mathbf{q}_{out} \mid \mathbf{p}_{in} \right> = i^2 \int d^{4}xd^{4}y e^{ipx} \left( \Box_x + m^2 \right) \partial_{y_{0}}e^{iqy}\left<0_{out} \right| T \left[ \pi\left( y \right) \phi \left( y \right)  \right] - i\omega_qT\left[ \phi \left( y \right) \phi \left( x \right)  \right] \left|0_{in} \right>
.\end{align*}
For now we will just look at the $y$-part of the integral, we'll leave out the bra-ket, and just do the derivation, and then fetch the integral when we need it.
\begin{align*}
&= \partial_{y_{0}}e^{iqy} (T \left[\partial_{y_0} \phi\left( y \right) \phi \left( y \right)  \right] - i\omega_qT\left[ \phi \left( y \right) \phi \left( x \right)  \right])   \\
&= \partial_{y_{0}}e^{iqy} (\partial_{y_0}T \left[ \phi\left( y \right) \phi \left( y \right)  \right] - i\omega_qT\left[ \phi \left( y \right) \phi \left( x \right)  \right])   \\
&= \partial_{y_{0}}e^{iqy}\partial_{y_0}T \left[ \phi\left( y \right) \phi \left( y \right)  \right] - \partial_{y_{0}}e^{iqy}i\omega_qT\left[ \phi \left( y \right) \phi \left( x \right)  \right]   \\
&= i\omega_qe^{iqy}\partial_{y_0}T \left[ \phi\left( y \right) \phi \left( y \right)  \right] + e^{iqy}\partial^2_{y_0}T\left[ \phi \left( y \right) \phi \left( x \right)  \right]   \\
&- i\omega_{q} e^{iqy}\partial_{y_0}T \left[ \phi\left( y \right) \phi \left( y \right)  \right]  + \omega_q^2 e^{iqy}T\left[ \phi \left( y \right) \phi \left( x \right)  \right]   \\
&=  e^{iqy}\partial^2_{y_0}T\left[ \phi \left( y \right) \phi \left( x \right)  \right] + \omega_q^2 e^{iqy}T\left[ \phi \left( y \right) \phi \left( x \right)  \right]   \\
&=  e^{iqy}\partial^2_{y_0}T\left[ \phi \left( y \right) \phi \left( x \right)  \right] + \left( p^2 + m^2 \right)  e^{iqy}T\left[ \phi \left( y \right) \phi \left( x \right)  \right]   \\
&=  e^{iqy}\partial^2_{y_0}T\left[ \phi \left( y \right) \phi \left( x \right)  \right] + \left( -\nabla ^2 + m^2 \right)  e^{iqy}T\left[ \phi \left( y \right) \phi \left( x \right)  \right]   \\
.\end{align*}
At this point we have something that looks very amenable, we can do integration by parts to move things around, and we'll end up with what we need.
\end{solution}

\begin{exercise}[10]
Using the method above argue that for a general transition, not involving spectators, the $S$ matrix element is
\begin{align*}
    S_{fi} = \langle \mathbf{q}_1, \dots, \mathbf{q}_n; \text{out} | \mathbf{p}_1, \dots, \mathbf{p}_m; \text{in} \rangle &= i^{n+m} \int \prod_{i=1}^m d^4x_i e^{-ip_i x_i} (\Box_{x_i} + m^2) \\ \int \prod_{j=1}^n d^4y_j e^{ip_j y_j} (\Box_{y_j} + m^2) &\langle 0_{\text{out}} | T \left[ \phi(y_1) \dots \phi(y_n) \phi(x_1) \dots \phi(x_m) \right] | 0_{\text{in}} \rangle. \tag{98}
\end{align*}
\end{exercise}

\begin{exercise}[11]
Consider the Fourier transform of the $n$-particle Green's function
\begin{align*}
\Delta^{(n)}(k_1, \dots, k_n) = \int d^4x_1 \dots d^4x_n e^{i(k_1 x_1 + \dots + k_n x_n)} \langle 0 | T \left[ \phi(x_1) \dots \phi(x_n) \right] | 0 \rangle, \tag{99}
\end{align*}
and show that the LSZ reduction formula in momentum space takes the form
\begin{align*}
S_{fi} = (-i)^{n+m} \prod_{i=1}^m (p_i^2 - m^2) \prod_{j=1}^n (q_j^2 - m^2) \Delta^{(n+m)}(-p_1, \dots, -p_n, q_1, \dots, q_m). \tag{100}
\end{align*}
Up to a phase factor.
\end{exercise}
\end{document}
