\documentclass[working, oneside]{../../Preambles/tuftebook}
% Import xcolor and define some colors
\usepackage{xcolor}
\definecolor{background}{HTML}{ffffff}
\definecolor{foreground}{HTML}{000000}
\definecolor{math}{HTML}{000000}

%%%%%%%%%%%%%%%%%%%%
%% SUPER PREAMBLE %%
%%%%%%%%%%%%%%%%%%%%

% \usepackage[utf8]{inputenc}
\usepackage[T1]{fontenc} % Fonts and stuff
\usepackage{amsmath, amsfonts, mathtools, amsthm, amssymb} % math

\usepackage{fancyhdr} % Header, Footer etc.
\usepackage{adforn}
\usepackage{efbox}
\usepackage{lastpage}
\usepackage{marvosym}
\usepackage{pict2e}
\usepackage{caption}
\usepackage{wrapfig}
\usepackage{graphicx}
\usepackage{sidecap}
% \usepackage{mathpazo} 
% \usepackage{cmbright}
\usepackage{mathptmx}

\usepackage[
    sorting=nyt,
    style=alphabetic
]{biblatex}
\addbibresource{references.bib}
\usepackage[noabbrev]{cleveref}

\pagestyle{fancy}
\fancyhead[R]{}
\fancyhead[L]{}
\fancyfoot[C]{\efbox[margin = 10pt,
                    topline = false,
                    leftline = false,
                    rightline = false,
                    backgroundcolor = background,
                    linewidth = 1pt,
                    linecolor = foreground]{\thepage\ of \pageref{LastPage}}}
% \fancyfoot[C]{\color{foreground} \thepage}

% \renewcommand{\headrule}{%
% 	\hrulefill
% }
% \renewcommand{\footrulewidth}{0pt}
\renewcommand{\headrulewidth}{0pt}

% \setlength{\headheight}{15pt}
% \setlength{\footheight}{15pt}

%% Margin Control %%

% \def\changemargin#1#2{\list{}{\rightmargin#2\leftmargin#1}\item[]}
% \let\endchangemargin=\endlist


%%%%%%%%%%%%%%%%%%%%%%%%%%%%%%%%%%%%%%%%%%%%%%%%%%%%%%%%%%

% figure support

\usepackage{import}
\usepackage{transparent}


\newcommand{\incfig}[2][1]{%
    \def\svgwidth{#1\columnwidth}
    \import{figures/}{#2.pdf_tex}
}

% \pdfsuppresswarningpagegroup=1

%%%%%%%%%%%%%%%%%%%%%%%%%%%%%%%%%%%%%%%%%%%%%%%%%%%%%%%%%%

\usepackage{tikzsymbols} % Symbols
\usepackage[framemethod=TikZ]{mdframed} % Boxes around theorem environments
\usepackage{thmtools}




% \everymath{\color{math}}
% \everydisplay{\color{math}}
% \def\m@th{\normalcolor\mathsurround\z@}

\color{foreground}

	% \end{changemargin}
	% } 

 % \newenvironment{subexercise}[1]
 % {\noindent
	 % \textbf{(#1)} \quad \adforn{10} \quad \em
 % }{}

% Mathematical typesetting stuff.

 % \newcommand{\dd}{\mathrm{\textbf{d}}}

 % Change font

% \usepackage{tgadventor}
% \usepackage{cmbright}
% \usepackage{bm}

% \usepackage{microtype} % Microtypography
% \usepackage{fontspec}% Hyperlinks
% \usepackage{fouriernc}

% \def\MT@set@inh@list#1#2{%
%   \MT@ifempty\MT@inh@feat{%
%     \MT@map@clist@c\MT@features{\begingroup % <--
%       \MT@ifstreq{##1}{tr}\relax{\MT@declare@char@inh{##1}{#1}{#2}}%
%     \endgroup}% <--
%   }{%
%     \MT@map@clist@c\MT@inh@feat{\begingroup % <--
%       \KV@@sp@def\@tempa{##1}%
%       \MT@ifempty\@tempa\relax{%
%         \edef\@tempa{\csname MT@rbba@\@tempa\endcsname}%
%         \MT@ifstreq\@tempa{tr}\relax{%
%           \MT@exp@one@n\MT@declare@char@inh{\@tempa}{#1}{#2}}}%
%     \endgroup}% <--
%   }%      
% \DeclareCaptionFormat{custom}{\bfseries#1#2\itshape#3}%   \MT@end@catcodes
\DeclareCaptionFormat{custom}{\bfseries\itshape#1#2\normalfont\small#3}
\captionsetup{
    format=custom,
    labelsep=space,
    width=\textwidth, % Set the caption width to be 80% of the text width
    % justification=jusitified, % Center-align the caption
    % font=it % Italicize the caption text
}
% }
\usepackage{setspace}
\DeclareCaptionFormat{margin}{\small\bfseries#1#2#3}

\usepackage{xparse} % For advanced command definitions with optional arguments

\NewDocumentCommand{\marginfig}{O{0cm} m m m}{%
  % #1 = optional padding (default 0cm), #2 = filename, #3 = label, #4 = caption
  \marginpar{%
    \includegraphics[width=\marginparwidth]{#2}%
    \captionsetup{format=custom, labelsep=space, width=\marginparwidth, justification=raggedright, font=small}
    \captionof{figure}{#4}%
    \label{fig:#3}%
    % \rule{\marginparwidth}{0.4pt} % Adds a line below the caption
    \vspace{#1} % Adds the specified padding below the caption
  }%
}

\NewDocumentCommand{\maintextfig}{O{0cm} m m m}{
  % #1 = optional vertical adjustment for the caption (default 0cm)
  % #2 = filename for the figure
  % #3 = label for the figure
  % #4 = caption text

  % Place the figure in the text
  \begin{figure}[htbp]
    \centering
    \includegraphics[width=\textwidth]{#2}
    \marginnote{\captionsetup{format=custom, labelsep=space, width=\marginparwidth, justification=fill, font={stretch=1}}
        \captionof{figure}[#4]{#4}\label{fig:#3}}[#1]
    \label{fig:#3}
  \end{figure}

  % Place the caption in the margin
  % \marginnote{\captionsetup{format=custom, labelsep=space, width=\marginparwidth, justification=raggedright, font={stretch=1}}
  %   \captionof{figure}[#4]{#4}\label{fig:#3}}[#1]
}
% \NewDocumentCommand{\marginfig}{m m m}{
%   % #1 = filename, #2 = label, #3 = caption
%   \begin{wrapfigure}{r}{5cm} % "r" for right side, and "5cm" for the width of the figure
%       \centering
%       \includegraphics[width=5cm]{#1}
%      \captionsetup{format=custom, labelsep=space, width=6cm, justification=raggedright, font={stretch=1}}
%       \captionof{figure}{#3}%
%       \label{fig:#2}
%   \end{wrapfigure}
% }{}
% \newcommand{\marginfig}[3]{%
%   \marginpar{%
%     \includegraphics[width=\marginparwidth]{#1}%
%     \captionsetup{format=custom, labelsep=space, width=\marginparwidth, justification=raggedright, font={stretch=1}}
%     \captionof{figure}{\fontsize{11pt}{11pt}\selectfont #3}%
%     \label{fig:#2}%
%     % \rule{\marginparwidth}{0.4pt}
%   }%
% }
% \newcommand{\marginfig}[4][0pt]{%
%   \marginpar{%
%     \raisebox{#1}{%
%       \includegraphics[width=\marginparwidth]{#2}%
%       \captionsetup{format=margin, labelsep=space, justification=raggedright}
%       \captionof{figure}{#4}%
%       \label{fig:#3}%
%     }%
%   }%
% }
% \newcommand{\marginfig}[2][0pt]{%
%   \marginpar{\raisebox{#1}{%
%       \includegraphics[width=1.0\marginparwidth]{#2}
%       % \label{fig:#3}
%       % \caption{#4}
%       % \parbox{\marginparwidth}{\smaller \textbf{figure:} #3}%
%   }}%
% }
\newcommand{\margintext}[2][0pt]{%
  \marginpar{\raisebox{#1}{%
    \parbox{\marginparwidth}{\smaller \textbf{figure:} #2}%
    }}%
}

\newcommand{\marginmath}[2][0pt]{%
  \marginpar{\raisebox{#1}{%
    \parbox{1.2\marginparwidth}{#2}%
    }}%
}
\pagecolor{background}
\usepackage{listings}
\definecolor{commentsColor}{rgb}{0.497495, 0.497587, 0.497464}
\definecolor{keywordsColor}{rgb}{0.000000, 0.000000, 0.635294}
\definecolor{stringColor}{rgb}{0.558215, 0.000000, 0.135316}
\renewcommand*\ttdefault{txtt}
\lstset{
  basicstyle=\ttfamily\small,                   % the size of the fonts that are used for the code
  breakatwhitespace=false,                      % sets if automatic breaks should only happen at whitespace
  breaklines=true,                              % sets automatic line breaking
  frame=tb,                                     % adds a frame around the code
  commentstyle=\color{commentsColor}\textit,    % comment style
  keywordstyle=\color{keywordsColor}\bfseries,  % keyword style
  stringstyle=\color{stringColor},              % string literal style
  numbers=left,                                 % where to put the line-numbers; possible values are (none, left, right)
  numbersep=5pt,                                % how far the line-numbers are from the code
  numberstyle=\tiny\color{commentsColor},       % the style that is used for the line-numbers
  showstringspaces=false,                       % underline spaces within strings only
  tabsize=2,                                    % sets default tabsize to 2 spaces
  language=Scala
}


\usepackage{marvosym}

% \renewcommand\qedsymbol{\CoffeeCup}

\usepackage{changepage}

\newenvironment{subexercise}[1]{%
    \begin{mdframed}[linewidth=0.5pt, linecolor=foreground, backgroundcolor=background, leftmargin=0cm, innerleftmargin=1em, innertopmargin=0pt, innerbottommargin=0pt, innerrightmargin=0pt, topline=false, rightline=false, bottomline=false]
    \par\noindent\textcolor{foreground}{\textbf{#1.}}\hspace{1em}\ignorespaces
}{%
    \par\addvspace{\baselineskip}\end{mdframed}\ignorespacesafterend
}
\newenvironment{solution}{%
    % \par\addvspace{\baselineskip}\noindent\makebox[\textwidth]{\textcolor{foreground}{\textbullet\hspace{1em}\textbullet\hspace{1em}\textbullet}}\par\addvspace{\baselineskip}
    \begin{mdframed}[linewidth=0.5pt, linecolor=foreground, backgroundcolor=background, rightmargin=0cm, innerleftmargin=0cm, innertopmargin=0pt, innerbottommargin=0pt, innerrightmargin=1em, topline=false, leftline=false, bottomline=false]
    \par\noindent\textcolor{foreground}{\textit{Solution.}}\hspace{1em}\ignorespaces
}{%
    \par\addvspace{\baselineskip}\noindent\hfill\textcolor{foreground}{\Coffeecup}\par\addvspace{\baselineskip}\end{mdframed}\ignorespacesafterend
}
% Exercise environment

\declaretheoremstyle[
    name= \textcolor{foreground}{Exercise},
    postheadspace = \newline,
    bodyfont = \normalfont\color{foreground},
    postheadhook={\textcolor{math}{\rule[.4ex]{\linewidth}{0.5pt}}\\},
    % numberwithin=chapter,
    mdframed={
        backgroundcolor = background,
        linecolor = foreground,
        linewidth = 0.5pt,
        rightline =  true,
        topline = true,
        bottomline = true,
        skipabove=20pt,
        skipbelow=20pt,
        innerleftmargin=15pt,
        innertopmargin=10pt,
        innerrightmargin=15pt,
        innerbottommargin=10pt}
    ]{exercise}
\declaretheorem[style=exercise,numbered=no]{exercise}

% \etocsetlevel{exercise}{2}

% \AtEndEnvironment{exercise}{%
%   \etoctoccontentsline{exercise}{\protect\numberline{\theexercise}}%
% }%
% \etocsetstyle{exercise}
% {}
% {}
% % this will be rendered like a non-numbered section, but we could have used
% % \numberline here also
% {\etocsavedsectiontocline{Exercise \etocnumber}{\etocpage}}
%     {}

% theorem environment

\declaretheoremstyle[
    name= \textcolor{foreground}{Theorem},
    postheadspace = \newline,
    bodyfont = \normalfont\color{foreground},
    postheadhook={\textcolor{math}{\rule[.4ex]{\linewidth}{1pt}}\\},
    mdframed={
        backgroundcolor = background,
        linecolor = foreground,
        linewidth = 1pt,
        rightline =  true,
        topline = true,
        bottomline = true,
        skipabove=20pt,
        skipbelow=20pt,
        innerleftmargin=15pt,
        innertopmargin=10pt,
        innerrightmargin=15pt,
        innerbottommargin=10pt}
    ]{theorem}
\declaretheorem[style=theorem,numbered=yes]{theorem}

\declaretheoremstyle[
    name= \textcolor{foreground}{Definition},
    postheadspace = \newline,
    bodyfont = \normalfont\color{foreground},
    postheadhook={\textcolor{math}{\rule[.4ex]{\linewidth}{1pt}}\\},
    mdframed={
        backgroundcolor = background,
        linecolor = foreground,
        linewidth = 1pt,
        rightline =  true,
        topline = true,
        bottomline = true,
        skipabove=20pt,
        skipbelow=20pt,
        innerleftmargin=15pt,
        innertopmargin=10pt,
        innerrightmargin=15pt,
        innerbottommargin=10pt}
    ]{definition}
\declaretheorem[style=definition,numbered=yes]{definition}
% Example environment

\declaretheoremstyle[
name= \quad \underline{Proof:},
     headfont = \bfseries\sffamily,
     postheadspace = \newline,
     % notebraces = \bfseries{(}{)a},
     headpunct = {},
     bodyfont = ,
     postheadhook={\textcolor{foreground}{\rule[0.4ex]{\linewidth}{0pt}}\\},
     qed=\qedsymbol,
    % spacebelow = 10pt,
    mdframed={
  backgroundcolor = background,
  linecolor = foreground,
  linewidth = 1pt,
  skipabove=10pt,
  skipbelow=10pt,
  rightline = false,
  topline = false,
  leftline = false,
  bottomline = false,
  innerleftmargin=15pt,
  innertopmargin=15pt,
  innerrightmargin=15pt,
  innerbottommargin=15pt}
]{pro}
    % \declaretheorem[style=pro,numbered=no]{Proof}

\declaretheoremstyle[
name= \quad \underline{\textcolor{foreground}{Example}},
     headfont = \bfseries\sffamily,
     postheadspace = \newline,
     % notebraces = \bfseries{(}{)a},
     headpunct = {},
     bodyfont = \normalfont\color{foreground},
     postheadhook={\textcolor{foreground}{\rule[0.4ex]{\linewidth}{0pt}}\\},
     % spacebelow = 10pt,
    mdframed={
  backgroundcolor = background,
  linecolor = foreground,
  linewidth = 1pt,
  skipabove=10pt,
  skipbelow=10pt,
  rightline = false,
  topline = false,
  leftline = false,
  bottomline = false,
  innerleftmargin=15pt,
  innertopmargin=15pt,
  innerrightmargin=15pt,
  innerbottommargin=15pt}
]{ex}
\declaretheorem[style=ex,numbered=no]{example}

\declaretheoremstyle[
     name=,
     headfont = \bfseries\sffamily,
     notebraces = \bfseries{},
     headpunct = { -},
     bodyfont = \color{foreground}\normalfont,
     % postheadhook={\textcolor{black}{\rule[.4ex]{\linewidth}{0.2pt}}\\},
    % spacebelow = 10pt,
    mdframed={
  backgroundcolor = background,
  linecolor = foreground,
  linewidth = 1pt,
  skipabove=0pt,
  skipbelow=0pt,
  innerleftmargin=10pt,
  innertopmargin=10pt,
  innerrightmargin=10pt,
  innerbottommargin=10pt,
  rightline = false,
  topline = false,
  leftline = false,
  bottomline = true}
]{subexercise}
% \declaretheorem[style=subexercise,numbered=no]{subexercise}

\declaretheoremstyle[
     name= \color{losning}Løsning,
     headfont = \bfseries\sffamily,
     notebraces = \bfseries{},
     postheadspace = \newline,
     headpunct = {:},
     bodyfont = \normalfont,
     % qed = ,
     % postheadhook={\textcolor{black}{\rule[.4ex]{\linewidth}{0.2pt}}\\},
    % spacebelow = 10pt,
    mdframed={
  backgroundcolor = background,
  linecolor = losning!75,
  linewidth = 1pt,
  skipabove=0pt,
  skipbelow=10pt,
  innerleftmargin=10pt,
  innertopmargin=10pt,
  innerrightmargin=10pt,
  innerbottommargin=10pt,
  leftline = false,
  rightline = true,
  topline = false,
  bottomline = true}
]{solution}

\newenvironment{SimpleBox}[1]{%
  \begin{mdframed}%
    \noindent\textbf{#1}\\[1ex]
}{%
  \end{mdframed}%
}


\begin{document}
\let\cleardoublepage\clearpage
\thispagestyle{fancy}
\chapter{9 - Propagators and the Hidden Causality Term}

When deriving the propagators using the equation of motion approach, we obtain some very straightforward expressions for the Green's function or two-point function. However, that derivation tends to hide the fact that there is a causality requirement in the time-ordering (operator with smallest time acts to the right first and so on). Here we will look at this in more detail.

\begin{exercise}[1]
Show that the vacuum expectation value of the two-point time-ordered product of scalar fields can be written in the form
\begin{align*}
    \langle 0 | T \left[ \phi(x_1) \phi(x_2) \right] | 0 \rangle &= \int \frac{d^3 \mathbf{p}}{(2\pi)^3} \frac{1}{2 \omega_\mathbf{p}} [ \theta(t_1 - t_2) e^{-i \omega_\mathbf{p} (t_1 - t_2)} e^{i \mathbf{p} \cdot (\mathbf{x}_1 - \mathbf{x}_2)} \\
&+ \theta(t_2 - t_1) e^{-i \omega_\mathbf{p} (t_2 - t_1)} e^{i \mathbf{p} \cdot (\mathbf{x}_2 - \mathbf{x}_1)} ],
\end{align*}
where $\omega_\mathbf{p} = \sqrt{\mathbf{p}^2 + m^2}$.
\end{exercise}

\begin{exercise}[2]
Use contour integration to show that one can represent the Heaviside step function in the following form
\begin{align*}
\theta(t) = i \int_{-\infty}^\infty \frac{dz}{2\pi} \frac{e^{-izt}}{z + i\epsilon},
\end{align*}
where $\epsilon$ is an infinitesimal small positive number, i.e., $\epsilon \to 0^+$. Now show that
\begin{align*}
\theta(t) e^{-i \omega_\mathbf{p} t} = i \int_{-\infty}^\infty \frac{dz}{2\pi} \frac{e^{-izt}}{z - (\omega_\mathbf{p} - i\epsilon)}.
\end{align*}
\end{exercise}
\begin{solution}
We change this to a countour integral. where the loop we construct runs over the real line, and then loops back around in the imaginary plane. For $t>0$, the loop goes through the negative side. Notice that $\theta \left( t \right) $ has a pole at $t = -i\epsilon$. We can rewrite integral in the following way,
\begin{align*}
\oint_{\gamma} \frac{dz}{2\pi} \frac{e^{-izt}}{z + i\epsilon} =
\int_{-\infty}^{\infty} \frac{dz}{2\pi} \frac{e^{-izt}}{z + i\epsilon} +
\int_{arc} \frac{dz}{2\pi} \frac{e^{-izt}}{z + i\epsilon} 
\end{align*}
To solve the lefthand side we can use the residual theorem,
\[
\oint_{\gamma} dz f\left( z \right) = \pm 2\pi i \sum_n \text{Res}\left[ f\left( z \right)  \right] _{z = z_n}
.\] 
where $z_n$ are the poles, and the residue is everything in $f$, without the part that gives rise to the pole. We can then compute the integral,
\begin{align*}
    i\oint_{\gamma} \frac{dz}{2\pi} \frac{e^{-izt}}{z + i\epsilon} &= \left( i \right) -2\pi i \left[ \frac{e^{-izt}}{2\pi} \right]_{z = -i\epsilon} \\
    &= 2\pi \left( \frac{e^{\epsilon t}}{2\pi} \right) = e^{\epsilon t}
.\end{align*}
as $\epsilon \to 0$, the value of this integral approaches $1$. Lets now look at the second integral,
 \begin{align*}
     \int_{arc} \frac{dz}{2\pi} \frac{e^{-izt}}{z + i\epsilon} &= \int_{\pi}^{2\pi}d\theta \lim_{R \to \infty} R \left( \frac{e^{-iRt\left( \cos\left( \theta  \right) + i\sin\left( \theta  \right)  \right) }}{R\left( \cos\left( \theta  \right) + i\sin\left( \theta  \right)  \right) } \right) \\
&=   \int_{\pi}^{2\pi}d\theta \lim_{R \to \infty} R \left( \frac{e^{-iRt\cos\theta }e^{Rt\sin \theta }   }{R\left( \cos\left( \theta  \right) + i\sin\left( \theta  \right)  \right) } \right) 
.\end{align*}
This integral goes to zero as $R \to \infty$ as $\sin \theta < 0$ for $\theta \in \left( \pi, 2\pi \right) $. For $t<0$ we do the same, except the loop will now run through the positive side. We then have no poles, so the contour integral is zero, and the integral along the half circle is zero again. \\
And now for the second integral
\begin{align*}
    i \oint \frac{dz}{2\pi} \frac{e^{-izt}}{z - \left( \omega_{\mathbf{p}} - i\epsilon
    \right) } = i 2\pi Res\left[ \frac{e^{-izt}}{2\pi} \right] _{z = \omega_{\mathbf{p}} - i\epsilon} = e^{-i\omega_{\mathbf{p}}t}e^{\epsilon t} = e^{-i\omega_{\mathbf{p}}t}
.\end{align*}
for $\epsilon \to 0$. The arc integral is still zero, so we get the desired result.
\end{solution}
\begin{exercise}[3]
Use the result of 1) and 2) to show that
\begin{align*}
\langle 0 | T \left[ \phi(x_1) \phi(x_2) \right] | 0 \rangle = \int \frac{d^4 p}{(2\pi)^4} \frac{i}{2 \omega_\mathbf{p}} \left[ \frac{e^{-ip \cdot (x_1 - x_2)}}{p_0 - (\omega_\mathbf{p} - i\epsilon)} + \frac{e^{ip \cdot (x_1 - x_2)}}{p_0 - (\omega_\mathbf{p} - i\epsilon)} \right],
\end{align*}
where the four-vector is $p = (p_0, \mathbf{p})$ as usual.
\end{exercise}
We can use the result of 5.1 week 1, to rewrite the integral,
\[
\int \frac{d^{3}\mathbf{p}}{\left( 2\pi \right) ^3}\frac{1}{2\omega_{\mathbf{p}}} \left( 
    i \int_{-\infty}^\infty \frac{dz}{2\pi} \frac{e^{-iz\left( t_1-t_2 \right) }}{z - (\omega_\mathbf{p} - i\epsilon)} e^{i\mathbf{p}\cdot \left( \mathbf{x_1}-\mathbf{x_2} \right)} 
    + 
    i \int_{-\infty}^\infty \frac{dz}{2\pi} \frac{e^{-iz\left( t_2-t_1 \right) }}{z - (\omega_\mathbf{p} - i\epsilon)} e^{i\mathbf{p}\cdot \left( \mathbf{x_2}-\mathbf{x_1} \right)} 
\right)
.\] 
and now we'll do a change of variables $z \to  p_0$, this allows us to collect the integrals,
\begin{align*}
&=  \int \frac{d^{4}\mathbf{p}}{\left( 2\pi \right)^4} \frac{i}{2\omega_{\mathbf{p}}} \left( 
    \frac{e^{-ip_0\left( t_1-t_2 \right)}}{p_0 - (\omega_\mathbf{p} - i\epsilon)} e^{i\mathbf{p}\cdot \left( \mathbf{x_1}-\mathbf{x_2} \right)} 
    + 
    \frac{e^{-ip_0\left( t_2-t_1 \right)}}{p_0 - (\omega_\mathbf{p} - i\epsilon)} e^{i\mathbf{p}\cdot \left( \mathbf{x_2}-\mathbf{x_1} \right)} 
\right) \\
& = \int \frac{d^4 p}{(2\pi)^4} \frac{i}{2 \omega_\mathbf{p}} \left[ \frac{e^{-ip \cdot (x_1 - x_2)}}{p_0 - (\omega_\mathbf{p} - i\epsilon)} + \frac{e^{ip \cdot (x_1 - x_2)}}{p_0 - (\omega_\mathbf{p} - i\epsilon)} \right],
\end{align*}

\begin{exercise}[4]
Substitute the four-vector $p \to -p$ in the second term in the expression in 3) and show that
\begin{align*}
\langle 0 | T \left[ \phi(x_1) \phi(x_2) \right] | 0 \rangle = \int \frac{d^4 p}{(2\pi)^4} e^{-ip \cdot (x_1 - x_2)} \frac{i}{p_0^2 - (\omega_\mathbf{p} - i\epsilon)^2},
\end{align*}
where the four-vector is $p = (p_0, \mathbf{p})$ as usual.
\end{exercise}
Well just calculate the summed fractions,
\begin{align*}
    \frac{1}{p_0 - \left( \omega_\mathbf{p} - i\epsilon \right) } + \frac{1}{-p_0 - \left( \omega_\mathbf{p} - i\epsilon \right) } &= \frac{-2\left( \omega_{\mathbf{p}} - i\epsilon \right) }{\left( p_0 - \omega_{\mathbf{p}} + i\epsilon \right)\left( -p_0 - \omega_{\mathbf{p}} + i\epsilon \right)  } \\
    &= \frac{-2\left( \omega_{\mathbf{p}}-i\epsilon \right) }{-p_0^2 - p_0\omega_{\mathbf{p}} + p_0i\epsilon + p_0\omega_{\mathbf{p}} +\omega_{\mathbf{p}}^2 -\omega_{\mathbf{p}}i\epsilon - p_0i\epsilon - i\epsilon\omega_{\mathbf{p}} -\epsilon^2} \\
&= \frac{-2\left( \omega_p - i\epsilon \right) }{-p_0^2 + \omega_p^2 - \epsilon^2 -2i\epsilon\omega_p} \\
&= \frac{2\left( \omega_p - i\epsilon \right) }{p_0^2 - \left( \omega_p - i\epsilon \right)^2 } \\
.\end{align*}
mulitplying through by $i /2\omega_p$ yields,
\[
\frac{i - \frac{i\epsilon}{\omega_p} }{p_0^2 - \left( \omega_p - i\epsilon \right)^2 } 
.\] 
\begin{exercise}[5]
Finally, argue that this result can be expressed as
\begin{align*}
\langle 0 | T \left[ \phi(x_1) \phi(x_2) \right] | 0 \rangle = \int \frac{d^4 p}{(2\pi)^4} e^{-ip \cdot (x_1 - x_2)} \frac{i}{p^2 - m^2 + i\epsilon},
\end{align*}
since $\epsilon$ is infinitesimal. So the propagator really comes with a positive imaginary part which we usually do not write or care to much about since it drops out of most calculations. But in principle it is always present.
\end{exercise}
\end{document}
