\documentclass[working, oneside]{../../Preambles/tuftebook}
% Import xcolor and define some colors
\usepackage{{xcolor}}
\definecolor{{background}}{{HTML}}{{{background}}}
\definecolor{{foreground}}{{HTML}}{{{foreground}}}
\definecolor{{math}}{{HTML}}{{{color6}}}

%%%%%%%%%%%%%%%%%%%%%%%%%%%%%%%%%%%%%%%% IMPORTS %%%%%%%%%%%%%%%%%%%%%%%%%%%%%%%%%%%%%%%%
\documentclass[11pt,onesize,a4paper,titlepage]{article}

%%%%%%%%%%%%%%% Formatting %%%%%%%%%%%%%%% 
\usepackage[english]{babel}
\usepackage[utf8]{inputenc}
\usepackage{adjustbox}
\usepackage{geometry} % Margins
\usepackage{sectsty} % Custom Sections

%%%%%%%%%%%%%%% Font %%%%%%%%%%%%%%% 
\usepackage{Archivo}
\usepackage[T1]{fontenc}
\sffamily

%%%%%%%%%%%%%%% Graphics %%%%%%%%%%%%%%% 
\usepackage{fontawesome5} % Icons
\usepackage{graphicx} % Images
\usepackage[most]{tcolorbox} % Color Box
\usepackage{xcolor} % Colors
\usepackage{tikz} % For Drawing Shapes
%%%\usepackage{emoji} % For flags
\tcbuselibrary{breakable}
%%%\usepackage{academicons}

%%%%%%%%%%%%%%% Miscelanous %%%%%%%%%%%%%%% 
\usepackage{lipsum} % Lorem Ipsum
\usepackage{hyperref} % For Hyperlinks

%%%%%%%%%%%%%%% Colors %%%%%%%%%%%%%%% 
\definecolor{title}{HTML}{b5bff5} % Color of the title
\definecolor{bars}{HTML}{889af0} % Color of the title
\definecolor{backdrop}{HTML}{f2f2f2} % Color of the side column
\definecolor{lightgray}{HTML}{dfdfdf} % Color for the skill bars

%%% TU green: #639a00
%%% TU gray: #e6e6e6
%\definecolor{title}{HTML}{639a00} % Color of the title TU
%\definecolor{bars}{HTML}{889af0} % Color of the title TU

% \definecolor{backdrop}{HTML}{f2f2f2} % Color of the side column
\definecolor{backdrop}{HTML}{e6e6e6} % Color of the side column

\definecolor{subtitle}{HTML}{606060} % 


%%%%%%%%%%%%%%% Section Format %%%%%%%%%%%%%%% 
\sectionfont{                     
    \LARGE % Font size
    \sectionrule{0pt}{0pt}{-8pt}{1pt} % Rule under Section name
}

\subsectionfont{
    \Large % Font size
    \fontfamily{phv}\selectfont % Font family
    %\sectionrule{0pt}{0pt}{-8pt}{1pt} % Rule under Subsection name
    \sectionrule{5pt}{0pt}{0pt}{0pt} % Rule under Subsection name
}

%%%%%%%%%%%%%%% Margins and Headers %%%%%%%%%%%%%%%
\geometry{
  a4paper,
  left=7mm,
  right=7mm,
  bottom=10mm,
  top=10mm
}

\pagestyle{empty} % Empty Headers

\usepackage{marvosym}

% \renewcommand\qedsymbol{\CoffeeCup}

\usepackage{changepage}

\newenvironment{subexercise}[1]{%
    \begin{mdframed}[linewidth=0.5pt, linecolor=foreground, backgroundcolor=background, leftmargin=0cm, innerleftmargin=1em, innertopmargin=0pt, innerbottommargin=0pt, innerrightmargin=0pt, topline=false, rightline=false, bottomline=false]
    \par\noindent\textcolor{foreground}{\textbf{#1.}}\hspace{1em}\ignorespaces
}{%
    \par\addvspace{\baselineskip}\end{mdframed}\ignorespacesafterend
}
\newenvironment{solution}{%
    % \par\addvspace{\baselineskip}\noindent\makebox[\textwidth]{\textcolor{foreground}{\textbullet\hspace{1em}\textbullet\hspace{1em}\textbullet}}\par\addvspace{\baselineskip}
    \begin{mdframed}[linewidth=0.5pt, linecolor=foreground, backgroundcolor=background, rightmargin=0cm, innerleftmargin=0cm, innertopmargin=0pt, innerbottommargin=0pt, innerrightmargin=1em, topline=false, leftline=false, bottomline=false]
    \par\noindent\textcolor{foreground}{\textit{Solution.}}\hspace{1em}\ignorespaces
}{%
    \par\addvspace{\baselineskip}\noindent\hfill\textcolor{foreground}{\Coffeecup}\par\addvspace{\baselineskip}\end{mdframed}\ignorespacesafterend
}
% Exercise environment

\declaretheoremstyle[
    name= \textcolor{foreground}{Exercise},
    postheadspace = \newline,
    bodyfont = \normalfont\color{foreground},
    postheadhook={\textcolor{math}{\rule[.4ex]{\linewidth}{0.5pt}}\\},
    % numberwithin=chapter,
    mdframed={
        backgroundcolor = background,
        linecolor = foreground,
        linewidth = 0.5pt,
        rightline =  true,
        topline = true,
        bottomline = true,
        skipabove=20pt,
        skipbelow=20pt,
        innerleftmargin=15pt,
        innertopmargin=10pt,
        innerrightmargin=15pt,
        innerbottommargin=10pt}
    ]{exercise}
\declaretheorem[style=exercise,numbered=no]{exercise}

% \etocsetlevel{exercise}{2}

% \AtEndEnvironment{exercise}{%
%   \etoctoccontentsline{exercise}{\protect\numberline{\theexercise}}%
% }%
% \etocsetstyle{exercise}
% {}
% {}
% % this will be rendered like a non-numbered section, but we could have used
% % \numberline here also
% {\etocsavedsectiontocline{Exercise \etocnumber}{\etocpage}}
%     {}

% theorem environment

\declaretheoremstyle[
    name= \textcolor{foreground}{Theorem},
    postheadspace = \newline,
    bodyfont = \normalfont\color{foreground},
    postheadhook={\textcolor{math}{\rule[.4ex]{\linewidth}{1pt}}\\},
    mdframed={
        backgroundcolor = background,
        linecolor = foreground,
        linewidth = 1pt,
        rightline =  true,
        topline = true,
        bottomline = true,
        skipabove=20pt,
        skipbelow=20pt,
        innerleftmargin=15pt,
        innertopmargin=10pt,
        innerrightmargin=15pt,
        innerbottommargin=10pt}
    ]{theorem}
\declaretheorem[style=theorem,numbered=yes]{theorem}

\declaretheoremstyle[
    name= \textcolor{foreground}{Definition},
    postheadspace = \newline,
    bodyfont = \normalfont\color{foreground},
    postheadhook={\textcolor{math}{\rule[.4ex]{\linewidth}{1pt}}\\},
    mdframed={
        backgroundcolor = background,
        linecolor = foreground,
        linewidth = 1pt,
        rightline =  true,
        topline = true,
        bottomline = true,
        skipabove=20pt,
        skipbelow=20pt,
        innerleftmargin=15pt,
        innertopmargin=10pt,
        innerrightmargin=15pt,
        innerbottommargin=10pt}
    ]{definition}
\declaretheorem[style=definition,numbered=yes]{definition}
% Example environment

\declaretheoremstyle[
name= \quad \underline{Proof:},
     headfont = \bfseries\sffamily,
     postheadspace = \newline,
     % notebraces = \bfseries{(}{)a},
     headpunct = {},
     bodyfont = ,
     postheadhook={\textcolor{foreground}{\rule[0.4ex]{\linewidth}{0pt}}\\},
     qed=\qedsymbol,
    % spacebelow = 10pt,
    mdframed={
  backgroundcolor = background,
  linecolor = foreground,
  linewidth = 1pt,
  skipabove=10pt,
  skipbelow=10pt,
  rightline = false,
  topline = false,
  leftline = false,
  bottomline = false,
  innerleftmargin=15pt,
  innertopmargin=15pt,
  innerrightmargin=15pt,
  innerbottommargin=15pt}
]{pro}
    % \declaretheorem[style=pro,numbered=no]{Proof}

\declaretheoremstyle[
name= \quad \underline{\textcolor{foreground}{Example}},
     headfont = \bfseries\sffamily,
     postheadspace = \newline,
     % notebraces = \bfseries{(}{)a},
     headpunct = {},
     bodyfont = \normalfont\color{foreground},
     postheadhook={\textcolor{foreground}{\rule[0.4ex]{\linewidth}{0pt}}\\},
     % spacebelow = 10pt,
    mdframed={
  backgroundcolor = background,
  linecolor = foreground,
  linewidth = 1pt,
  skipabove=10pt,
  skipbelow=10pt,
  rightline = false,
  topline = false,
  leftline = false,
  bottomline = false,
  innerleftmargin=15pt,
  innertopmargin=15pt,
  innerrightmargin=15pt,
  innerbottommargin=15pt}
]{ex}
\declaretheorem[style=ex,numbered=no]{example}

\declaretheoremstyle[
     name=,
     headfont = \bfseries\sffamily,
     notebraces = \bfseries{},
     headpunct = { -},
     bodyfont = \color{foreground}\normalfont,
     % postheadhook={\textcolor{black}{\rule[.4ex]{\linewidth}{0.2pt}}\\},
    % spacebelow = 10pt,
    mdframed={
  backgroundcolor = background,
  linecolor = foreground,
  linewidth = 1pt,
  skipabove=0pt,
  skipbelow=0pt,
  innerleftmargin=10pt,
  innertopmargin=10pt,
  innerrightmargin=10pt,
  innerbottommargin=10pt,
  rightline = false,
  topline = false,
  leftline = false,
  bottomline = true}
]{subexercise}
% \declaretheorem[style=subexercise,numbered=no]{subexercise}

\declaretheoremstyle[
     name= \color{losning}Løsning,
     headfont = \bfseries\sffamily,
     notebraces = \bfseries{},
     postheadspace = \newline,
     headpunct = {:},
     bodyfont = \normalfont,
     % qed = ,
     % postheadhook={\textcolor{black}{\rule[.4ex]{\linewidth}{0.2pt}}\\},
    % spacebelow = 10pt,
    mdframed={
  backgroundcolor = background,
  linecolor = losning!75,
  linewidth = 1pt,
  skipabove=0pt,
  skipbelow=10pt,
  innerleftmargin=10pt,
  innertopmargin=10pt,
  innerrightmargin=10pt,
  innerbottommargin=10pt,
  leftline = false,
  rightline = true,
  topline = false,
  bottomline = true}
]{solution}

\newenvironment{SimpleBox}[1]{%
  \begin{mdframed}%
    \noindent\textbf{#1}\\[1ex]
}{%
  \end{mdframed}%
}


\begin{document}
\let\cleardoublepage\clearpage
\thispagestyle{fancy}
\chapter{4 - The Time Evolution Operator and the S-matrix}
Consider the Schrödinger equation which we write
\begin{align*}
&i \frac{\partial}{\partial t} |\Psi, t\rangle_S = H |\Psi, t\rangle_S, \tag{44}
\end{align*}
where \(H\) is the Hamiltonian and \(|\Psi, t\rangle\) is the state at time \(t\). The subscript \(S\) refers to the Schrödinger picture where operators are time-independent (except for explicit time-dependent terms) and states are time-dependent. Let us furthermore split the Hamiltonian into a free Hamiltonian (containing typically kinetic energy and mass terms) and an interacting part that contains the interactions of different particles, i.e. \(H = H_F + H_{I,S}\). Here the notation \(H_{I,S}\) means the interaction part of the Hamiltonian in the Schrödinger picture. Define the state
\begin{align*}
|\Psi, t\rangle_I = e^{i H_F t} |\Psi, t\rangle_S, \tag{45}
\end{align*}
which is called the interaction picture state and has subscript \(I\), and also define the interaction picture operators
\begin{align*}
O_I = e^{i H_F t} O_S e^{-i H_F t}, \tag{46}
\end{align*}
where \(O_S\) is a Schrödinger picture operator (typically time-independent).
\begin{exercise}[1]
Show that,
\begin{align*}
i \frac{\partial}{\partial t} |\Psi, t\rangle_I = H_I |\Psi, t\rangle_I, \tag{47}
\end{align*}
where $H_I = e^{i H_F t} H_{I,S} e^{-i H_F t}$, and show that 
\[
\frac{d}{dt} O_I = -i [O_I, H_F], \tag{48}
.\] 
where you assume no explicit time-dependence in \(O_S\). From now on \(H_I\) will denote the interaction part of the Hamiltonian in the interaction picture.
\end{exercise}
\begin{solution}
\begin{align*}
i \frac{\partial}{\partial t} |\Psi, t\rangle_I &= i \frac{\partial}{\partial t} e^{i H_F t} |\Psi, t\rangle_S \\
&= i \frac{\partial}{\partial t} (e^{i H_F t}) |\Psi, t\rangle_S + e^{i H_F t} (i \frac{\partial}{\partial t} |\Psi, t\rangle_S) \\
&= -H_F e^{i H_F t} |\Psi, t\rangle_S + e^{i H_F t} H |\Psi, t\rangle_S \\
&= -e^{i H_F t} H_F e^{-i H_F t}|\Psi, t\rangle_I + e^{i H_F t} H e^{-i H_F t} |\Psi, t\rangle_I \\
&= e^{i H_F t} H_{I,S} e^{-i H_F t} |\Psi, t\rangle_I = H_I |\Psi, t\rangle_I
\end{align*}
And now the second part,
\begin{align*}
\frac{d}{dt} O_I &= \frac{d}{dt} (e^{i H_F t} O_S e^{-i H_F t}) \\
&= i H_F (e^{i H_F t} O_S e^{-i H_F t}) + e^{i H_F t} O_S (-i H_F) e^{-i H_F t} \\
&= i H_F O_I - O_I i H_F = i [H_F, O_I].
\end{align*}
\end{solution}
\begin{exercise}[2]
Introduce the time evolution operator, \( U(t, t_0) \), that evolves states from time \( t_0 \) to time \( t \), i.e. \( |\Psi, t\rangle_I = U(t, t_0) |\Psi, t_0\rangle_I \). Clearly \( U(t_0, t_0) = 1 \). Show that
\begin{align*}
i \frac{\partial}{\partial t} U(t, t_0)
&= H_I(t) U(t, t_0). \tag{49}
\end{align*}
Note the explicit time-dependence on \( H_I(t) \)!
\end{exercise}
\begin{solution}
\begin{align*}
|\Psi_t\rangle_I
&= U(t, t_0) |\Psi, t_0\rangle_I
\end{align*}
Show that
\begin{align*}
i \frac{\partial}{\partial t} U(t, t_0)
&= H_I(t) U(t, t_0)
\end{align*}
Let \( |\Psi, t_0\rangle_I \) be the interaction part of some state.
\begin{align*}
i \frac{\partial}{\partial t} (U(t, t_0) |\Psi, t_0\rangle_I)
&= i \frac{\partial}{\partial t} (U(t, t_0) |\Psi, t_0\rangle)_I, \\
&= i \frac{\partial}{\partial t} |\Psi, t\rangle_I, \\
&= H_I |\Psi, t\rangle_I, \\
&= H_I(t) U(t, t_0) |\Psi, t_0\rangle_I
\end{align*}
\end{solution}
\begin{exercise}[3]
Show that
\begin{align*}
U(t, t_0)
&= 1 - i \int_{t_0}^t dt_1 H_I(t_1) U(t_1, t_0), \tag{50}
\end{align*}
and
\begin{align*}
U(t, t_0)
&= 1 + (-i) \int_{t_0}^t dt_1 H_I(t_1) + (-i)^2 \int_{t_0}^t dt_1 \int_{t_0}^{t_1} dt_2 H_I(t_1) H_I(t_2) + \dots, \tag{51}
\end{align*}
where \( \dots \) denote higher-order terms containing three or more factors of \( H_I(t) \).
\end{exercise}
\begin{solution}
\begin{align*}
i \frac{\partial}{\partial t} U(t, t_0)
&= H_I(t) U(t, t_0)
\end{align*}
Integrate on both sides
\begin{align*}
i \int_{t_0}^t \frac{\partial}{\partial t_1} U(t_1, t_0) dt_1
&= \int_{t_0}^t H_I(t_1) U(t_1, t_0) dt_1
\end{align*}
Multiply by \( i \), and evaluate the left-hand side.
\begin{align*}
i (U(t, t_0) - U(t_0, t_0))
&= \int_{t_0}^t H_I(t_1) U(t_1, t_0) dt_1, \\
(-1) (U(t, t_0) - 1)
&= i \int_{t_0}^t H_I(t_1) U(t_1, t_0) dt_1, \\
-U(t, t_0) + 1
&= i \int_{t_0}^t H_I(t_1) U(t_1, t_0) dt_1, \\
U(t, t_0)
&= 1 - i \int_{t_0}^t H_I(t_1) U(t_1, t_0) dt_1
\end{align*}
Alright this seems reasonable. Let's try inserting the RHS in the differential equation and see where it gets us.
\begin{align*}
U(t, t_0)
&= 1 - i \int_{t_0}^t H_I(t_1) \left( 1 - i \int_{t_0}^{t_1} H_I(t_2) U(t_2, t_0) dt_2 \right) dt_1
\end{align*}
By letting the outermost $H\left( t_1 \right) $ integral distribute, we arrive at the expression we were looking for.
\end{solution}
\begin{exercise}[4]
Now define the S-matrix for the process from initial state \( i \) to final state \( f \), i.e. \( i \rightarrow f \), by
\begin{align*}
S_{fi}
&= \lim_{t_0 \rightarrow -\infty} \lim_{t \rightarrow \infty} \langle f | U(t, t_0) | i \rangle. \tag{52}
\end{align*}
Give a physical interpretation of this matrix element given what you know about \( U(t, t_0) \) and relate it to how experiments are done. What have we assumed about the states \( |f\rangle \) and \( |i\rangle \)? If you calculate \( S_{fi} \) in the Schrödinger picture will it be the same result?
\end{exercise}

\begin{exercise}[5]
Show that the first order contribution to \( S_{fi} \) can be written
\begin{align*}
S_{fi}^{(1)}
&= \delta_{fi} - i \int d^4x \langle f | \mathcal{H} | i \rangle, \tag{53}
\end{align*}
where \( H_I = \int d^3x \mathcal{H} \). \( \mathcal{H} \) is called the Hamiltonian density. Since it contains only interaction terms, it differs from the Lagrangian density only by a sign (remember the basic idea that \( L = T - V \) while \( H = T + V \)).

\end{exercise}

\begin{solution}
Show that the first order contribution to \( S_{fi} \) can be written
\begin{align*}
S_{fi}^{(1)}
&= \delta_{fi} - i \int d^4x \langle f | \mathcal{H} | i \rangle
\end{align*}
Where \( H_I = \int d^3x \mathcal{H} \). Alright, let us take a look at \( S_{fi} \)
\begin{align*}
S_{fi}
&= \lim_{t_0 \rightarrow -\infty} \lim_{t \rightarrow \infty} \langle f | U(t, t_0) | i \rangle
\end{align*}
Here \( f \) and \( i \) are initial and final states. Let's insert \( U(t, t_0) \)
\begin{align*}
S_{fi}
&= \lim_{t_0 \rightarrow -\infty} \lim_{t \rightarrow \infty} \left\langle f \left| 1 - i \int_{t_0}^t H_I(t_1) U(t_1, t_0) dt_1 \right| i \right\rangle
\end{align*}
I could do this expansion an indefinite amount of times. I'll do it once more and then just throw away the remaining terms.
\begin{align*}
S_{fi}
&\approx \lim_{t_0 \rightarrow -\infty} \lim_{t \rightarrow \infty} \left( \langle f | i \rangle - i \int_{t_0}^t \left<f \right| H_I(t_1) dt_1 \left|i \right>\right), \\
&= \delta_{if} - i \int_{-\infty}^\infty \left<f \right|H_I(t_1) dt_1\left|i \right>, \\
&= \delta_{if} - i \int \int  dt d^3x \left<f \right|\mathcal{H}\left|i \right>, \\
&= \delta_{if} - i \int d^4x\left<f \right| \mathcal{H}\left|i \right>
\end{align*}
\end{solution}

If we assume that the interactions conserve energy and momentum, we have the commutation relation \( [P^\mu, H] = 0 \), where \( P^\mu \) is the total energy and momentum operator. This operator acts on plane wave as \( P^\mu | k \rangle = k^\mu | k \rangle \). It can also be used to generate finite translations in space and time by application of \( e^{i P^\mu a_\mu} \), where \( a^\mu \) is some space-time vector.
\begin{exercise}[6]
Show that \( [P^\mu, H] = 0 \) implies that \( [e^{i P^\mu a_\mu}, H] = 0 \).
\end{exercise}
\begin{solution}
Show that \( [P^\mu, H] = 0 \) implies \( [e^{i P^\mu a_\mu}, H] = 0 \).

There are a couple of ways to show. One is to do an expansion into a power series. But I'm going to try to use exercise 1 instead.
\begin{align*}
\frac{d}{dt} e^{i P_\mu a^\mu}
&= \frac{d}{dt} (i P_\mu a^\mu) e^{i P_\mu a^\mu}, \\
&= i a^\mu e^{i P_\mu a^\mu} \frac{d}{dt} P_\mu, \\
&= -i [P^\mu, H] \cdot \left( ia_{\mu }e^{iP^{\mu }a_{\mu }} \right)  = 0, \\
\Rightarrow 0
&= -i [e^{i P^\mu a^\mu}, H] = 0
\end{align*}
\end{solution}
\begin{exercise}[7]
The Hamiltonian density depends on space-time coordinates in general, \( \mathcal{H}(x) \). Argue we can use translation operators to write \( \mathcal{H}(x) = e^{i P^\mu x_\mu} \mathcal{H}(0) e^{-i P^\mu x_\mu} \) (Hint: Consider how \( \mathcal{H}(x) \) looks when written in terms of quantum field operators and use the properties of the fields under translations in space and time).
\end{exercise}
\begin{solution}
Let's recall \( \mathcal{H}(x) \) in terms of field operators
\begin{align*}
\mathcal{H}(x, t)
&= \pi(x, t)^\dagger \pi(x, t) + \nabla \phi(x, t) \cdot \nabla \phi(x, t)^\dagger + m^2 \phi(x, t) \phi(x, t)
\end{align*}
From exercise 3.12 we know that
\begin{align*}
\phi(x, t)
&= e^{i P^\mu x_\mu} \phi(0, 0) e^{-i P^\mu x_\mu}
\end{align*}
So we essentially need to show that this holds true for \( \pi, \pi^\dagger, \nabla \phi, \nabla \phi^\dagger \) and \( \phi^\dagger \), and then we will have shown the result.

Let's start with the daggered field operator
\begin{align*}
\phi^\dagger(x, t)
&= \left( e^{i P^\mu x_\mu} \phi(0, 0) e^{-i P^\mu x_\mu} \right)^\dagger
\end{align*}
Recall that the \( \dagger \) reorders the terms and finds complex conjugate as well
\begin{align*}
\Rightarrow
&= e^{i P^\mu x_\mu} \phi(0, 0)^\dagger e^{-i P^\mu x_\mu}
\end{align*}
Since \( \pi \) is so similar to \( \phi \) I will just assume that it holds for it as well.

Let's look at the gradient of the field operator
\begin{align*}
\nabla \phi(x, t)
&= \nabla \left( e^{i P^\mu x_\mu} \phi(0, 0) e^{-i P^\mu x_\mu} \right), \\
&= \left( \nabla e^{i P^\mu x_\mu} \right) \phi(0, 0) e^{-i P^\mu x_\mu} + e^{i P^\mu x_\mu} \phi(0, 0) \nabla \left( e^{-i P^\mu x_\mu} \right)
\end{align*}
The gradient just pulls down the 3-momentum \( \mathbf{p} \) and \( i \)
\begin{align*}
&= (i \mathbf{p}) \left( e^{i P^\mu x_\mu} \phi(0, 0) e^{-i P^\mu x_\mu} \right) + e^{i P^\mu x_\mu} \phi(0, 0) (-i \mathbf{p}) e^{-i P^\mu x_\mu}
\end{align*}

Now we can commute \( \phi(0, 0) \) and \( (-i \mathbf{p}) \) if we pick up a commutator as well. The other terms cancel, and the sign is flipped
\begin{align*}
&= i e^{i P^\mu x_\mu} [\phi(0, 0), \mathbf{p}] e^{-i P^\mu x_\mu}
\end{align*}
Let's evaluate this commutator, now we clearly want to look at equal times,
\begin{align*}
[\phi(0, 0), \mathbf{p}]
&= - \left[ \phi(0, 0), \int d^3x \left\{ \pi(x, 0)^\dagger \nabla \phi(x, 0)^\dagger + \nabla \phi(x, 0) \pi(x, 0) \right\} \right]
\end{align*}
Now a bunch of these are zero the first terms is, and using the piano on the second term, we get a single non-zero term.
\begin{align*}
&= - \int d^3x \left[ \phi(0, 0), \pi(x, 0) \right] \nabla \phi(x, 0), \\
&= - \int d^3x i \delta^3(x - 0) \nabla \phi(x, 0), \\
&= -i \nabla \phi(0, 0)
\end{align*}
We can insert this commutator.
\begin{align*}
\nabla \phi(x, t)
&= i e^{i P^\mu x_\mu} (-i \nabla \phi(0, 0)) e^{-i P^\mu x_\mu}, \\
&= e^{i P^\mu x_\mu} \nabla \phi(0, 0) e^{-i P^\mu x_\mu}
\end{align*}
As a final note before we do the last calculation
\begin{align*}
\phi(x, t) \phi(x, t)^\dagger
&= e^{i P^\mu x_\mu} \phi(0, 0) e^{-i P^\mu x_\mu} e^{i P^\mu x_\mu} \phi(0, 0)^\dagger e^{-i P^\mu x_\mu}, \\
&= e^{i P^\mu x_\mu} \phi(0, 0) \phi(0, 0)^\dagger e^{-i P^\mu x_\mu}
\end{align*}

So now when we consider
\begin{align*}
\mathcal{H}(0)
&= \pi(0, 0)^\dagger \pi(0, 0) + \nabla \phi(0, 0) \nabla \phi(0, 0)^\dagger + m^2 \phi(0, 0)^\dagger \phi(0, 0)
\end{align*}
We see that
\begin{align*}
e^{i P^\mu x_\mu} \mathcal{H}(0) e^{-i P^\mu x_\mu}
&= \mathcal{H}(x, t)
\end{align*}
\end{solution}
\begin{exercise}[8]
Show that this implies that we can isolate the space-time dependence of \( S_{fi}^{(1)} \) through
\begin{align*}
\langle f | \mathcal{H}(x) | i \rangle
&= \langle f | \mathcal{H}(0) | i \rangle e^{i (p_f^\mu - p_i^\mu) x_\mu}, \tag{54}
\end{align*}
where \( p_f \) and \( p_i \) are the total four-momenta of final and initial states respectively.

Show that we now obtain
\begin{align*}
S_{fi}^{(1)}
&= \delta_{fi} - i (2\pi)^4 \langle f | \mathcal{H}(0) | i \rangle \delta^4(p_f - p_i). \tag{55}
\end{align*}
\end{exercise}
\begin{solution}
Let's take a look at
\begin{align*}
\langle f | \mathcal{H}(x) | i \rangle
&= \left\langle f \left| e^{i P^\mu x_\mu} \mathcal{H}(0) e^{-i P^\mu x_\mu} \right| i \right\rangle
\end{align*}
We should consider the operator acts on the state \( i \).
\begin{align*}
e^{-i P^\mu x_\mu} | i \rangle
\end{align*}
Now clearly \( P_\mu | i \rangle = p_i | i \rangle \) it just gives the momentum. And we can consider coordinates as well \( P_k^\mu | i \rangle = p_{i_k}^\mu | i \rangle \), \( k = 1, 2, 3, 4 \).
\noindent
Let's write it out as a power series and a sum
\begin{align*}
e^{-i P^\mu x_\mu} | i \rangle
&= \prod_k^4 e^{-i P_k x_k} | i \rangle = \prod_k^4 \sum_n^\infty \frac{(-i P_k x_k)^n}{n!} | i \rangle, \\
&= \prod_k^4 \sum_n^\infty \frac{(-i x_k)^n (P_k)^n}{n!} | i \rangle, \\
&= \prod_k^4 \sum_n^\infty \frac{(-i x_k p_{i_k})^n}{n!} | i \rangle, \\
&= e^{-i p_i x_\mu}
\end{align*}
We can do a similar trick for the left side,
\begin{align*}
\langle f | e^{i P^\mu x_\mu}
&= \left( e^{-i P^\mu x_\mu} | f \rangle \right)^\dagger = \left<f \right|e^{i p_f x_\mu}
\end{align*}
Therefore,
\begin{align*}
\langle f | \mathcal{H}(x) | i \rangle
&= \langle f | \mathcal{H}(0) | i \rangle e^{i (p_f - p_i) x_\mu}
\end{align*}
We can just plug this in,
\begin{align*}
S_{fi}^{(1)}
&= \delta_{fi} - i \int d^4x \langle f | \mathcal{H}(0) | i \rangle e^{i (p_f - p_i) x_\mu}
\end{align*}
We can move the bracket outside the integral since we have isolated the space time dependence
\begin{align*}
&= \delta_{fi} - i \int d^4x \left( e^{i (p_f - p_i) x_\mu} \right) \langle f | \mathcal{H}(0) | i \rangle, \\
&= \delta_{fi} + (2\pi)^4 \delta^4(p_f - p_i) \langle f | \mathcal{H}(0) | i \rangle
\end{align*}
I get a sign error here, we can hopefully resolve this at some point 
\end{solution}
\begin{exercise}[9]
Consider now the second order contribution to the S-matrix. Show that it can we written in the form
\begin{align*}
S_{fi}^{(2)}
&= (-i)^2 \int d^4 x_1 \int d^4 x_2 \theta(t_1 - t_2) \langle f | \mathcal{H}(x_1) \mathcal{H}(x_2) | i \rangle, \tag{56}
\end{align*}
where \( x_1^\mu = (t_1, \mathbf{x}_1) \) and \( x_2^\mu = (t_2, \mathbf{x}_2) \), while \( \theta(x) \) is the Heaviside step function which is \( \theta(x) = 1 \) for \( x > 0 \) and \( \theta(x) = 0 \) for \( x < 0 \). Finally, show that we can write this in the form
\begin{align*}
S_{fi}^{(2)}
&= (-i)^2 (2\pi)^4 \delta^4(p_f - p_i) \int d^4 x \theta(t) \langle f | \mathcal{H}(x) \mathcal{H}(0) | i \rangle, \tag{57}
\end{align*}
where \( x = x_1 - x_2 \).
\end{exercise}
\begin{solution}
The first rewrite is rather simple. Let's recall the second order term.
\begin{align*}
S_{fi}^{(2)}
&= \lim_{t_0 \rightarrow -\infty} \lim_{t \rightarrow +\infty} (-i)^2 \int_{t_0}^t dt_1 \int_{t_0}^{t_1} dt_2 \int d^3x_1 \int d^3x_2 \langle f | \mathcal{H}(x_1) \mathcal{H}(x_2) | i \rangle
\end{align*}
Where I have inserted the hamiltonian densities. We can evaluate the limits.
\begin{align*}
S_{fi}^{(2)}
&= (-i)^2 \int_{-\infty}^\infty dt_1 \int_{-\infty}^{t_1} dt_2 \int d^3x_1 \int d^3x_2 \langle f | \mathcal{H}(x_1) \mathcal{H}(x_2) | i \rangle
\end{align*}
Where we can rewrite the innermost integral using a heaviside function
\begin{align*}
\int_{-\infty}^{t_1} dt_2
&= \int_{-\infty}^\infty \theta(t_1 - t_2) dt_2, \quad \theta(t_1 - t_2) = \begin{cases} 0 & t_2 > t_1 \\ 1 & t_1 > t_2 \end{cases}
\end{align*}
Inserting this and collecting the integrals we get
\begin{align*}
S_{fi}^{(2)}
&= (-i)^2 \int d^4x_1 \int d^4x_2 \theta(t_1 - t_2) \langle f | \mathcal{H}(x_1) \mathcal{H}(x_2) | i \rangle
\end{align*}
At this point we should make a substitution
\begin{align*}
X&= x_2, \quad x = x_1 - x_2, \\
dX &= dx_2 \quad x_1 = x + x_2 = x + X, \\
dx_1 &= dx
\end{align*}
Also note that in this substitution \( t_1 - t_2 \rightarrow t \).
\begin{align*}
S_{fi}^{(2)}
&= (-i)^2 \int d^4x \int d^4X \theta(t) \langle f | \mathcal{H}(x + X) \mathcal{H}(X) | i \rangle
\end{align*}
At this point we can employ the relations we showed in 4.8. We shall expand the hamiltonian densities.
\begin{align*}
S_{fi}^{(2)}
&= (-i)^2 \int d^4x \int d^4X \theta(t) \left\langle f \left| e^{i P^\mu X_\mu} \mathcal{H}(x) e^{-i P^\mu X_\mu} e^{i P^\mu X_\mu} \mathcal{H}(0) e^{-i P^\mu X_\mu} \right| i \right\rangle
\end{align*}
The middle terms cancel and we can pull out the exponents by letting them act on the states.
\begin{align*}
&= (-i)^2 \int d^4x \int d^4X \theta(t) \langle f | \mathcal{H}(x) \mathcal{H}(0) | i \rangle e^{i (p_f - p_i) X}
\end{align*}
We can pull out the bracket,
\begin{align*}
&= (-i)^2 \int d^4x \langle f | \mathcal{H}(x) \mathcal{H}(0) | i \rangle \int d^4\overline{X} \theta(t) e^{i (p_f - p_i) X}
\end{align*}
We can also pull out the heaviside step-function and evaluate the integral
\begin{align*}
&= (-i)^2 (2\pi)^4 \delta^4(p_f - p_i) \int d^4x \langle f | \mathcal{H}(x) \mathcal{H}(0) | i \rangle
\end{align*}
\end{solution}
\end{document}
