\documentclass[working, oneside]{../../Preambles/tuftebook}
% Import xcolor and define some colors
\usepackage{xcolor}
\definecolor{background}{HTML}{ffffff}
\definecolor{foreground}{HTML}{000000}
\definecolor{math}{HTML}{000000}

%%%%%%%%%%%%%%%%%%%%
%% SUPER PREAMBLE %%
%%%%%%%%%%%%%%%%%%%%

% \usepackage[utf8]{inputenc}
\usepackage[T1]{fontenc} % Fonts and stuff
\usepackage{amsmath, amsfonts, mathtools, amsthm, amssymb} % math

\usepackage{fancyhdr} % Header, Footer etc.
\usepackage{adforn}
\usepackage{efbox}
\usepackage{lastpage}
\usepackage{marvosym}
\usepackage{pict2e}
\usepackage{caption}
\usepackage{wrapfig}
\usepackage{graphicx}
\usepackage{sidecap}
% \usepackage{mathpazo} 
% \usepackage{cmbright}
\usepackage{mathptmx}

\usepackage[
    sorting=nyt,
    style=alphabetic
]{biblatex}
\addbibresource{references.bib}
\usepackage[noabbrev]{cleveref}

\pagestyle{fancy}
\fancyhead[R]{}
\fancyhead[L]{}
\fancyfoot[C]{\efbox[margin = 10pt,
                    topline = false,
                    leftline = false,
                    rightline = false,
                    backgroundcolor = background,
                    linewidth = 1pt,
                    linecolor = foreground]{\thepage\ of \pageref{LastPage}}}
% \fancyfoot[C]{\color{foreground} \thepage}

% \renewcommand{\headrule}{%
% 	\hrulefill
% }
% \renewcommand{\footrulewidth}{0pt}
\renewcommand{\headrulewidth}{0pt}

% \setlength{\headheight}{15pt}
% \setlength{\footheight}{15pt}

%% Margin Control %%

% \def\changemargin#1#2{\list{}{\rightmargin#2\leftmargin#1}\item[]}
% \let\endchangemargin=\endlist


%%%%%%%%%%%%%%%%%%%%%%%%%%%%%%%%%%%%%%%%%%%%%%%%%%%%%%%%%%

% figure support

\usepackage{import}
\usepackage{transparent}


\newcommand{\incfig}[2][1]{%
    \def\svgwidth{#1\columnwidth}
    \import{figures/}{#2.pdf_tex}
}

% \pdfsuppresswarningpagegroup=1

%%%%%%%%%%%%%%%%%%%%%%%%%%%%%%%%%%%%%%%%%%%%%%%%%%%%%%%%%%

\usepackage{tikzsymbols} % Symbols
\usepackage[framemethod=TikZ]{mdframed} % Boxes around theorem environments
\usepackage{thmtools}




% \everymath{\color{math}}
% \everydisplay{\color{math}}
% \def\m@th{\normalcolor\mathsurround\z@}

\color{foreground}

	% \end{changemargin}
	% } 

 % \newenvironment{subexercise}[1]
 % {\noindent
	 % \textbf{(#1)} \quad \adforn{10} \quad \em
 % }{}

% Mathematical typesetting stuff.

 % \newcommand{\dd}{\mathrm{\textbf{d}}}

 % Change font

% \usepackage{tgadventor}
% \usepackage{cmbright}
% \usepackage{bm}

% \usepackage{microtype} % Microtypography
% \usepackage{fontspec}% Hyperlinks
% \usepackage{fouriernc}

% \def\MT@set@inh@list#1#2{%
%   \MT@ifempty\MT@inh@feat{%
%     \MT@map@clist@c\MT@features{\begingroup % <--
%       \MT@ifstreq{##1}{tr}\relax{\MT@declare@char@inh{##1}{#1}{#2}}%
%     \endgroup}% <--
%   }{%
%     \MT@map@clist@c\MT@inh@feat{\begingroup % <--
%       \KV@@sp@def\@tempa{##1}%
%       \MT@ifempty\@tempa\relax{%
%         \edef\@tempa{\csname MT@rbba@\@tempa\endcsname}%
%         \MT@ifstreq\@tempa{tr}\relax{%
%           \MT@exp@one@n\MT@declare@char@inh{\@tempa}{#1}{#2}}}%
%     \endgroup}% <--
%   }%      
% \DeclareCaptionFormat{custom}{\bfseries#1#2\itshape#3}%   \MT@end@catcodes
\DeclareCaptionFormat{custom}{\bfseries\itshape#1#2\normalfont\small#3}
\captionsetup{
    format=custom,
    labelsep=space,
    width=\textwidth, % Set the caption width to be 80% of the text width
    % justification=jusitified, % Center-align the caption
    % font=it % Italicize the caption text
}
% }
\usepackage{setspace}
\DeclareCaptionFormat{margin}{\small\bfseries#1#2#3}

\usepackage{xparse} % For advanced command definitions with optional arguments

\NewDocumentCommand{\marginfig}{O{0cm} m m m}{%
  % #1 = optional padding (default 0cm), #2 = filename, #3 = label, #4 = caption
  \marginpar{%
    \includegraphics[width=\marginparwidth]{#2}%
    \captionsetup{format=custom, labelsep=space, width=\marginparwidth, justification=raggedright, font=small}
    \captionof{figure}{#4}%
    \label{fig:#3}%
    % \rule{\marginparwidth}{0.4pt} % Adds a line below the caption
    \vspace{#1} % Adds the specified padding below the caption
  }%
}

\NewDocumentCommand{\maintextfig}{O{0cm} m m m}{
  % #1 = optional vertical adjustment for the caption (default 0cm)
  % #2 = filename for the figure
  % #3 = label for the figure
  % #4 = caption text

  % Place the figure in the text
  \begin{figure}[htbp]
    \centering
    \includegraphics[width=\textwidth]{#2}
    \marginnote{\captionsetup{format=custom, labelsep=space, width=\marginparwidth, justification=fill, font={stretch=1}}
        \captionof{figure}[#4]{#4}\label{fig:#3}}[#1]
    \label{fig:#3}
  \end{figure}

  % Place the caption in the margin
  % \marginnote{\captionsetup{format=custom, labelsep=space, width=\marginparwidth, justification=raggedright, font={stretch=1}}
  %   \captionof{figure}[#4]{#4}\label{fig:#3}}[#1]
}
% \NewDocumentCommand{\marginfig}{m m m}{
%   % #1 = filename, #2 = label, #3 = caption
%   \begin{wrapfigure}{r}{5cm} % "r" for right side, and "5cm" for the width of the figure
%       \centering
%       \includegraphics[width=5cm]{#1}
%      \captionsetup{format=custom, labelsep=space, width=6cm, justification=raggedright, font={stretch=1}}
%       \captionof{figure}{#3}%
%       \label{fig:#2}
%   \end{wrapfigure}
% }{}
% \newcommand{\marginfig}[3]{%
%   \marginpar{%
%     \includegraphics[width=\marginparwidth]{#1}%
%     \captionsetup{format=custom, labelsep=space, width=\marginparwidth, justification=raggedright, font={stretch=1}}
%     \captionof{figure}{\fontsize{11pt}{11pt}\selectfont #3}%
%     \label{fig:#2}%
%     % \rule{\marginparwidth}{0.4pt}
%   }%
% }
% \newcommand{\marginfig}[4][0pt]{%
%   \marginpar{%
%     \raisebox{#1}{%
%       \includegraphics[width=\marginparwidth]{#2}%
%       \captionsetup{format=margin, labelsep=space, justification=raggedright}
%       \captionof{figure}{#4}%
%       \label{fig:#3}%
%     }%
%   }%
% }
% \newcommand{\marginfig}[2][0pt]{%
%   \marginpar{\raisebox{#1}{%
%       \includegraphics[width=1.0\marginparwidth]{#2}
%       % \label{fig:#3}
%       % \caption{#4}
%       % \parbox{\marginparwidth}{\smaller \textbf{figure:} #3}%
%   }}%
% }
\newcommand{\margintext}[2][0pt]{%
  \marginpar{\raisebox{#1}{%
    \parbox{\marginparwidth}{\smaller \textbf{figure:} #2}%
    }}%
}

\newcommand{\marginmath}[2][0pt]{%
  \marginpar{\raisebox{#1}{%
    \parbox{1.2\marginparwidth}{#2}%
    }}%
}
\pagecolor{background}
\usepackage{listings}
\definecolor{commentsColor}{rgb}{0.497495, 0.497587, 0.497464}
\definecolor{keywordsColor}{rgb}{0.000000, 0.000000, 0.635294}
\definecolor{stringColor}{rgb}{0.558215, 0.000000, 0.135316}
\renewcommand*\ttdefault{txtt}
\lstset{
  basicstyle=\ttfamily\small,                   % the size of the fonts that are used for the code
  breakatwhitespace=false,                      % sets if automatic breaks should only happen at whitespace
  breaklines=true,                              % sets automatic line breaking
  frame=tb,                                     % adds a frame around the code
  commentstyle=\color{commentsColor}\textit,    % comment style
  keywordstyle=\color{keywordsColor}\bfseries,  % keyword style
  stringstyle=\color{stringColor},              % string literal style
  numbers=left,                                 % where to put the line-numbers; possible values are (none, left, right)
  numbersep=5pt,                                % how far the line-numbers are from the code
  numberstyle=\tiny\color{commentsColor},       % the style that is used for the line-numbers
  showstringspaces=false,                       % underline spaces within strings only
  tabsize=2,                                    % sets default tabsize to 2 spaces
  language=Scala
}


\usepackage{marvosym}

% \renewcommand\qedsymbol{\CoffeeCup}

\usepackage{changepage}

\newenvironment{subexercise}[1]{%
    \begin{mdframed}[linewidth=0.5pt, linecolor=foreground, backgroundcolor=background, leftmargin=0cm, innerleftmargin=1em, innertopmargin=0pt, innerbottommargin=0pt, innerrightmargin=0pt, topline=false, rightline=false, bottomline=false]
    \par\noindent\textcolor{foreground}{\textbf{#1.}}\hspace{1em}\ignorespaces
}{%
    \par\addvspace{\baselineskip}\end{mdframed}\ignorespacesafterend
}
\newenvironment{solution}{%
    % \par\addvspace{\baselineskip}\noindent\makebox[\textwidth]{\textcolor{foreground}{\textbullet\hspace{1em}\textbullet\hspace{1em}\textbullet}}\par\addvspace{\baselineskip}
    \begin{mdframed}[linewidth=0.5pt, linecolor=foreground, backgroundcolor=background, rightmargin=0cm, innerleftmargin=0cm, innertopmargin=0pt, innerbottommargin=0pt, innerrightmargin=1em, topline=false, leftline=false, bottomline=false]
    \par\noindent\textcolor{foreground}{\textit{Solution.}}\hspace{1em}\ignorespaces
}{%
    \par\addvspace{\baselineskip}\noindent\hfill\textcolor{foreground}{\Coffeecup}\par\addvspace{\baselineskip}\end{mdframed}\ignorespacesafterend
}
% Exercise environment

\declaretheoremstyle[
    name= \textcolor{foreground}{Exercise},
    postheadspace = \newline,
    bodyfont = \normalfont\color{foreground},
    postheadhook={\textcolor{math}{\rule[.4ex]{\linewidth}{0.5pt}}\\},
    % numberwithin=chapter,
    mdframed={
        backgroundcolor = background,
        linecolor = foreground,
        linewidth = 0.5pt,
        rightline =  true,
        topline = true,
        bottomline = true,
        skipabove=20pt,
        skipbelow=20pt,
        innerleftmargin=15pt,
        innertopmargin=10pt,
        innerrightmargin=15pt,
        innerbottommargin=10pt}
    ]{exercise}
\declaretheorem[style=exercise,numbered=no]{exercise}

% \etocsetlevel{exercise}{2}

% \AtEndEnvironment{exercise}{%
%   \etoctoccontentsline{exercise}{\protect\numberline{\theexercise}}%
% }%
% \etocsetstyle{exercise}
% {}
% {}
% % this will be rendered like a non-numbered section, but we could have used
% % \numberline here also
% {\etocsavedsectiontocline{Exercise \etocnumber}{\etocpage}}
%     {}

% theorem environment

\declaretheoremstyle[
    name= \textcolor{foreground}{Theorem},
    postheadspace = \newline,
    bodyfont = \normalfont\color{foreground},
    postheadhook={\textcolor{math}{\rule[.4ex]{\linewidth}{1pt}}\\},
    mdframed={
        backgroundcolor = background,
        linecolor = foreground,
        linewidth = 1pt,
        rightline =  true,
        topline = true,
        bottomline = true,
        skipabove=20pt,
        skipbelow=20pt,
        innerleftmargin=15pt,
        innertopmargin=10pt,
        innerrightmargin=15pt,
        innerbottommargin=10pt}
    ]{theorem}
\declaretheorem[style=theorem,numbered=yes]{theorem}

\declaretheoremstyle[
    name= \textcolor{foreground}{Definition},
    postheadspace = \newline,
    bodyfont = \normalfont\color{foreground},
    postheadhook={\textcolor{math}{\rule[.4ex]{\linewidth}{1pt}}\\},
    mdframed={
        backgroundcolor = background,
        linecolor = foreground,
        linewidth = 1pt,
        rightline =  true,
        topline = true,
        bottomline = true,
        skipabove=20pt,
        skipbelow=20pt,
        innerleftmargin=15pt,
        innertopmargin=10pt,
        innerrightmargin=15pt,
        innerbottommargin=10pt}
    ]{definition}
\declaretheorem[style=definition,numbered=yes]{definition}
% Example environment

\declaretheoremstyle[
name= \quad \underline{Proof:},
     headfont = \bfseries\sffamily,
     postheadspace = \newline,
     % notebraces = \bfseries{(}{)a},
     headpunct = {},
     bodyfont = ,
     postheadhook={\textcolor{foreground}{\rule[0.4ex]{\linewidth}{0pt}}\\},
     qed=\qedsymbol,
    % spacebelow = 10pt,
    mdframed={
  backgroundcolor = background,
  linecolor = foreground,
  linewidth = 1pt,
  skipabove=10pt,
  skipbelow=10pt,
  rightline = false,
  topline = false,
  leftline = false,
  bottomline = false,
  innerleftmargin=15pt,
  innertopmargin=15pt,
  innerrightmargin=15pt,
  innerbottommargin=15pt}
]{pro}
    % \declaretheorem[style=pro,numbered=no]{Proof}

\declaretheoremstyle[
name= \quad \underline{\textcolor{foreground}{Example}},
     headfont = \bfseries\sffamily,
     postheadspace = \newline,
     % notebraces = \bfseries{(}{)a},
     headpunct = {},
     bodyfont = \normalfont\color{foreground},
     postheadhook={\textcolor{foreground}{\rule[0.4ex]{\linewidth}{0pt}}\\},
     % spacebelow = 10pt,
    mdframed={
  backgroundcolor = background,
  linecolor = foreground,
  linewidth = 1pt,
  skipabove=10pt,
  skipbelow=10pt,
  rightline = false,
  topline = false,
  leftline = false,
  bottomline = false,
  innerleftmargin=15pt,
  innertopmargin=15pt,
  innerrightmargin=15pt,
  innerbottommargin=15pt}
]{ex}
\declaretheorem[style=ex,numbered=no]{example}

\declaretheoremstyle[
     name=,
     headfont = \bfseries\sffamily,
     notebraces = \bfseries{},
     headpunct = { -},
     bodyfont = \color{foreground}\normalfont,
     % postheadhook={\textcolor{black}{\rule[.4ex]{\linewidth}{0.2pt}}\\},
    % spacebelow = 10pt,
    mdframed={
  backgroundcolor = background,
  linecolor = foreground,
  linewidth = 1pt,
  skipabove=0pt,
  skipbelow=0pt,
  innerleftmargin=10pt,
  innertopmargin=10pt,
  innerrightmargin=10pt,
  innerbottommargin=10pt,
  rightline = false,
  topline = false,
  leftline = false,
  bottomline = true}
]{subexercise}
% \declaretheorem[style=subexercise,numbered=no]{subexercise}

\declaretheoremstyle[
     name= \color{losning}Løsning,
     headfont = \bfseries\sffamily,
     notebraces = \bfseries{},
     postheadspace = \newline,
     headpunct = {:},
     bodyfont = \normalfont,
     % qed = ,
     % postheadhook={\textcolor{black}{\rule[.4ex]{\linewidth}{0.2pt}}\\},
    % spacebelow = 10pt,
    mdframed={
  backgroundcolor = background,
  linecolor = losning!75,
  linewidth = 1pt,
  skipabove=0pt,
  skipbelow=10pt,
  innerleftmargin=10pt,
  innertopmargin=10pt,
  innerrightmargin=10pt,
  innerbottommargin=10pt,
  leftline = false,
  rightline = true,
  topline = false,
  bottomline = true}
]{solution}

\newenvironment{SimpleBox}[1]{%
  \begin{mdframed}%
    \noindent\textbf{#1}\\[1ex]
}{%
  \end{mdframed}%
}


\begin{document}
\let\cleardoublepage\clearpage
\thispagestyle{fancy}
\chapter{Opgave 2}

% \documentclass{article}


\begin{exercise}[2.6]
Show that we can ensure the commutation relation $[\phi(x), \pi(x')] = i\delta^3(x - x')$ if we take the operator commutation relations to be
\begin{align*}
[a_{\mathbf{p}}, a_{\mathbf{p'}}^{\dagger}] &= (2\pi)^3 \delta^3(\mathbf{p} - \mathbf{p'}) \\
[c_{\mathbf{p}}, c_{\mathbf{p'}}^{\dagger}] &= (2\pi)^3 \delta^3(\mathbf{p} - \mathbf{p'})
\end{align*}
\end{exercise}

\begin{solution}
Recall,
\begin{align*}
\phi(x) &= \int \frac{d^3p}{(2\pi)^3} \frac{1}{\sqrt{2\omega_{\mathbf{p}}}} (a_{\mathbf{p}} e^{i\mathbf{p}\cdot\mathbf{x}} + c_{\mathbf{p}}^{\dagger} e^{-i\mathbf{p}\cdot\mathbf{x}}) \\
\pi(x') &= \int \frac{d^3p'}{(2\pi)^3} (-i) \sqrt{\frac{\omega_{\mathbf{p'}}}{2}} (a_{\mathbf{p'}} e^{i\mathbf{p'}\cdot\mathbf{x'}} - c_{\mathbf{p'}}^{\dagger} e^{-i\mathbf{p'}\cdot\mathbf{x'}})
\end{align*}

And the following rules for commutators
\begin{align*}
[\alpha A, B] &= \alpha [A, B] \\
[A, B \pm C] &= [A, B] \pm [A, C]
\end{align*}

Let's take a look at the commutator
\begin{align*}
[\phi(x), \pi(x')] &= \left[ \int \frac{d^3p}{(2\pi)^3} \frac{1}{\sqrt{2\omega_p}} (a_p e^{ip\cdot x} + c_p^\dagger e^{-ip\cdot x}), \int \frac{d^3p'}{(2\pi)^3} i \sqrt{\frac{\omega_{p'}}{2}} (a_{p'}^\dagger e^{-ip'\cdot x'} - c_{p'} e^{ip'\cdot x'}) \right] \\
\end{align*}

For simplicity let's introduce $k = ip\cdot x$, $k' = ip'\cdot x'$

\begin{align*}
&= \frac{1}{2} \left( \frac{i}{(2\pi)^6} \right) \int d^3p \int d^3p' \left[ a_p e^k + c_p^\dagger e^{-k}, a_{p'}^\dagger e^{-k'} - c_{p'} e^{k'} \right] \\
&= A \left( \left[ a_p e^k, a_{p'}^\dagger e^{-k'} - c_{p'} e^{k'} \right] + \left[ c_p^\dagger e^{-k}, a_{p'}^\dagger e^{-k'} - c_{p'} e^{k'} \right] \right) \\
&= A \left( \left[ a_p e^k, a_{p'}^\dagger e^{-k'} \right] - \left[ a_p e^k, c_{p'} e^{k'} \right] + \left[ c_p^\dagger e^{-k}, a_{p'}^\dagger e^{-k'} \right] - \left[ c_p^\dagger e^{-k}, c_{p'} e^{k'} \right] \right)
\end{align*}

At this point we see that the terms with different operators definitely cancel.

\begin{align*}
&= A \left( e^{k-k'} [a_p, a_{p'}^\dagger] - e^{k'-k} [c_p^\dagger, c_{p'}] \right)
\end{align*}

Before moving on let's quickly look at the commutator,
\begin{align*}
[c_p, c_{p'}^\dagger]^\dagger &= \left( (2\pi)^3 \delta^3(p-p') \right)^\dagger = (2\pi)^3 \delta^3(p-p') \\
&= [c_p, c_{p'}^\dagger]
\end{align*}

And,
\begin{align*}
[c_p, c_{p'}^\dagger]^\dagger &= (c_p c_{p'}^\dagger)^\dagger - (c_{p'}^\dagger c_p)^\dagger \\
&= c_{p'}^{\dagger \dagger} c_p^\dagger - c_p^\dagger c_{p'}^{\dagger \dagger} = c_{p'} c_p^\dagger - c_p^\dagger c_{p'} = [c_{p'}, c_p^\dagger] \\
&= -[c_p^\dagger, c_{p'}] = [c_p, c_{p'}^\dagger]
\end{align*}

We can then insert this.
\begin{align*}
&= A \left( e^{k-k'} [a_p, a_{p'}^\dagger] + e^{k'-k} [c_p, c_{p'}^\dagger] \right) \\
&= A \left( e^{k-k'} (2\pi)^3 \delta^3(p-p') + e^{k'-k} (2\pi)^3 \delta^3(p-p') \right)
\end{align*}

We pull out the $(2\pi)^3$ and begin evaluating the integrals.
\begin{align*}
&= \frac{1}{2} \left( \frac{i}{(2\pi)^3} \right) \int d^3p \int d^3p' \left( e^{i(p\cdot x - p'\cdot x')} \delta^3(p-p') + e^{i(p'\cdot x' - p\cdot x)} \delta^3(p-p') \right)
\end{align*}
The innermost integral just selects the value where the delta function is 0. Therefore, we get $p' = p$.

\begin{align*}
&= \frac{1}{2} \left( \frac{i}{(2\pi)^3} \right) \int d^3p \left( e^{ip\cdot(x-x')} + e^{ip\cdot(x'-x)} \right)
\end{align*}

This is just the definition of the delta function

\begin{align*}
&= \frac{1}{2} \left( \frac{i}{(2\pi)^3} \right) \left( (2\pi)^3 \delta^3(x-x') + (2\pi)^3 \delta^3(x'-x) \right)
\end{align*}

Recall
\[
\delta(-x) = \delta(x) \implies \delta(x-x') = \delta(-(x-x')) = \delta(x'-x)
.\] 
So we get,
\begin{align*}
&= \frac{1}{2} i \left( \frac{1}{(2\pi)^3} \right) 2 \cdot (2\pi)^3 \delta^3(x-x') = i \delta^3(x-x')
\end{align*}
\end{solution}

\begin{exercise}[8]
For the field operators, the conserved charge becomes:

\[
Q = i \int d^3x \left( \phi^\dagger(x) \pi^\dagger(x) - \pi(x) \phi(x) \right)
\]

Argue that this expression makes sense when compared to the flow current of a classical scalar field.
\end{exercise}

\begin{solution} In the classical picture we have:

\[
\rho = i \left( \phi^* \frac{\partial \phi}{\partial t} - \frac{\partial \phi^*}{\partial t} \phi \right)
\]

With Lagrangian:

\[
\mathcal{L}_{KG} = (\partial^\mu \phi^*) \partial_\mu \phi - m^2 \phi^* \phi
\]

Now, we also know that:

\[
\pi = \frac{\partial \mathcal{L}_{KG}}{\partial (\partial_t \phi)} = (\partial^t \phi)^* = \frac{\partial \phi^*}{\partial t}
\]

And likewise:

\[
\pi^* = \frac{\partial \phi}{\partial t}
\]

Translating this we get a nice correspondence between:

\[
i \left( \phi^* \frac{\partial \phi}{\partial t} - \frac{\partial \phi^*}{\partial t} \phi \right) \rightarrow i \left( \phi^\dagger \pi^\dagger - \pi \phi \right)
\]

So it does make sense

\end{solution}
\begin{exercise}[2.9]
Show that,
\begin{align*}
    [H, a_p^\dagger] &= \omega_p a_p^\dagger \quad \text{and} \quad [H, a_p] = -\omega_p a_p
\end{align*}
And assuming $a_p |0\rangle = 0$, show that the full spectrum of energy can be obtained from repeated application of the creation operator.
\end{exercise}
\begin{solution}
Let's write it out,

\begin{align*}
    [H, a_p^\dagger] &= \left[ \int \frac{d^3p'}{(2\pi)^3} \omega_{p'} (a_{p'}^\dagger a_{p'} + \frac{1}{2} [a_{p'}, a_{p'}^\dagger] + c_{p'}^\dagger c_{p'} + \frac{1}{2} [c_{p'}, c_{p'}^\dagger], a_p^\dagger \right]
\end{align*}

We can move out the integrals, and also see that the terms with \(c_{p'}^\dagger c_{p'}\) will generate mixed commutators, which we know are zero.

\begin{align*}
    &= \omega_p \int \frac{d^3p'}{(2\pi)^3} [a_{p'}^\dagger a_{p'} + \frac{1}{2} [a_{p'}, a_{p'}^\dagger], a_p^\dagger] \\
    &= \omega_p \int \frac{d^3p'}{(2\pi)^3} \left( [a_{p'}^\dagger a_{p'}, a_p^\dagger] + \frac{1}{2} [[a_{p'}, a_{p'}^\dagger], a_p^\dagger] \right)
\end{align*}

Now \( [a_{p'}, a_{p'}^\dagger] = \alpha \delta^{(3)}(0) \), i.e. is a fixed quantity, so it commutes with everything.

\begin{align*}
    &= \omega_p \int \frac{d^3p'}{(2\pi)^3} [a_{p'}^\dagger a_{p'}, a_p^\dagger]
\end{align*}

Let's stop for a while.

\begin{align*}
    [a_{p'}^\dagger a_{p'}, a_p^\dagger] &= -[a_p^\dagger, a_{p'}^\dagger a_{p'}] \\
    &= -[a_p^\dagger, a_{p'}^\dagger] a_{p'} - a_{p'}^\dagger [a_p^\dagger, a_{p'}] \\
    &= a_{p'}^\dagger [a_{p'}, a_p^\dagger] = a_{p'}^\dagger (2\pi)^3 \delta^{(3)}(p' - p)
\end{align*}

Insert,

\begin{align*}
    &= \omega_p \int \frac{d^3p'}{(2\pi)^3} a_{p'}^\dagger (2\pi)^3 \delta^{(3)}(p' - p) = \omega_p a_p^\dagger
\end{align*}
We now show, the corresponding commutator for the annihilation operator
\begin{align*}
    [H, a_p] &= \int \frac{d^3p}{(2\pi)^3} \omega_p [a_p^\dagger a_p + \frac{1}{2} [a_p, a_p^\dagger], a_p'] \\
    &= \int \frac{d^3p}{(2\pi)^3} \omega_p ([a_p^\dagger a_p, a_p'] + \frac{1}{2} [[a_p, a_p^\dagger], a_p'])
\end{align*}

Now \( [a_p, a_p^\dagger] = (2\pi)^3 \delta^{(3)}(0) \). So it commutes with \( a_p' \).

\begin{align*}
    &= -\int \frac{d^3p}{(2\pi)^3} \omega_p [a_p', a_p^\dagger a_p] \\
    &= -\int \frac{d^3p}{(2\pi)^3} \omega_p ([a_p', a_p^\dagger] a_p + a_p^\dagger [a_p', a_p]) \\
    &= -\int \frac{d^3p}{(2\pi)^3} \omega_p (2\pi)^3 \delta^{(3)}(p' - p) a_p' \\
    &= -\omega_p' a_p'
\end{align*}
We can now move on to showing that the creation operator generates the different states. Let's assume that the groundstate exists. Then it should be an eigenstate of the Hamiltonian.
\begin{align*}
    \hat{H} |0\rangle = E_0 |0\rangle
\end{align*}
We can then apply the creation operator.
\begin{align*}
    (\hat{H} a_p^\dagger) |0\rangle = (a_p^\dagger \hat{H} + [\hat{H}, a_p^\dagger]) |0\rangle = a_p^\dagger \hat{H} |0\rangle + \omega_p a_p^\dagger |0\rangle = a_p^\dagger (E_0 + \omega_p) |0\rangle
\end{align*}
Lets find out what $E_0$ is. We apply the hamiltonian operator to the ground state.
\begin{align*}
    \hat{H} |0\rangle &= \int \frac{d^3p}{(2\pi)^3} \omega_p \left(a_p^\dagger a_p |0\rangle + \frac{1}{2} [a_p, a_p^\dagger] |0\rangle+ c_p^\dagger c_p |0\rangle + \frac{1}{2} [c_p, c_p^\dagger] |0\rangle\right) \\
    &= \int \frac{d^3p}{(2\pi)^3} \omega_p \left( \frac{1}{2} [a_p, a_p^\dagger] + \frac{1}{2} [c_p, c_p^\dagger] |0\rangle\right) \\
    &= \int \frac{d^3p}{(2\pi)^3} 2 \cdot \frac{1}{2} (2\pi)^3 \delta^{(3)}(0) |0\rangle \\
    &= \omega_0 |0\rangle
\end{align*}
\end{solution}
\begin{exercise}[2.10]
Show that $a^\dagger_{\mathbf{p}_1} a^\dagger_{\mathbf{p}_2} |0\rangle = a^\dagger_{\mathbf{p}_2} a^\dagger_{\mathbf{p}_1} |0\rangle$ and argue that this implies that these spin zero particles obey Bose-Einstein statistics.
\end{exercise}
\begin{exercise}[2.11]
Apply $\phi(\mathbf{x})$ to the vacuum and show that
\[
\phi(\mathbf{x}) |0\rangle = \int \frac{d^3p}{(2\pi)^3} \frac{1}{\sqrt{2\omega_p}} e^{-i\mathbf{p}\cdot\mathbf{x}} |\mathbf{p}\rangle_c, \tag{29}
.\] 
where $|\mathbf{p}\rangle_c = \sqrt{2\omega_p} c^\dagger_{\mathbf{p}} |0\rangle$ where the subscript $c$ indicates that we are creating a $c$ particle. The factor $\sqrt{2\omega_p}$ is introduced to ensure that the states are normalized in a Lorentz invariant fashion (more precisely, $\langle \mathbf{p}|\mathbf{q} \rangle = (2\pi)^3 2\omega_p \delta^3(\mathbf{p} - \mathbf{q})$ can be shown to be Lorentz invariant, just consider a boost operation along one direction). Likewise, if we use $\phi^\dagger(\mathbf{x})$ we would be creating an $a$ particle. Argue that for small momenta, $\mathbf{p}$, $\omega_p$ is nearly constant and in that case the above expression is a linear superposition of plane wave states with well-defined momentum which is the Fourier transform of a non-relativistic basis state of position, $\mathbf{x}$. We thus interpret $\phi(\mathbf{x})$ as a field operator that creates a particle at position $\mathbf{x}$.
\end{exercise}
\begin{solution}
\[
\phi(x) | 0 \rangle = \int \frac{d^3p}{(2\pi)^3} \frac{1}{\sqrt{2\omega_p}} 
\left( a_p e^{i\mathbf{p} \cdot \mathbf{x}} + c_p^\dagger e^{-i\mathbf{p} \cdot \mathbf{x}} \right) | 0 \rangle
\]

\[
= \int \frac{d^3p}{(2\pi)^3} \frac{1}{\sqrt{2\omega_p}} e^{-i\mathbf{p} \cdot \mathbf{x}} c_p^\dagger | 0 \rangle
\]

Now inserting \(| \mathbf{p} \rangle_c = \sqrt{2\omega_p} c_p^\dagger | 0 \rangle\), we get:
\[
= \int \frac{d^3p}{(2\pi)^3} \frac{1}{2\omega_p} e^{-i\mathbf{p} \cdot \mathbf{x}} | \mathbf{p} \rangle_c
\]
\end{solution}
\begin{exercise}[2.12]
12) Show that $\langle 0| \phi(\mathbf{x}) |\mathbf{p}\rangle = e^{i\mathbf{p}\cdot\mathbf{x}}$. If we interpret this as the position-space representation of the single-particle wave function of the state $|\mathbf{p}\rangle$, then we see that $\langle \mathbf{x}|\mathbf{p}\rangle \propto e^{i\mathbf{p}\cdot\mathbf{x}}$ is the wave function just as in non-relativistic quantum mechanics.
\end{exercise}
\begin{solution}
\begin{align*}
    \langle 0 | \phi(x) | p \rangle &= \left( \langle p | \phi(x)^\dagger | 0 \rangle \right)^\dagger \\
                                    &= \left(\langle p | \int \frac{d^3p'}{(2\pi)^3} \frac{1}{\sqrt{2\omega_{p'}}} e^{i\mathbf{p}' \cdot \mathbf{x}} | p' \rangle   \right) ^\dagger \\
                                    &= \left(   \int \frac{d^3p'}{(2\pi)^3} \frac{1}{\sqrt{2\omega_{p'}}} e^{i\mathbf{p}' \cdot \mathbf{x}} \langle p | p' \rangle\right)^\dagger \\
                                    &= \left(   \int \frac{d^3p'}{(2\pi)^3} \frac{1}{\sqrt{2\omega_{p'}}} e^{i\mathbf{p}' \cdot \mathbf{x}} (2\pi)^3 2\omega_{p'} \delta^3(\mathbf{p}' - \mathbf{p})\right)^\dagger \\
                                    &= \left( e^{i\mathbf{p} \cdot \mathbf{x}} \right)^\dagger = e^{-i\mathbf{p} \cdot \mathbf{x}}
\end{align*}
\end{solution}
\end{document}
\begin{exercise}[3.1]
The previous problem showed how to quantize the Klein-Gordon field in the 
Schrödinger picture of quantum mechanics where the operators are independent 
of time while the state vectors carry all the time-dependence. Here we will 
consider the (equivalent) Heisenberg picture where the state vectors are 
time-independent and the operators carry the time-dependence. The definition 
of an operator in the Heisenberg picture is straightforward
\[
\mathcal{O}_H(x) = \mathcal{O}_H(x,t) = e^{iHt} \mathcal{O}_S(x) e^{-iHt}, \tag{30}
.\]
\color{foreground} 
where $\mathcal{O}_S(x)$ is an operator in the Schrödinger picture and $H$ is the 
Hamiltonian operator which we assume has no explicit time-dependence in this 
problem. Assume that $\mathcal{O}$ does not have any explicit dependence on time $t$. 
Derive the Heisenberg equation of motion
\[
i \frac{\partial }{\partial t} \mathcal{O}_H = \left[ \mathcal{O}_H, H \right] 
.\] 
\end{exercise}
\begin{solution}
\begin{align*}
    i\frac{\partial}{\partial t} \left( e^{i\hat{H}t} \hat{O}(x) e^{-i\hat{H}t} \right) &= i\hat{H}e^{i\hat{H}t}\hat{O}\left( x \right) e^{-i\hat{H}t} +e^{i\hat{H}t}\hat{O}\left( x \right)\left( -i\hat{H} \right)  e^{-i\hat{H}t}  \\
&= -i \left( \hat{H} \hat{O}_H - \hat{O}_H \hat{H} \right) \\
&= [\hat{O}_H, \hat{H}]
\end{align*}
\end{solution}
\begin{exercise}[3.2]
The quantized version of the Hamiltonian for the Klein-Gordon field follows 
from Eq. (8) above and is:
\[
H = \int d^3x \left( \pi(x, t)^\dagger \pi(x, t) 
+ \nabla \phi(x, t)^\dagger \cdot \nabla \phi(x, t) 
+ m^2 \phi(x, t)^\dagger \phi(x, t) \right). \tag{32}
\]

Calculate \( [\phi(x, t), H] \) and show that:
\[
i \frac{\partial}{\partial t} \phi(x, t) = i \pi(x, t)^\dagger. \tag{33}
\]

You will need the equal-time commutator in Eq. (11) and the fact that all 
combinations like \( [\phi, \pi^\dagger] = 0 \) vanish.

\end{exercise}
\begin{solution}
Calculate,
\[
[\phi(x, t), H]
\]

where
\[
H = \int d^3x \left( \pi^\dagger \pi + \nabla \phi^\dagger \cdot \nabla \phi + m^2 \phi^\dagger \phi \right)
\]

Step-by-step:
\[
[\phi, H] = \int d^3x \left( [\phi, \pi^\dagger \pi] + [\phi, \nabla \phi^\dagger \cdot \nabla \phi] + [\phi, m^2 \phi^\dagger \phi] \right)
\]

The second and third terms are zero:
\[
[\phi, H] = \int d^3x [\phi, \pi^\dagger \pi] = \int d^3x \left( \pi^\dagger [\phi, \pi] + [\phi, \pi^\dagger] \pi \right)
\]

Using the canonical commutation relation \( [\phi(x), \pi(x')] = i \delta^3(x - x') \), we get:
\[
[\phi, H] = \int d^3x \pi^\dagger i \delta^3(x - x') = i \pi^\dagger(x', t)
\]

We also know that:
\[
i \frac{\partial}{\partial t} \phi(x, t) = [\phi(x, t), H] = i \pi^\dagger(x', t)
\]
\end{solution}
\begin{exercise}[3.3]
3) By calculating the commutator \( [\pi(x, t), H] \), show that:
\[
i \frac{\partial}{\partial t} \pi(x, t) = -i \left( -\nabla^2 + m^2 \right) \phi(x, t)^\dagger. \tag{34}
\]

(Hint: You will need to do partial integration and throw away a boundary term 
which we assume vanishes at infinity.)
\end{exercise}
\begin{solution}

\end{solution}
\end{document}
