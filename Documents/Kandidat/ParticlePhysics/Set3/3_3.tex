\documentclass[working, oneside]{../../../Preambles/tuftebook}
% Import xcolor and define some colors
\usepackage{{xcolor}}
\definecolor{{background}}{{HTML}}{{{background}}}
\definecolor{{foreground}}{{HTML}}{{{foreground}}}
\definecolor{{math}}{{HTML}}{{{color6}}}

%%%%%%%%%%%%%%%%%%%%%%%%%%%%%%%%%%%%%%%% IMPORTS %%%%%%%%%%%%%%%%%%%%%%%%%%%%%%%%%%%%%%%%
\documentclass[11pt,onesize,a4paper,titlepage]{article}

%%%%%%%%%%%%%%% Formatting %%%%%%%%%%%%%%% 
\usepackage[english]{babel}
\usepackage[utf8]{inputenc}
\usepackage{adjustbox}
\usepackage{geometry} % Margins
\usepackage{sectsty} % Custom Sections

%%%%%%%%%%%%%%% Font %%%%%%%%%%%%%%% 
\usepackage{Archivo}
\usepackage[T1]{fontenc}
\sffamily

%%%%%%%%%%%%%%% Graphics %%%%%%%%%%%%%%% 
\usepackage{fontawesome5} % Icons
\usepackage{graphicx} % Images
\usepackage[most]{tcolorbox} % Color Box
\usepackage{xcolor} % Colors
\usepackage{tikz} % For Drawing Shapes
%%%\usepackage{emoji} % For flags
\tcbuselibrary{breakable}
%%%\usepackage{academicons}

%%%%%%%%%%%%%%% Miscelanous %%%%%%%%%%%%%%% 
\usepackage{lipsum} % Lorem Ipsum
\usepackage{hyperref} % For Hyperlinks

%%%%%%%%%%%%%%% Colors %%%%%%%%%%%%%%% 
\definecolor{title}{HTML}{b5bff5} % Color of the title
\definecolor{bars}{HTML}{889af0} % Color of the title
\definecolor{backdrop}{HTML}{f2f2f2} % Color of the side column
\definecolor{lightgray}{HTML}{dfdfdf} % Color for the skill bars

%%% TU green: #639a00
%%% TU gray: #e6e6e6
%\definecolor{title}{HTML}{639a00} % Color of the title TU
%\definecolor{bars}{HTML}{889af0} % Color of the title TU

% \definecolor{backdrop}{HTML}{f2f2f2} % Color of the side column
\definecolor{backdrop}{HTML}{e6e6e6} % Color of the side column

\definecolor{subtitle}{HTML}{606060} % 


%%%%%%%%%%%%%%% Section Format %%%%%%%%%%%%%%% 
\sectionfont{                     
    \LARGE % Font size
    \sectionrule{0pt}{0pt}{-8pt}{1pt} % Rule under Section name
}

\subsectionfont{
    \Large % Font size
    \fontfamily{phv}\selectfont % Font family
    %\sectionrule{0pt}{0pt}{-8pt}{1pt} % Rule under Subsection name
    \sectionrule{5pt}{0pt}{0pt}{0pt} % Rule under Subsection name
}

%%%%%%%%%%%%%%% Margins and Headers %%%%%%%%%%%%%%%
\geometry{
  a4paper,
  left=7mm,
  right=7mm,
  bottom=10mm,
  top=10mm
}

\pagestyle{empty} % Empty Headers

\usepackage{marvosym}

% \renewcommand\qedsymbol{\CoffeeCup}

\usepackage{changepage}

\newenvironment{subexercise}[1]{%
    \begin{mdframed}[linewidth=0.5pt, linecolor=foreground, backgroundcolor=background, leftmargin=0cm, innerleftmargin=1em, innertopmargin=0pt, innerbottommargin=0pt, innerrightmargin=0pt, topline=false, rightline=false, bottomline=false]
    \par\noindent\textcolor{foreground}{\textbf{#1.}}\hspace{1em}\ignorespaces
}{%
    \par\addvspace{\baselineskip}\end{mdframed}\ignorespacesafterend
}
\newenvironment{solution}{%
    % \par\addvspace{\baselineskip}\noindent\makebox[\textwidth]{\textcolor{foreground}{\textbullet\hspace{1em}\textbullet\hspace{1em}\textbullet}}\par\addvspace{\baselineskip}
    \begin{mdframed}[linewidth=0.5pt, linecolor=foreground, backgroundcolor=background, rightmargin=0cm, innerleftmargin=0cm, innertopmargin=0pt, innerbottommargin=0pt, innerrightmargin=1em, topline=false, leftline=false, bottomline=false]
    \par\noindent\textcolor{foreground}{\textit{Solution.}}\hspace{1em}\ignorespaces
}{%
    \par\addvspace{\baselineskip}\noindent\hfill\textcolor{foreground}{\Coffeecup}\par\addvspace{\baselineskip}\end{mdframed}\ignorespacesafterend
}
% Exercise environment

\declaretheoremstyle[
    name= \textcolor{foreground}{Exercise},
    postheadspace = \newline,
    bodyfont = \normalfont\color{foreground},
    postheadhook={\textcolor{math}{\rule[.4ex]{\linewidth}{0.5pt}}\\},
    % numberwithin=chapter,
    mdframed={
        backgroundcolor = background,
        linecolor = foreground,
        linewidth = 0.5pt,
        rightline =  true,
        topline = true,
        bottomline = true,
        skipabove=20pt,
        skipbelow=20pt,
        innerleftmargin=15pt,
        innertopmargin=10pt,
        innerrightmargin=15pt,
        innerbottommargin=10pt}
    ]{exercise}
\declaretheorem[style=exercise,numbered=no]{exercise}

% \etocsetlevel{exercise}{2}

% \AtEndEnvironment{exercise}{%
%   \etoctoccontentsline{exercise}{\protect\numberline{\theexercise}}%
% }%
% \etocsetstyle{exercise}
% {}
% {}
% % this will be rendered like a non-numbered section, but we could have used
% % \numberline here also
% {\etocsavedsectiontocline{Exercise \etocnumber}{\etocpage}}
%     {}

% theorem environment

\declaretheoremstyle[
    name= \textcolor{foreground}{Theorem},
    postheadspace = \newline,
    bodyfont = \normalfont\color{foreground},
    postheadhook={\textcolor{math}{\rule[.4ex]{\linewidth}{1pt}}\\},
    mdframed={
        backgroundcolor = background,
        linecolor = foreground,
        linewidth = 1pt,
        rightline =  true,
        topline = true,
        bottomline = true,
        skipabove=20pt,
        skipbelow=20pt,
        innerleftmargin=15pt,
        innertopmargin=10pt,
        innerrightmargin=15pt,
        innerbottommargin=10pt}
    ]{theorem}
\declaretheorem[style=theorem,numbered=yes]{theorem}

\declaretheoremstyle[
    name= \textcolor{foreground}{Definition},
    postheadspace = \newline,
    bodyfont = \normalfont\color{foreground},
    postheadhook={\textcolor{math}{\rule[.4ex]{\linewidth}{1pt}}\\},
    mdframed={
        backgroundcolor = background,
        linecolor = foreground,
        linewidth = 1pt,
        rightline =  true,
        topline = true,
        bottomline = true,
        skipabove=20pt,
        skipbelow=20pt,
        innerleftmargin=15pt,
        innertopmargin=10pt,
        innerrightmargin=15pt,
        innerbottommargin=10pt}
    ]{definition}
\declaretheorem[style=definition,numbered=yes]{definition}
% Example environment

\declaretheoremstyle[
name= \quad \underline{Proof:},
     headfont = \bfseries\sffamily,
     postheadspace = \newline,
     % notebraces = \bfseries{(}{)a},
     headpunct = {},
     bodyfont = ,
     postheadhook={\textcolor{foreground}{\rule[0.4ex]{\linewidth}{0pt}}\\},
     qed=\qedsymbol,
    % spacebelow = 10pt,
    mdframed={
  backgroundcolor = background,
  linecolor = foreground,
  linewidth = 1pt,
  skipabove=10pt,
  skipbelow=10pt,
  rightline = false,
  topline = false,
  leftline = false,
  bottomline = false,
  innerleftmargin=15pt,
  innertopmargin=15pt,
  innerrightmargin=15pt,
  innerbottommargin=15pt}
]{pro}
    % \declaretheorem[style=pro,numbered=no]{Proof}

\declaretheoremstyle[
name= \quad \underline{\textcolor{foreground}{Example}},
     headfont = \bfseries\sffamily,
     postheadspace = \newline,
     % notebraces = \bfseries{(}{)a},
     headpunct = {},
     bodyfont = \normalfont\color{foreground},
     postheadhook={\textcolor{foreground}{\rule[0.4ex]{\linewidth}{0pt}}\\},
     % spacebelow = 10pt,
    mdframed={
  backgroundcolor = background,
  linecolor = foreground,
  linewidth = 1pt,
  skipabove=10pt,
  skipbelow=10pt,
  rightline = false,
  topline = false,
  leftline = false,
  bottomline = false,
  innerleftmargin=15pt,
  innertopmargin=15pt,
  innerrightmargin=15pt,
  innerbottommargin=15pt}
]{ex}
\declaretheorem[style=ex,numbered=no]{example}

\declaretheoremstyle[
     name=,
     headfont = \bfseries\sffamily,
     notebraces = \bfseries{},
     headpunct = { -},
     bodyfont = \color{foreground}\normalfont,
     % postheadhook={\textcolor{black}{\rule[.4ex]{\linewidth}{0.2pt}}\\},
    % spacebelow = 10pt,
    mdframed={
  backgroundcolor = background,
  linecolor = foreground,
  linewidth = 1pt,
  skipabove=0pt,
  skipbelow=0pt,
  innerleftmargin=10pt,
  innertopmargin=10pt,
  innerrightmargin=10pt,
  innerbottommargin=10pt,
  rightline = false,
  topline = false,
  leftline = false,
  bottomline = true}
]{subexercise}
% \declaretheorem[style=subexercise,numbered=no]{subexercise}

\declaretheoremstyle[
     name= \color{losning}Løsning,
     headfont = \bfseries\sffamily,
     notebraces = \bfseries{},
     postheadspace = \newline,
     headpunct = {:},
     bodyfont = \normalfont,
     % qed = ,
     % postheadhook={\textcolor{black}{\rule[.4ex]{\linewidth}{0.2pt}}\\},
    % spacebelow = 10pt,
    mdframed={
  backgroundcolor = background,
  linecolor = losning!75,
  linewidth = 1pt,
  skipabove=0pt,
  skipbelow=10pt,
  innerleftmargin=10pt,
  innertopmargin=10pt,
  innerrightmargin=10pt,
  innerbottommargin=10pt,
  leftline = false,
  rightline = true,
  topline = false,
  bottomline = true}
]{solution}

\newenvironment{SimpleBox}[1]{%
  \begin{mdframed}%
    \noindent\textbf{#1}\\[1ex]
}{%
  \end{mdframed}%
}


\begin{document}
\let\cleardoublepage\clearpage
\thispagestyle{fancy}
\chapter{3 - Dirac Spinors in the Weyl Representation}

Here we explore a different representation of the Dirac matrices, \( \gamma_\mu \), that will generate slightly different expressions for the explicit solutions. Let us define the so-called Weyl representation
\begin{align*}
\gamma^0
&= \begin{pmatrix} 0 & 1 \\ 1 & 0 \end{pmatrix}, \quad \gamma^i = \begin{pmatrix} 0 & \sigma_i \\ -\sigma_i & 0 \end{pmatrix}, \quad \text{and} \quad \gamma_5 = \begin{pmatrix} -1 & 0 \\ 0 & 1 \end{pmatrix}. \tag{18}
\end{align*}
We take the positive energy solutions to have the form
\begin{align*}
\psi_s
&= u_s(p) e^{-i p x} = (\not{p} + m) \begin{bmatrix} \chi_s \\ 0 \end{bmatrix} e^{-i p x}, \tag{19}
\end{align*}
and the negative energy ones to be
\begin{align*}
\psi_s
&= v_s(p) e^{i p x} = (\not{p} - m) \begin{bmatrix} 0 \\ \chi_{-s} \end{bmatrix} e^{i p x}, \tag{20}
\end{align*}
where \( \chi_s \) are 2-spinors with quantization axis along the direction of momentum \( \mathbf{p} \) with projection \( s = \pm 1 \).

\begin{exercise}[1]
Find the explicit form of the spinors if we insist on the normalization \( \overline{\psi} \psi = 2m \) for positive energy solutions and \( \overline{\psi} \psi = -2m \) for negative energy solutions.
\end{exercise}
\begin{solution}
\begin{align*}
\Psi_s
&= A (\not{p} + m) \begin{bmatrix} \chi_s \\ 0 \end{bmatrix} e^{-i p x} \\
\overline{\Psi} \Psi
&= \Psi^\dagger \gamma^0 \Psi = A^2 \Psi^\dagger \gamma^0 (\not{p} + m) \begin{bmatrix} \chi_s \\ 0 \end{bmatrix} \\
&= A^2 \Psi^\dagger \gamma^0 (\not{p} + m\mathbb{I}) \begin{bmatrix} \chi_s \\ 0 \end{bmatrix} \\
&= A^2 \Psi^\dagger \gamma^0 \begin{bmatrix}
    E+m & -\mathbf{\sigma}\cdot \mathbf{p} \\
     \mathbf{\sigma}\cdot \mathbf{p}& m -E  \\
\end{bmatrix}
\begin{bmatrix} \chi_s \\ 0 \end{bmatrix} \\
&= A^2 \Psi^\dagger  \begin{bmatrix}
    E+m & -\mathbf{\sigma}\cdot \mathbf{p} \\
     \mathbf{\sigma}\cdot \mathbf{p}& E - m  \\
\end{bmatrix}
\begin{bmatrix} \chi_s \\ 0 \end{bmatrix} \\
&= A^2 \Psi^\dagger 
\begin{bmatrix}
    \left( E + m \right) \chi _s  \\
    \mathbf{\sigma}\cdot \mathbf{p }\chi _s \\
\end{bmatrix} = A^2 \left( \left( \not{p} + \mathbb{I} \right) ^\dagger
\begin{bmatrix}
    \chi _s \\
    0 \\
\end{bmatrix}
\right)  
\begin{bmatrix}
    \left( E + m \right) \chi _s  \\
    \mathbf{\sigma}\cdot \mathbf{p }\chi _s \\
\end{bmatrix} \\
&=  A^2
\begin{bmatrix}
    \chi _s^\dagger&0 \\
\end{bmatrix}
\begin{bmatrix}
    E+m & -\mathbf{\sigma}\cdot \mathbf{p} \\
     \mathbf{\sigma}\cdot \mathbf{p}& m -E  \\
\end{bmatrix}^{T}
\begin{bmatrix}
    \left( E + m \right) \chi _s  \\
    \mathbf{\sigma}\cdot \mathbf{p }\chi _s \\
\end{bmatrix} \\
&=  A^2
\begin{bmatrix}
    \chi _s^\dagger&0 \\
\end{bmatrix}
\begin{bmatrix}
    E+m & \mathbf{\sigma}\cdot \mathbf{p} \\
     -\mathbf{\sigma}\cdot \mathbf{p}& m -E  \\
\end{bmatrix}
\begin{bmatrix}
    \left( E + m \right) \chi _s  \\
    \mathbf{\sigma}\cdot \mathbf{p }\chi _s \\
\end{bmatrix} \\
&=
\begin{bmatrix}
    \chi _s^\dagger&0 \\
\end{bmatrix}
\begin{bmatrix}
    \left( E + m \right) ^2 \chi _s + \left( \mathbf{\sigma} \cdot \mathbf{p} \right) ^2 \chi _s \\
    -\left( \mathbf{\sigma} \cdot \mathbf{p} \right) \left( E + m \right) \chi _s + \left( \mathbf{\sigma} \cdot \mathbf{p} \right)\left( m - E \right)  \chi _s \\
\end{bmatrix} \\
    =& A^2 \left( \left( E + m \right) ^2 + \left( \mathbf{\sigma}\cdot \mathbf{p} \right) ^2 \right) = A^2\left( E^2 + m^2 + 2mE + p^2 \right) = A^2\left( 2E\left( E+m \right)  \right) = 2m \\ 
% &= \begin{bmatrix} (\not{p} + m)^\dagger \begin{bmatrix} \chi_s \\ 0 \end{bmatrix} \end{bmatrix}^\dagger \begin{bmatrix} (E + m) \chi_s \\ \boldsymbol{\sigma} \cdot \mathbf{p} \chi_s \end{bmatrix}, \\
% &= A^2 \begin{bmatrix} \chi_s^\dagger & 0 \end{bmatrix} \begin{bmatrix} (E + m) & -\boldsymbol{\sigma} \cdot \mathbf{p} \\ \boldsymbol{\sigma} \cdot \mathbf{p} & (E - m) \end{bmatrix} \begin{bmatrix} (E + m) \chi_s \\ \boldsymbol{\sigma} \cdot \mathbf{p} \chi_s \end{bmatrix}, \\
% &= A^2 \begin{bmatrix} \chi_s^\dagger & 0 \end{bmatrix} \begin{bmatrix} (E + m) \chi_s \\ \boldsymbol{\sigma} \cdot \mathbf{p} \chi_s \end{bmatrix}, \\
% &= A^2 \left[ \chi_s^\dagger \quad 0 \right] \begin{bmatrix} (E + m) & \boldsymbol{\sigma} \cdot \mathbf{p} \\ \boldsymbol{\sigma} \cdot \mathbf{p} & m - E \end{bmatrix} \begin{bmatrix} (E + m) \chi_s \\ \boldsymbol{\sigma} \cdot \mathbf{p} \chi_s \end{bmatrix}, \\
% &= A^2 \begin{bmatrix} \chi_s^\dagger & 0 \end{bmatrix} \begin{bmatrix} (E + m) \chi_s \\ \boldsymbol{\sigma} \cdot \mathbf{p} \chi_s \end{bmatrix}, \\
% &= A^2 \left[ \chi_s^\dagger \quad 0 \right] \begin{bmatrix} (E + m)^2 \chi_s + (\boldsymbol{\sigma} \cdot \mathbf{p} \chi_s) \\ -(\boldsymbol{\sigma} \cdot \mathbf{p}) (E + m) \chi_s + (M - E) \boldsymbol{\sigma} \cdot \mathbf{p} \chi_s \end{bmatrix}, \\
% &= A^2 \left[ (E + m)^2 + (\boldsymbol{\sigma} \cdot \mathbf{p}^2) \right], \\
% &= A^2 (E^2 + m^2 + 2mE + \mathbf{p}^2) = A^2 (2E (E + M)) = 2m, \\
\Rightarrow A
&= \sqrt{\frac{2m}{2E (E + m)}} = \sqrt{\frac{m}{E (E + m)}}
\end{align*}
\end{solution}

\begin{exercise}[2]
Find the solutions in the massless limit, \( m = 0 \). How does the Dirac equation simplify in this limit?
\end{exercise}

\begin{exercise}[3]
The Dirac Hamiltonian operator has the form \( H_D = -i \gamma^0 \boldsymbol{\gamma} \cdot \nabla + \gamma^0 m \). Introduce the helicity operator, \( h = \boldsymbol{\sigma} \cdot \mathbf{p} / |\mathbf{p}| \). Show that the Dirac operator and \( h \) commute. Note that the while \( h \) looks like a \( 2 \times 2 \) matrix due to the \( \boldsymbol{\sigma} \) part, when we apply it to 4-spinors it is understood that it is a \( 4 \times 4 \) matrix also (it has an implicit \( 2 \times 2 \) matrix multiplied on it). Argue that the spinors, \( \chi_s \), defined above are in fact the helicity eigenfunctions.
\end{exercise}

\begin{exercise}[4]
Look the solutions in the massless limit from 2). Determine the helicity of the four solutions when \( m = 0 \). How is the spin and helicity connected for positive and negative energy solutions?
\end{exercise}

\end{document}
