\documentclass[working, oneside]{../../../Preambles/tuftebook}
% Import xcolor and define some colors
\usepackage{xcolor}
\definecolor{background}{HTML}{ffffff}
\definecolor{foreground}{HTML}{000000}
\definecolor{math}{HTML}{000000}

%%%%%%%%%%%%%%%%%%%%
%% SUPER PREAMBLE %%
%%%%%%%%%%%%%%%%%%%%

% \usepackage[utf8]{inputenc}
\usepackage[T1]{fontenc} % Fonts and stuff
\usepackage{amsmath, amsfonts, mathtools, amsthm, amssymb} % math

\usepackage{fancyhdr} % Header, Footer etc.
\usepackage{adforn}
\usepackage{efbox}
\usepackage{lastpage}
\usepackage{marvosym}
\usepackage{pict2e}
\usepackage{caption}
\usepackage{wrapfig}
\usepackage{graphicx}
\usepackage{sidecap}
% \usepackage{mathpazo} 
% \usepackage{cmbright}
\usepackage{mathptmx}

\usepackage[
    sorting=nyt,
    style=alphabetic
]{biblatex}
\addbibresource{references.bib}
\usepackage[noabbrev]{cleveref}

\pagestyle{fancy}
\fancyhead[R]{}
\fancyhead[L]{}
\fancyfoot[C]{\efbox[margin = 10pt,
                    topline = false,
                    leftline = false,
                    rightline = false,
                    backgroundcolor = background,
                    linewidth = 1pt,
                    linecolor = foreground]{\thepage\ of \pageref{LastPage}}}
% \fancyfoot[C]{\color{foreground} \thepage}

% \renewcommand{\headrule}{%
% 	\hrulefill
% }
% \renewcommand{\footrulewidth}{0pt}
\renewcommand{\headrulewidth}{0pt}

% \setlength{\headheight}{15pt}
% \setlength{\footheight}{15pt}

%% Margin Control %%

% \def\changemargin#1#2{\list{}{\rightmargin#2\leftmargin#1}\item[]}
% \let\endchangemargin=\endlist


%%%%%%%%%%%%%%%%%%%%%%%%%%%%%%%%%%%%%%%%%%%%%%%%%%%%%%%%%%

% figure support

\usepackage{import}
\usepackage{transparent}


\newcommand{\incfig}[2][1]{%
    \def\svgwidth{#1\columnwidth}
    \import{figures/}{#2.pdf_tex}
}

% \pdfsuppresswarningpagegroup=1

%%%%%%%%%%%%%%%%%%%%%%%%%%%%%%%%%%%%%%%%%%%%%%%%%%%%%%%%%%

\usepackage{tikzsymbols} % Symbols
\usepackage[framemethod=TikZ]{mdframed} % Boxes around theorem environments
\usepackage{thmtools}




% \everymath{\color{math}}
% \everydisplay{\color{math}}
% \def\m@th{\normalcolor\mathsurround\z@}

\color{foreground}

	% \end{changemargin}
	% } 

 % \newenvironment{subexercise}[1]
 % {\noindent
	 % \textbf{(#1)} \quad \adforn{10} \quad \em
 % }{}

% Mathematical typesetting stuff.

 % \newcommand{\dd}{\mathrm{\textbf{d}}}

 % Change font

% \usepackage{tgadventor}
% \usepackage{cmbright}
% \usepackage{bm}

% \usepackage{microtype} % Microtypography
% \usepackage{fontspec}% Hyperlinks
% \usepackage{fouriernc}

% \def\MT@set@inh@list#1#2{%
%   \MT@ifempty\MT@inh@feat{%
%     \MT@map@clist@c\MT@features{\begingroup % <--
%       \MT@ifstreq{##1}{tr}\relax{\MT@declare@char@inh{##1}{#1}{#2}}%
%     \endgroup}% <--
%   }{%
%     \MT@map@clist@c\MT@inh@feat{\begingroup % <--
%       \KV@@sp@def\@tempa{##1}%
%       \MT@ifempty\@tempa\relax{%
%         \edef\@tempa{\csname MT@rbba@\@tempa\endcsname}%
%         \MT@ifstreq\@tempa{tr}\relax{%
%           \MT@exp@one@n\MT@declare@char@inh{\@tempa}{#1}{#2}}}%
%     \endgroup}% <--
%   }%      
% \DeclareCaptionFormat{custom}{\bfseries#1#2\itshape#3}%   \MT@end@catcodes
\DeclareCaptionFormat{custom}{\bfseries\itshape#1#2\normalfont\small#3}
\captionsetup{
    format=custom,
    labelsep=space,
    width=\textwidth, % Set the caption width to be 80% of the text width
    % justification=jusitified, % Center-align the caption
    % font=it % Italicize the caption text
}
% }
\usepackage{setspace}
\DeclareCaptionFormat{margin}{\small\bfseries#1#2#3}

\usepackage{xparse} % For advanced command definitions with optional arguments

\NewDocumentCommand{\marginfig}{O{0cm} m m m}{%
  % #1 = optional padding (default 0cm), #2 = filename, #3 = label, #4 = caption
  \marginpar{%
    \includegraphics[width=\marginparwidth]{#2}%
    \captionsetup{format=custom, labelsep=space, width=\marginparwidth, justification=raggedright, font=small}
    \captionof{figure}{#4}%
    \label{fig:#3}%
    % \rule{\marginparwidth}{0.4pt} % Adds a line below the caption
    \vspace{#1} % Adds the specified padding below the caption
  }%
}

\NewDocumentCommand{\maintextfig}{O{0cm} m m m}{
  % #1 = optional vertical adjustment for the caption (default 0cm)
  % #2 = filename for the figure
  % #3 = label for the figure
  % #4 = caption text

  % Place the figure in the text
  \begin{figure}[htbp]
    \centering
    \includegraphics[width=\textwidth]{#2}
    \marginnote{\captionsetup{format=custom, labelsep=space, width=\marginparwidth, justification=fill, font={stretch=1}}
        \captionof{figure}[#4]{#4}\label{fig:#3}}[#1]
    \label{fig:#3}
  \end{figure}

  % Place the caption in the margin
  % \marginnote{\captionsetup{format=custom, labelsep=space, width=\marginparwidth, justification=raggedright, font={stretch=1}}
  %   \captionof{figure}[#4]{#4}\label{fig:#3}}[#1]
}
% \NewDocumentCommand{\marginfig}{m m m}{
%   % #1 = filename, #2 = label, #3 = caption
%   \begin{wrapfigure}{r}{5cm} % "r" for right side, and "5cm" for the width of the figure
%       \centering
%       \includegraphics[width=5cm]{#1}
%      \captionsetup{format=custom, labelsep=space, width=6cm, justification=raggedright, font={stretch=1}}
%       \captionof{figure}{#3}%
%       \label{fig:#2}
%   \end{wrapfigure}
% }{}
% \newcommand{\marginfig}[3]{%
%   \marginpar{%
%     \includegraphics[width=\marginparwidth]{#1}%
%     \captionsetup{format=custom, labelsep=space, width=\marginparwidth, justification=raggedright, font={stretch=1}}
%     \captionof{figure}{\fontsize{11pt}{11pt}\selectfont #3}%
%     \label{fig:#2}%
%     % \rule{\marginparwidth}{0.4pt}
%   }%
% }
% \newcommand{\marginfig}[4][0pt]{%
%   \marginpar{%
%     \raisebox{#1}{%
%       \includegraphics[width=\marginparwidth]{#2}%
%       \captionsetup{format=margin, labelsep=space, justification=raggedright}
%       \captionof{figure}{#4}%
%       \label{fig:#3}%
%     }%
%   }%
% }
% \newcommand{\marginfig}[2][0pt]{%
%   \marginpar{\raisebox{#1}{%
%       \includegraphics[width=1.0\marginparwidth]{#2}
%       % \label{fig:#3}
%       % \caption{#4}
%       % \parbox{\marginparwidth}{\smaller \textbf{figure:} #3}%
%   }}%
% }
\newcommand{\margintext}[2][0pt]{%
  \marginpar{\raisebox{#1}{%
    \parbox{\marginparwidth}{\smaller \textbf{figure:} #2}%
    }}%
}

\newcommand{\marginmath}[2][0pt]{%
  \marginpar{\raisebox{#1}{%
    \parbox{1.2\marginparwidth}{#2}%
    }}%
}
\pagecolor{background}
\usepackage{listings}
\definecolor{commentsColor}{rgb}{0.497495, 0.497587, 0.497464}
\definecolor{keywordsColor}{rgb}{0.000000, 0.000000, 0.635294}
\definecolor{stringColor}{rgb}{0.558215, 0.000000, 0.135316}
\renewcommand*\ttdefault{txtt}
\lstset{
  basicstyle=\ttfamily\small,                   % the size of the fonts that are used for the code
  breakatwhitespace=false,                      % sets if automatic breaks should only happen at whitespace
  breaklines=true,                              % sets automatic line breaking
  frame=tb,                                     % adds a frame around the code
  commentstyle=\color{commentsColor}\textit,    % comment style
  keywordstyle=\color{keywordsColor}\bfseries,  % keyword style
  stringstyle=\color{stringColor},              % string literal style
  numbers=left,                                 % where to put the line-numbers; possible values are (none, left, right)
  numbersep=5pt,                                % how far the line-numbers are from the code
  numberstyle=\tiny\color{commentsColor},       % the style that is used for the line-numbers
  showstringspaces=false,                       % underline spaces within strings only
  tabsize=2,                                    % sets default tabsize to 2 spaces
  language=Scala
}


\usepackage{marvosym}

% \renewcommand\qedsymbol{\CoffeeCup}

\usepackage{changepage}

\newenvironment{subexercise}[1]{%
    \begin{mdframed}[linewidth=0.5pt, linecolor=foreground, backgroundcolor=background, leftmargin=0cm, innerleftmargin=1em, innertopmargin=0pt, innerbottommargin=0pt, innerrightmargin=0pt, topline=false, rightline=false, bottomline=false]
    \par\noindent\textcolor{foreground}{\textbf{#1.}}\hspace{1em}\ignorespaces
}{%
    \par\addvspace{\baselineskip}\end{mdframed}\ignorespacesafterend
}
\newenvironment{solution}{%
    % \par\addvspace{\baselineskip}\noindent\makebox[\textwidth]{\textcolor{foreground}{\textbullet\hspace{1em}\textbullet\hspace{1em}\textbullet}}\par\addvspace{\baselineskip}
    \begin{mdframed}[linewidth=0.5pt, linecolor=foreground, backgroundcolor=background, rightmargin=0cm, innerleftmargin=0cm, innertopmargin=0pt, innerbottommargin=0pt, innerrightmargin=1em, topline=false, leftline=false, bottomline=false]
    \par\noindent\textcolor{foreground}{\textit{Solution.}}\hspace{1em}\ignorespaces
}{%
    \par\addvspace{\baselineskip}\noindent\hfill\textcolor{foreground}{\Coffeecup}\par\addvspace{\baselineskip}\end{mdframed}\ignorespacesafterend
}
% Exercise environment

\declaretheoremstyle[
    name= \textcolor{foreground}{Exercise},
    postheadspace = \newline,
    bodyfont = \normalfont\color{foreground},
    postheadhook={\textcolor{math}{\rule[.4ex]{\linewidth}{0.5pt}}\\},
    % numberwithin=chapter,
    mdframed={
        backgroundcolor = background,
        linecolor = foreground,
        linewidth = 0.5pt,
        rightline =  true,
        topline = true,
        bottomline = true,
        skipabove=20pt,
        skipbelow=20pt,
        innerleftmargin=15pt,
        innertopmargin=10pt,
        innerrightmargin=15pt,
        innerbottommargin=10pt}
    ]{exercise}
\declaretheorem[style=exercise,numbered=no]{exercise}

% \etocsetlevel{exercise}{2}

% \AtEndEnvironment{exercise}{%
%   \etoctoccontentsline{exercise}{\protect\numberline{\theexercise}}%
% }%
% \etocsetstyle{exercise}
% {}
% {}
% % this will be rendered like a non-numbered section, but we could have used
% % \numberline here also
% {\etocsavedsectiontocline{Exercise \etocnumber}{\etocpage}}
%     {}

% theorem environment

\declaretheoremstyle[
    name= \textcolor{foreground}{Theorem},
    postheadspace = \newline,
    bodyfont = \normalfont\color{foreground},
    postheadhook={\textcolor{math}{\rule[.4ex]{\linewidth}{1pt}}\\},
    mdframed={
        backgroundcolor = background,
        linecolor = foreground,
        linewidth = 1pt,
        rightline =  true,
        topline = true,
        bottomline = true,
        skipabove=20pt,
        skipbelow=20pt,
        innerleftmargin=15pt,
        innertopmargin=10pt,
        innerrightmargin=15pt,
        innerbottommargin=10pt}
    ]{theorem}
\declaretheorem[style=theorem,numbered=yes]{theorem}

\declaretheoremstyle[
    name= \textcolor{foreground}{Definition},
    postheadspace = \newline,
    bodyfont = \normalfont\color{foreground},
    postheadhook={\textcolor{math}{\rule[.4ex]{\linewidth}{1pt}}\\},
    mdframed={
        backgroundcolor = background,
        linecolor = foreground,
        linewidth = 1pt,
        rightline =  true,
        topline = true,
        bottomline = true,
        skipabove=20pt,
        skipbelow=20pt,
        innerleftmargin=15pt,
        innertopmargin=10pt,
        innerrightmargin=15pt,
        innerbottommargin=10pt}
    ]{definition}
\declaretheorem[style=definition,numbered=yes]{definition}
% Example environment

\declaretheoremstyle[
name= \quad \underline{Proof:},
     headfont = \bfseries\sffamily,
     postheadspace = \newline,
     % notebraces = \bfseries{(}{)a},
     headpunct = {},
     bodyfont = ,
     postheadhook={\textcolor{foreground}{\rule[0.4ex]{\linewidth}{0pt}}\\},
     qed=\qedsymbol,
    % spacebelow = 10pt,
    mdframed={
  backgroundcolor = background,
  linecolor = foreground,
  linewidth = 1pt,
  skipabove=10pt,
  skipbelow=10pt,
  rightline = false,
  topline = false,
  leftline = false,
  bottomline = false,
  innerleftmargin=15pt,
  innertopmargin=15pt,
  innerrightmargin=15pt,
  innerbottommargin=15pt}
]{pro}
    % \declaretheorem[style=pro,numbered=no]{Proof}

\declaretheoremstyle[
name= \quad \underline{\textcolor{foreground}{Example}},
     headfont = \bfseries\sffamily,
     postheadspace = \newline,
     % notebraces = \bfseries{(}{)a},
     headpunct = {},
     bodyfont = \normalfont\color{foreground},
     postheadhook={\textcolor{foreground}{\rule[0.4ex]{\linewidth}{0pt}}\\},
     % spacebelow = 10pt,
    mdframed={
  backgroundcolor = background,
  linecolor = foreground,
  linewidth = 1pt,
  skipabove=10pt,
  skipbelow=10pt,
  rightline = false,
  topline = false,
  leftline = false,
  bottomline = false,
  innerleftmargin=15pt,
  innertopmargin=15pt,
  innerrightmargin=15pt,
  innerbottommargin=15pt}
]{ex}
\declaretheorem[style=ex,numbered=no]{example}

\declaretheoremstyle[
     name=,
     headfont = \bfseries\sffamily,
     notebraces = \bfseries{},
     headpunct = { -},
     bodyfont = \color{foreground}\normalfont,
     % postheadhook={\textcolor{black}{\rule[.4ex]{\linewidth}{0.2pt}}\\},
    % spacebelow = 10pt,
    mdframed={
  backgroundcolor = background,
  linecolor = foreground,
  linewidth = 1pt,
  skipabove=0pt,
  skipbelow=0pt,
  innerleftmargin=10pt,
  innertopmargin=10pt,
  innerrightmargin=10pt,
  innerbottommargin=10pt,
  rightline = false,
  topline = false,
  leftline = false,
  bottomline = true}
]{subexercise}
% \declaretheorem[style=subexercise,numbered=no]{subexercise}

\declaretheoremstyle[
     name= \color{losning}Løsning,
     headfont = \bfseries\sffamily,
     notebraces = \bfseries{},
     postheadspace = \newline,
     headpunct = {:},
     bodyfont = \normalfont,
     % qed = ,
     % postheadhook={\textcolor{black}{\rule[.4ex]{\linewidth}{0.2pt}}\\},
    % spacebelow = 10pt,
    mdframed={
  backgroundcolor = background,
  linecolor = losning!75,
  linewidth = 1pt,
  skipabove=0pt,
  skipbelow=10pt,
  innerleftmargin=10pt,
  innertopmargin=10pt,
  innerrightmargin=10pt,
  innerbottommargin=10pt,
  leftline = false,
  rightline = true,
  topline = false,
  bottomline = true}
]{solution}

\newenvironment{SimpleBox}[1]{%
  \begin{mdframed}%
    \noindent\textbf{#1}\\[1ex]
}{%
  \end{mdframed}%
}


\begin{document}
\let\cleardoublepage\clearpage
\thispagestyle{fancy}
\chapter{3 - Dirac Spinors in the Weyl Representation}

Here we explore a different representation of the Dirac matrices, \( \gamma_\mu \), that will generate slightly different expressions for the explicit solutions. Let us define the so-called Weyl representation
\begin{align*}
\gamma^0
&= \begin{pmatrix} 0 & 1 \\ 1 & 0 \end{pmatrix}, \quad \gamma^i = \begin{pmatrix} 0 & \sigma_i \\ -\sigma_i & 0 \end{pmatrix}, \quad \text{and} \quad \gamma_5 = \begin{pmatrix} -1 & 0 \\ 0 & 1 \end{pmatrix}. \tag{18}
\end{align*}
We take the positive energy solutions to have the form
\begin{align*}
\psi_s
&= u_s(p) e^{-i p x} = (\not{p} + m) \begin{bmatrix} \chi_s \\ 0 \end{bmatrix} e^{-i p x}, \tag{19}
\end{align*}
and the negative energy ones to be
\begin{align*}
\psi_s
&= v_s(p) e^{i p x} = (\not{p} - m) \begin{bmatrix} 0 \\ \chi_{-s} \end{bmatrix} e^{i p x}, \tag{20}
\end{align*}
where \( \chi_s \) are 2-spinors with quantization axis along the direction of momentum \( \mathbf{p} \) with projection \( s = \pm 1 \).

\begin{exercise}[1]
Find the explicit form of the spinors if we insist on the normalization \( \overline{\psi} \psi = 2m \) for positive energy solutions and \( \overline{\psi} \psi = -2m \) for negative energy solutions.
\end{exercise}
\begin{solution}
\begin{align*}
\Psi_s
&= A (\not{p} + m) \begin{bmatrix} \chi_s \\ 0 \end{bmatrix} e^{-i p x} \\
\overline{\Psi} \Psi
&= \Psi^\dagger \gamma^0 \Psi = A^2 \Psi^\dagger \gamma^0 (\not{p} + m) \begin{bmatrix} \chi_s \\ 0 \end{bmatrix} \\
&= A^2 \Psi^\dagger \gamma^0 (\not{p} + m\mathbb{I}) \begin{bmatrix} \chi_s \\ 0 \end{bmatrix} \\
&= A^2 \Psi^\dagger \gamma^0 \begin{bmatrix}
    E+m & -\mathbf{\sigma}\cdot \mathbf{p} \\
     \mathbf{\sigma}\cdot \mathbf{p}& m -E  \\
\end{bmatrix}
\begin{bmatrix} \chi_s \\ 0 \end{bmatrix} \\
&= A^2 \Psi^\dagger  \begin{bmatrix}
    E+m & -\mathbf{\sigma}\cdot \mathbf{p} \\
     \mathbf{\sigma}\cdot \mathbf{p}& E - m  \\
\end{bmatrix}
\begin{bmatrix} \chi_s \\ 0 \end{bmatrix} \\
&= A^2 \Psi^\dagger 
\begin{bmatrix}
    \left( E + m \right) \chi _s  \\
    \mathbf{\sigma}\cdot \mathbf{p }\chi _s \\
\end{bmatrix} = A^2 \left( \left( \not{p} + \mathbb{I} \right) ^\dagger
\begin{bmatrix}
    \chi _s \\
    0 \\
\end{bmatrix}
\right)  
\begin{bmatrix}
    \left( E + m \right) \chi _s  \\
    \mathbf{\sigma}\cdot \mathbf{p }\chi _s \\
\end{bmatrix} \\
&=  A^2
\begin{bmatrix}
    \chi _s^\dagger&0 \\
\end{bmatrix}
\begin{bmatrix}
    E+m & -\mathbf{\sigma}\cdot \mathbf{p} \\
     \mathbf{\sigma}\cdot \mathbf{p}& m -E  \\
\end{bmatrix}^{T}
\begin{bmatrix}
    \left( E + m \right) \chi _s  \\
    \mathbf{\sigma}\cdot \mathbf{p }\chi _s \\
\end{bmatrix} \\
&=  A^2
\begin{bmatrix}
    \chi _s^\dagger&0 \\
\end{bmatrix}
\begin{bmatrix}
    E+m & \mathbf{\sigma}\cdot \mathbf{p} \\
     -\mathbf{\sigma}\cdot \mathbf{p}& m -E  \\
\end{bmatrix}
\begin{bmatrix}
    \left( E + m \right) \chi _s  \\
    \mathbf{\sigma}\cdot \mathbf{p }\chi _s \\
\end{bmatrix} \\
&=
\begin{bmatrix}
    \chi _s^\dagger&0 \\
\end{bmatrix}
\begin{bmatrix}
    \left( E + m \right) ^2 \chi _s + \left( \mathbf{\sigma} \cdot \mathbf{p} \right) ^2 \chi _s \\
    -\left( \mathbf{\sigma} \cdot \mathbf{p} \right) \left( E + m \right) \chi _s + \left( \mathbf{\sigma} \cdot \mathbf{p} \right)\left( m - E \right)  \chi _s \\
\end{bmatrix} \\
    =& A^2 \left( \left( E + m \right) ^2 + \left( \mathbf{\sigma}\cdot \mathbf{p} \right) ^2 \right) = A^2\left( E^2 + m^2 + 2mE + p^2 \right) = A^2\left( 2E\left( E+m \right)  \right) = 2m \\ 
% &= \begin{bmatrix} (\not{p} + m)^\dagger \begin{bmatrix} \chi_s \\ 0 \end{bmatrix} \end{bmatrix}^\dagger \begin{bmatrix} (E + m) \chi_s \\ \boldsymbol{\sigma} \cdot \mathbf{p} \chi_s \end{bmatrix}, \\
% &= A^2 \begin{bmatrix} \chi_s^\dagger & 0 \end{bmatrix} \begin{bmatrix} (E + m) & -\boldsymbol{\sigma} \cdot \mathbf{p} \\ \boldsymbol{\sigma} \cdot \mathbf{p} & (E - m) \end{bmatrix} \begin{bmatrix} (E + m) \chi_s \\ \boldsymbol{\sigma} \cdot \mathbf{p} \chi_s \end{bmatrix}, \\
% &= A^2 \begin{bmatrix} \chi_s^\dagger & 0 \end{bmatrix} \begin{bmatrix} (E + m) \chi_s \\ \boldsymbol{\sigma} \cdot \mathbf{p} \chi_s \end{bmatrix}, \\
% &= A^2 \left[ \chi_s^\dagger \quad 0 \right] \begin{bmatrix} (E + m) & \boldsymbol{\sigma} \cdot \mathbf{p} \\ \boldsymbol{\sigma} \cdot \mathbf{p} & m - E \end{bmatrix} \begin{bmatrix} (E + m) \chi_s \\ \boldsymbol{\sigma} \cdot \mathbf{p} \chi_s \end{bmatrix}, \\
% &= A^2 \begin{bmatrix} \chi_s^\dagger & 0 \end{bmatrix} \begin{bmatrix} (E + m) \chi_s \\ \boldsymbol{\sigma} \cdot \mathbf{p} \chi_s \end{bmatrix}, \\
% &= A^2 \left[ \chi_s^\dagger \quad 0 \right] \begin{bmatrix} (E + m)^2 \chi_s + (\boldsymbol{\sigma} \cdot \mathbf{p} \chi_s) \\ -(\boldsymbol{\sigma} \cdot \mathbf{p}) (E + m) \chi_s + (M - E) \boldsymbol{\sigma} \cdot \mathbf{p} \chi_s \end{bmatrix}, \\
% &= A^2 \left[ (E + m)^2 + (\boldsymbol{\sigma} \cdot \mathbf{p}^2) \right], \\
% &= A^2 (E^2 + m^2 + 2mE + \mathbf{p}^2) = A^2 (2E (E + M)) = 2m, \\
\Rightarrow A
&= \sqrt{\frac{2m}{2E (E + m)}} = \sqrt{\frac{m}{E (E + m)}}
\end{align*}
\end{solution}

\begin{exercise}[2]
Find the solutions in the massless limit, \( m = 0 \). How does the Dirac equation simplify in this limit?
\end{exercise}

\begin{exercise}[3]
The Dirac Hamiltonian operator has the form \( H_D = -i \gamma^0 \boldsymbol{\gamma} \cdot \nabla + \gamma^0 m \). Introduce the helicity operator, \( h = \boldsymbol{\sigma} \cdot \mathbf{p} / |\mathbf{p}| \). Show that the Dirac operator and \( h \) commute. Note that the while \( h \) looks like a \( 2 \times 2 \) matrix due to the \( \boldsymbol{\sigma} \) part, when we apply it to 4-spinors it is understood that it is a \( 4 \times 4 \) matrix also (it has an implicit \( 2 \times 2 \) matrix multiplied on it). Argue that the spinors, \( \chi_s \), defined above are in fact the helicity eigenfunctions.
\end{exercise}

\begin{exercise}[4]
Look the solutions in the massless limit from 2). Determine the helicity of the four solutions when \( m = 0 \). How is the spin and helicity connected for positive and negative energy solutions?
\end{exercise}

\end{document}
