\documentclass[working, oneside]{../../../Preambles/tuftebook}
% Import xcolor and define some colors
\usepackage{{xcolor}}
\definecolor{{background}}{{HTML}}{{{background}}}
\definecolor{{foreground}}{{HTML}}{{{foreground}}}
\definecolor{{math}}{{HTML}}{{{color6}}}

%%%%%%%%%%%%%%%%%%%%%%%%%%%%%%%%%%%%%%%% IMPORTS %%%%%%%%%%%%%%%%%%%%%%%%%%%%%%%%%%%%%%%%
\documentclass[11pt,onesize,a4paper,titlepage]{article}

%%%%%%%%%%%%%%% Formatting %%%%%%%%%%%%%%% 
\usepackage[english]{babel}
\usepackage[utf8]{inputenc}
\usepackage{adjustbox}
\usepackage{geometry} % Margins
\usepackage{sectsty} % Custom Sections

%%%%%%%%%%%%%%% Font %%%%%%%%%%%%%%% 
\usepackage{Archivo}
\usepackage[T1]{fontenc}
\sffamily

%%%%%%%%%%%%%%% Graphics %%%%%%%%%%%%%%% 
\usepackage{fontawesome5} % Icons
\usepackage{graphicx} % Images
\usepackage[most]{tcolorbox} % Color Box
\usepackage{xcolor} % Colors
\usepackage{tikz} % For Drawing Shapes
%%%\usepackage{emoji} % For flags
\tcbuselibrary{breakable}
%%%\usepackage{academicons}

%%%%%%%%%%%%%%% Miscelanous %%%%%%%%%%%%%%% 
\usepackage{lipsum} % Lorem Ipsum
\usepackage{hyperref} % For Hyperlinks

%%%%%%%%%%%%%%% Colors %%%%%%%%%%%%%%% 
\definecolor{title}{HTML}{b5bff5} % Color of the title
\definecolor{bars}{HTML}{889af0} % Color of the title
\definecolor{backdrop}{HTML}{f2f2f2} % Color of the side column
\definecolor{lightgray}{HTML}{dfdfdf} % Color for the skill bars

%%% TU green: #639a00
%%% TU gray: #e6e6e6
%\definecolor{title}{HTML}{639a00} % Color of the title TU
%\definecolor{bars}{HTML}{889af0} % Color of the title TU

% \definecolor{backdrop}{HTML}{f2f2f2} % Color of the side column
\definecolor{backdrop}{HTML}{e6e6e6} % Color of the side column

\definecolor{subtitle}{HTML}{606060} % 


%%%%%%%%%%%%%%% Section Format %%%%%%%%%%%%%%% 
\sectionfont{                     
    \LARGE % Font size
    \sectionrule{0pt}{0pt}{-8pt}{1pt} % Rule under Section name
}

\subsectionfont{
    \Large % Font size
    \fontfamily{phv}\selectfont % Font family
    %\sectionrule{0pt}{0pt}{-8pt}{1pt} % Rule under Subsection name
    \sectionrule{5pt}{0pt}{0pt}{0pt} % Rule under Subsection name
}

%%%%%%%%%%%%%%% Margins and Headers %%%%%%%%%%%%%%%
\geometry{
  a4paper,
  left=7mm,
  right=7mm,
  bottom=10mm,
  top=10mm
}

\pagestyle{empty} % Empty Headers

\usepackage{marvosym}

% \renewcommand\qedsymbol{\CoffeeCup}

\usepackage{changepage}

\newenvironment{subexercise}[1]{%
    \begin{mdframed}[linewidth=0.5pt, linecolor=foreground, backgroundcolor=background, leftmargin=0cm, innerleftmargin=1em, innertopmargin=0pt, innerbottommargin=0pt, innerrightmargin=0pt, topline=false, rightline=false, bottomline=false]
    \par\noindent\textcolor{foreground}{\textbf{#1.}}\hspace{1em}\ignorespaces
}{%
    \par\addvspace{\baselineskip}\end{mdframed}\ignorespacesafterend
}
\newenvironment{solution}{%
    % \par\addvspace{\baselineskip}\noindent\makebox[\textwidth]{\textcolor{foreground}{\textbullet\hspace{1em}\textbullet\hspace{1em}\textbullet}}\par\addvspace{\baselineskip}
    \begin{mdframed}[linewidth=0.5pt, linecolor=foreground, backgroundcolor=background, rightmargin=0cm, innerleftmargin=0cm, innertopmargin=0pt, innerbottommargin=0pt, innerrightmargin=1em, topline=false, leftline=false, bottomline=false]
    \par\noindent\textcolor{foreground}{\textit{Solution.}}\hspace{1em}\ignorespaces
}{%
    \par\addvspace{\baselineskip}\noindent\hfill\textcolor{foreground}{\Coffeecup}\par\addvspace{\baselineskip}\end{mdframed}\ignorespacesafterend
}
% Exercise environment

\declaretheoremstyle[
    name= \textcolor{foreground}{Exercise},
    postheadspace = \newline,
    bodyfont = \normalfont\color{foreground},
    postheadhook={\textcolor{math}{\rule[.4ex]{\linewidth}{0.5pt}}\\},
    % numberwithin=chapter,
    mdframed={
        backgroundcolor = background,
        linecolor = foreground,
        linewidth = 0.5pt,
        rightline =  true,
        topline = true,
        bottomline = true,
        skipabove=20pt,
        skipbelow=20pt,
        innerleftmargin=15pt,
        innertopmargin=10pt,
        innerrightmargin=15pt,
        innerbottommargin=10pt}
    ]{exercise}
\declaretheorem[style=exercise,numbered=no]{exercise}

% \etocsetlevel{exercise}{2}

% \AtEndEnvironment{exercise}{%
%   \etoctoccontentsline{exercise}{\protect\numberline{\theexercise}}%
% }%
% \etocsetstyle{exercise}
% {}
% {}
% % this will be rendered like a non-numbered section, but we could have used
% % \numberline here also
% {\etocsavedsectiontocline{Exercise \etocnumber}{\etocpage}}
%     {}

% theorem environment

\declaretheoremstyle[
    name= \textcolor{foreground}{Theorem},
    postheadspace = \newline,
    bodyfont = \normalfont\color{foreground},
    postheadhook={\textcolor{math}{\rule[.4ex]{\linewidth}{1pt}}\\},
    mdframed={
        backgroundcolor = background,
        linecolor = foreground,
        linewidth = 1pt,
        rightline =  true,
        topline = true,
        bottomline = true,
        skipabove=20pt,
        skipbelow=20pt,
        innerleftmargin=15pt,
        innertopmargin=10pt,
        innerrightmargin=15pt,
        innerbottommargin=10pt}
    ]{theorem}
\declaretheorem[style=theorem,numbered=yes]{theorem}

\declaretheoremstyle[
    name= \textcolor{foreground}{Definition},
    postheadspace = \newline,
    bodyfont = \normalfont\color{foreground},
    postheadhook={\textcolor{math}{\rule[.4ex]{\linewidth}{1pt}}\\},
    mdframed={
        backgroundcolor = background,
        linecolor = foreground,
        linewidth = 1pt,
        rightline =  true,
        topline = true,
        bottomline = true,
        skipabove=20pt,
        skipbelow=20pt,
        innerleftmargin=15pt,
        innertopmargin=10pt,
        innerrightmargin=15pt,
        innerbottommargin=10pt}
    ]{definition}
\declaretheorem[style=definition,numbered=yes]{definition}
% Example environment

\declaretheoremstyle[
name= \quad \underline{Proof:},
     headfont = \bfseries\sffamily,
     postheadspace = \newline,
     % notebraces = \bfseries{(}{)a},
     headpunct = {},
     bodyfont = ,
     postheadhook={\textcolor{foreground}{\rule[0.4ex]{\linewidth}{0pt}}\\},
     qed=\qedsymbol,
    % spacebelow = 10pt,
    mdframed={
  backgroundcolor = background,
  linecolor = foreground,
  linewidth = 1pt,
  skipabove=10pt,
  skipbelow=10pt,
  rightline = false,
  topline = false,
  leftline = false,
  bottomline = false,
  innerleftmargin=15pt,
  innertopmargin=15pt,
  innerrightmargin=15pt,
  innerbottommargin=15pt}
]{pro}
    % \declaretheorem[style=pro,numbered=no]{Proof}

\declaretheoremstyle[
name= \quad \underline{\textcolor{foreground}{Example}},
     headfont = \bfseries\sffamily,
     postheadspace = \newline,
     % notebraces = \bfseries{(}{)a},
     headpunct = {},
     bodyfont = \normalfont\color{foreground},
     postheadhook={\textcolor{foreground}{\rule[0.4ex]{\linewidth}{0pt}}\\},
     % spacebelow = 10pt,
    mdframed={
  backgroundcolor = background,
  linecolor = foreground,
  linewidth = 1pt,
  skipabove=10pt,
  skipbelow=10pt,
  rightline = false,
  topline = false,
  leftline = false,
  bottomline = false,
  innerleftmargin=15pt,
  innertopmargin=15pt,
  innerrightmargin=15pt,
  innerbottommargin=15pt}
]{ex}
\declaretheorem[style=ex,numbered=no]{example}

\declaretheoremstyle[
     name=,
     headfont = \bfseries\sffamily,
     notebraces = \bfseries{},
     headpunct = { -},
     bodyfont = \color{foreground}\normalfont,
     % postheadhook={\textcolor{black}{\rule[.4ex]{\linewidth}{0.2pt}}\\},
    % spacebelow = 10pt,
    mdframed={
  backgroundcolor = background,
  linecolor = foreground,
  linewidth = 1pt,
  skipabove=0pt,
  skipbelow=0pt,
  innerleftmargin=10pt,
  innertopmargin=10pt,
  innerrightmargin=10pt,
  innerbottommargin=10pt,
  rightline = false,
  topline = false,
  leftline = false,
  bottomline = true}
]{subexercise}
% \declaretheorem[style=subexercise,numbered=no]{subexercise}

\declaretheoremstyle[
     name= \color{losning}Løsning,
     headfont = \bfseries\sffamily,
     notebraces = \bfseries{},
     postheadspace = \newline,
     headpunct = {:},
     bodyfont = \normalfont,
     % qed = ,
     % postheadhook={\textcolor{black}{\rule[.4ex]{\linewidth}{0.2pt}}\\},
    % spacebelow = 10pt,
    mdframed={
  backgroundcolor = background,
  linecolor = losning!75,
  linewidth = 1pt,
  skipabove=0pt,
  skipbelow=10pt,
  innerleftmargin=10pt,
  innertopmargin=10pt,
  innerrightmargin=10pt,
  innerbottommargin=10pt,
  leftline = false,
  rightline = true,
  topline = false,
  bottomline = true}
]{solution}

\newenvironment{SimpleBox}[1]{%
  \begin{mdframed}%
    \noindent\textbf{#1}\\[1ex]
}{%
  \end{mdframed}%
}


\begin{document}
\let\cleardoublepage\clearpage
\thispagestyle{fancy}
\chapter{1 The Dirac Equation and Dimensionality}
\begin{exercise}[1]
Dirac started by postulating an equation linear in time and space of the form
\begin{align}
(\boldsymbol{\alpha} \cdot \mathbf{p} + m\beta)\psi = i \frac{\partial \psi}{\partial t}, \tag{1}
\end{align}
and then proceeded to figure out what the conditions on $\boldsymbol{\alpha}$ and $\beta$ would have to be to ensure that it obeys the energy-momentum relation of special relativity, i.e. $E^2 = |\mathbf{p}|^2 + m^2$.
\end{exercise}

\begin{subexercise}{1}
Show that the conditions are
\begin{align}
\alpha_i \beta + \beta \alpha_i &= 0, \quad i = 1, 2, 3  \\
\alpha_i \alpha_j + \alpha_j \alpha_i &= 0, \quad i, j = 1, 2, 3; i \neq j  \\
\alpha_i^2 = \beta^2 &= 1, \quad i = 1, 2, 3. 
\end{align}
(Hint: Take the square of the operator on both sides of Eq. (1).)
\end{subexercise}
\begin{solution}
We use the time-dependent Schrodinger equation,
\[
\left( \alpha\cdot \mathbf{p}+m\beta \right) \Psi =i \frac{\partial \Psi}{\partial t} = \hat{H}\Psi = E\Psi
.\] 
This implies that,
\[
\left( \alpha\cdot \mathbf{p }+ m\beta \right) ^2 = E^2
.\] 
Now since we want $E^2 = \|\mathbf{p}\|^2 + m^2$ this leads to some restrictions on $\alpha_i$ and $\beta$. Lets by expanding the square,
\[
\left( \alpha\cdot \mathbf{p} + m\beta \right)^2 = \left( \alpha\cdot \mathbf{p} \right) ^2 + \left( m\beta \right) ^2 + \alpha\cdot \mathbf{p}m\beta +\beta m \alpha \cdot \mathbf{p}
.\] 
It seems natural that these cross-terms should dissapear. We can rewrite them as a sum,
\[
\alpha\cdot \mathbf{p}m\beta +\beta m \alpha \cdot \mathbf{p} = \sum_i mp_i \left( \alpha_i \beta + \beta \alpha_i \right) = 0
.\] 
Assuming condition (1) it is indeed zero. Lets rewrite the first term,
\begin{align*}
    \left( \alpha \cdot \mathbf{p} \right) ^2 &= \sum_{j=1}\sum_{i=1} p_ip_j\alpha_i\alpha_j = \frac{1}{2} \left(\sum_{j=1}\sum_{i=1} p_ip_j\alpha_i\alpha_j + p_ip_j\alpha_i\alpha_j\right) \\
    &=  \frac{1}{2} \left( \sum_{j=1}\sum_{i=1} p_ip_j\alpha_i\alpha_j + \sum_{i=1}\sum_{j=1}p_ip_j\alpha_i\alpha_j\right) = \frac{1}{2} \left( \sum_{j=1}\sum_{i=1} p_ip_j\alpha_i\alpha_j + \sum_{i=1}\sum_{j=1}p_jp_i\alpha_i\alpha_j\right) \\
    &=  \frac{1}{2} \left( \sum_{j=1}\sum_{i=1} p_ip_j\alpha_i\alpha_j + \sum_{j=1}\sum_{i=1}p_ip_j\alpha_i\alpha_j\right) = \frac{1}{2} \left( \sum_{j=1}\sum_{i=1} p_ip_j\left(   \alpha_i\alpha_j + \alpha_i\alpha_j\right)\right) \\
    &=  \frac{1}{2} \left( \sum_{i=1} 2p_i^2\alpha_i^2 + \sum_{j=1, j\neq i}\sum_{i=1} p_ip_j\left(   \alpha_i\alpha_j + \alpha_i\alpha_j\right)\right) = \sum_ip_i^2 = \mathbf{p}^2
    % \left( \sum_i \alpha_i p_i \right) ^2 = \sum_i \left( \alpha_i p_i  \right) ^2 + \sum_{j \neq i, j = 1}\sum_{i = 1} p_ip_j \alpha_i\alpha_j \\
    % &=   \sum_i \left( \alpha_i p_i  \right) ^2  + \frac{1}{2}\left(   \sum_{j \neq i, j = 1}\sum_{i = 1} p_ip_j \alpha_i\alpha_j + p_ip_j \alpha_ialpha_j\right)\\
.\end{align*}
\end{solution}
\begin{subexercise}{2}
Prove that $\alpha_i, i = 1, 2, 3$ and $\beta$ are all Hermitian. (Hint: First find the Hamiltonian for the Dirac particles).
\end{subexercise}
\begin{solution}
The hamiltonian for our system is,
\[
\hat{H} = \left( \alpha\cdot  \mathbf{p} + m\beta \right) 
.\] 
We know that hamiltonians are hermitian, so
\[
\hat{H}=\hat{H}^\dagger = \left( \alpha\cdot  \mathbf{p} + m\beta \right)^\dagger = \left( \alpha^\dagger\cdot  \mathbf{p} + m\beta^\dagger \right)
.\] 
Which implies $\alpha = \alpha^\dagger$ and $\beta=\beta^\dagger$.
\end{solution}
\begin{subexercise}{3}
Prove that $\text{Tr}(\alpha_i) = \text{Tr}(\beta) = 0$ where Tr is the matrix trace (sum of diagonal entries).
\end{subexercise}
\begin{solution}
We can use condition (2),
\[
\alpha_i \beta + \beta \alpha_i &= 0, \quad i = 1, 2, 3  
.\] 
This implies,
\[
\alpha_i = -\beta^{-1}\alpha_i\beta
.\] 
We also use the fact that $\text{Tr}\left( AB \right) = \text{Tr}\left( BA \right) $.
\[
\text{Tr}\left(\alpha_i\right) = -\text{Tr}\left(\beta^{-1}(\alpha_i\beta)\right) = -\text{Tr}\left(\beta\beta^{-1}\alpha_i\right) = -\text{Tr}\left(\alpha_i\right)
.\] 
Which implies that the trace is zero.
\end{solution}
\begin{subexercise}{4}
Prove that the eigenvalues of $\alpha_i$ and $\beta$ are all either +1 or -1.
\end{subexercise}

\begin{subexercise}{5}
Prove that the dimensionality of $\alpha_i$ and $\beta$ is even.
\end{subexercise}

\begin{subexercise}{6}
Argue that this implies that the dimension of $\alpha_i$ and $\beta$ must be at least 4.
\end{subexercise}
\end{document}
