\documentclass[working, oneside]{../../../Preambles/tuftebook}
% Import xcolor and define some colors
\usepackage{{xcolor}}
\definecolor{{background}}{{HTML}}{{{background}}}
\definecolor{{foreground}}{{HTML}}{{{foreground}}}
\definecolor{{math}}{{HTML}}{{{color6}}}

%%%%%%%%%%%%%%%%%%%%%%%%%%%%%%%%%%%%%%%% IMPORTS %%%%%%%%%%%%%%%%%%%%%%%%%%%%%%%%%%%%%%%%
\documentclass[11pt,onesize,a4paper,titlepage]{article}

%%%%%%%%%%%%%%% Formatting %%%%%%%%%%%%%%% 
\usepackage[english]{babel}
\usepackage[utf8]{inputenc}
\usepackage{adjustbox}
\usepackage{geometry} % Margins
\usepackage{sectsty} % Custom Sections

%%%%%%%%%%%%%%% Font %%%%%%%%%%%%%%% 
\usepackage{Archivo}
\usepackage[T1]{fontenc}
\sffamily

%%%%%%%%%%%%%%% Graphics %%%%%%%%%%%%%%% 
\usepackage{fontawesome5} % Icons
\usepackage{graphicx} % Images
\usepackage[most]{tcolorbox} % Color Box
\usepackage{xcolor} % Colors
\usepackage{tikz} % For Drawing Shapes
%%%\usepackage{emoji} % For flags
\tcbuselibrary{breakable}
%%%\usepackage{academicons}

%%%%%%%%%%%%%%% Miscelanous %%%%%%%%%%%%%%% 
\usepackage{lipsum} % Lorem Ipsum
\usepackage{hyperref} % For Hyperlinks

%%%%%%%%%%%%%%% Colors %%%%%%%%%%%%%%% 
\definecolor{title}{HTML}{b5bff5} % Color of the title
\definecolor{bars}{HTML}{889af0} % Color of the title
\definecolor{backdrop}{HTML}{f2f2f2} % Color of the side column
\definecolor{lightgray}{HTML}{dfdfdf} % Color for the skill bars

%%% TU green: #639a00
%%% TU gray: #e6e6e6
%\definecolor{title}{HTML}{639a00} % Color of the title TU
%\definecolor{bars}{HTML}{889af0} % Color of the title TU

% \definecolor{backdrop}{HTML}{f2f2f2} % Color of the side column
\definecolor{backdrop}{HTML}{e6e6e6} % Color of the side column

\definecolor{subtitle}{HTML}{606060} % 


%%%%%%%%%%%%%%% Section Format %%%%%%%%%%%%%%% 
\sectionfont{                     
    \LARGE % Font size
    \sectionrule{0pt}{0pt}{-8pt}{1pt} % Rule under Section name
}

\subsectionfont{
    \Large % Font size
    \fontfamily{phv}\selectfont % Font family
    %\sectionrule{0pt}{0pt}{-8pt}{1pt} % Rule under Subsection name
    \sectionrule{5pt}{0pt}{0pt}{0pt} % Rule under Subsection name
}

%%%%%%%%%%%%%%% Margins and Headers %%%%%%%%%%%%%%%
\geometry{
  a4paper,
  left=7mm,
  right=7mm,
  bottom=10mm,
  top=10mm
}

\pagestyle{empty} % Empty Headers

\usepackage{marvosym}

% \renewcommand\qedsymbol{\CoffeeCup}

\usepackage{changepage}

\newenvironment{subexercise}[1]{%
    \begin{mdframed}[linewidth=0.5pt, linecolor=foreground, backgroundcolor=background, leftmargin=0cm, innerleftmargin=1em, innertopmargin=0pt, innerbottommargin=0pt, innerrightmargin=0pt, topline=false, rightline=false, bottomline=false]
    \par\noindent\textcolor{foreground}{\textbf{#1.}}\hspace{1em}\ignorespaces
}{%
    \par\addvspace{\baselineskip}\end{mdframed}\ignorespacesafterend
}
\newenvironment{solution}{%
    % \par\addvspace{\baselineskip}\noindent\makebox[\textwidth]{\textcolor{foreground}{\textbullet\hspace{1em}\textbullet\hspace{1em}\textbullet}}\par\addvspace{\baselineskip}
    \begin{mdframed}[linewidth=0.5pt, linecolor=foreground, backgroundcolor=background, rightmargin=0cm, innerleftmargin=0cm, innertopmargin=0pt, innerbottommargin=0pt, innerrightmargin=1em, topline=false, leftline=false, bottomline=false]
    \par\noindent\textcolor{foreground}{\textit{Solution.}}\hspace{1em}\ignorespaces
}{%
    \par\addvspace{\baselineskip}\noindent\hfill\textcolor{foreground}{\Coffeecup}\par\addvspace{\baselineskip}\end{mdframed}\ignorespacesafterend
}
% Exercise environment

\declaretheoremstyle[
    name= \textcolor{foreground}{Exercise},
    postheadspace = \newline,
    bodyfont = \normalfont\color{foreground},
    postheadhook={\textcolor{math}{\rule[.4ex]{\linewidth}{0.5pt}}\\},
    % numberwithin=chapter,
    mdframed={
        backgroundcolor = background,
        linecolor = foreground,
        linewidth = 0.5pt,
        rightline =  true,
        topline = true,
        bottomline = true,
        skipabove=20pt,
        skipbelow=20pt,
        innerleftmargin=15pt,
        innertopmargin=10pt,
        innerrightmargin=15pt,
        innerbottommargin=10pt}
    ]{exercise}
\declaretheorem[style=exercise,numbered=no]{exercise}

% \etocsetlevel{exercise}{2}

% \AtEndEnvironment{exercise}{%
%   \etoctoccontentsline{exercise}{\protect\numberline{\theexercise}}%
% }%
% \etocsetstyle{exercise}
% {}
% {}
% % this will be rendered like a non-numbered section, but we could have used
% % \numberline here also
% {\etocsavedsectiontocline{Exercise \etocnumber}{\etocpage}}
%     {}

% theorem environment

\declaretheoremstyle[
    name= \textcolor{foreground}{Theorem},
    postheadspace = \newline,
    bodyfont = \normalfont\color{foreground},
    postheadhook={\textcolor{math}{\rule[.4ex]{\linewidth}{1pt}}\\},
    mdframed={
        backgroundcolor = background,
        linecolor = foreground,
        linewidth = 1pt,
        rightline =  true,
        topline = true,
        bottomline = true,
        skipabove=20pt,
        skipbelow=20pt,
        innerleftmargin=15pt,
        innertopmargin=10pt,
        innerrightmargin=15pt,
        innerbottommargin=10pt}
    ]{theorem}
\declaretheorem[style=theorem,numbered=yes]{theorem}

\declaretheoremstyle[
    name= \textcolor{foreground}{Definition},
    postheadspace = \newline,
    bodyfont = \normalfont\color{foreground},
    postheadhook={\textcolor{math}{\rule[.4ex]{\linewidth}{1pt}}\\},
    mdframed={
        backgroundcolor = background,
        linecolor = foreground,
        linewidth = 1pt,
        rightline =  true,
        topline = true,
        bottomline = true,
        skipabove=20pt,
        skipbelow=20pt,
        innerleftmargin=15pt,
        innertopmargin=10pt,
        innerrightmargin=15pt,
        innerbottommargin=10pt}
    ]{definition}
\declaretheorem[style=definition,numbered=yes]{definition}
% Example environment

\declaretheoremstyle[
name= \quad \underline{Proof:},
     headfont = \bfseries\sffamily,
     postheadspace = \newline,
     % notebraces = \bfseries{(}{)a},
     headpunct = {},
     bodyfont = ,
     postheadhook={\textcolor{foreground}{\rule[0.4ex]{\linewidth}{0pt}}\\},
     qed=\qedsymbol,
    % spacebelow = 10pt,
    mdframed={
  backgroundcolor = background,
  linecolor = foreground,
  linewidth = 1pt,
  skipabove=10pt,
  skipbelow=10pt,
  rightline = false,
  topline = false,
  leftline = false,
  bottomline = false,
  innerleftmargin=15pt,
  innertopmargin=15pt,
  innerrightmargin=15pt,
  innerbottommargin=15pt}
]{pro}
    % \declaretheorem[style=pro,numbered=no]{Proof}

\declaretheoremstyle[
name= \quad \underline{\textcolor{foreground}{Example}},
     headfont = \bfseries\sffamily,
     postheadspace = \newline,
     % notebraces = \bfseries{(}{)a},
     headpunct = {},
     bodyfont = \normalfont\color{foreground},
     postheadhook={\textcolor{foreground}{\rule[0.4ex]{\linewidth}{0pt}}\\},
     % spacebelow = 10pt,
    mdframed={
  backgroundcolor = background,
  linecolor = foreground,
  linewidth = 1pt,
  skipabove=10pt,
  skipbelow=10pt,
  rightline = false,
  topline = false,
  leftline = false,
  bottomline = false,
  innerleftmargin=15pt,
  innertopmargin=15pt,
  innerrightmargin=15pt,
  innerbottommargin=15pt}
]{ex}
\declaretheorem[style=ex,numbered=no]{example}

\declaretheoremstyle[
     name=,
     headfont = \bfseries\sffamily,
     notebraces = \bfseries{},
     headpunct = { -},
     bodyfont = \color{foreground}\normalfont,
     % postheadhook={\textcolor{black}{\rule[.4ex]{\linewidth}{0.2pt}}\\},
    % spacebelow = 10pt,
    mdframed={
  backgroundcolor = background,
  linecolor = foreground,
  linewidth = 1pt,
  skipabove=0pt,
  skipbelow=0pt,
  innerleftmargin=10pt,
  innertopmargin=10pt,
  innerrightmargin=10pt,
  innerbottommargin=10pt,
  rightline = false,
  topline = false,
  leftline = false,
  bottomline = true}
]{subexercise}
% \declaretheorem[style=subexercise,numbered=no]{subexercise}

\declaretheoremstyle[
     name= \color{losning}Løsning,
     headfont = \bfseries\sffamily,
     notebraces = \bfseries{},
     postheadspace = \newline,
     headpunct = {:},
     bodyfont = \normalfont,
     % qed = ,
     % postheadhook={\textcolor{black}{\rule[.4ex]{\linewidth}{0.2pt}}\\},
    % spacebelow = 10pt,
    mdframed={
  backgroundcolor = background,
  linecolor = losning!75,
  linewidth = 1pt,
  skipabove=0pt,
  skipbelow=10pt,
  innerleftmargin=10pt,
  innertopmargin=10pt,
  innerrightmargin=10pt,
  innerbottommargin=10pt,
  leftline = false,
  rightline = true,
  topline = false,
  bottomline = true}
]{solution}

\newenvironment{SimpleBox}[1]{%
  \begin{mdframed}%
    \noindent\textbf{#1}\\[1ex]
}{%
  \end{mdframed}%
}


\begin{document}
\let\cleardoublepage\clearpage
\thispagestyle{fancy}
\chapter{4 - The Fermion Propagator}
The following problem derives the propagator for a fermion. First we have to figure out the Hamiltonian. We start from
\begin{align*}
\mathcal{L}
&= \overline{\psi} (i \gamma_\mu \partial^\mu - m) \psi, \tag{24}
\end{align*}
where we use the convenient short-hand \( \partial^\mu = \frac{\partial}{\partial x_\mu} \).

\begin{exercise}[1]
Show that the corresponding Hamiltonian can be written
\begin{align*}
H
&= \int d^3x \overline{\psi} (-i \boldsymbol{\gamma} \cdot \nabla + m) \psi, \tag{25}
\end{align*}
where \( \boldsymbol{\gamma} \) is the 3-vector \( (\gamma^1, \gamma^2, \gamma^3) \). Remember that \( \partial^\mu = (\partial_t, -\nabla) \).
\end{exercise}

\begin{exercise}[2]
Now consider Heisenberg's equations of motion for the field operator \( \psi(\mathbf{x}, t) \), i.e.
\begin{align*}
\frac{\partial \psi_\alpha(\mathbf{x}, t)}{\partial t}
&= -i [\psi_\alpha(\mathbf{x}, t), H], \tag{26}
\end{align*}
where \( \alpha \) is an index indicating that keep track of the fact that there are four components in the field of Dirac particle. Show that
\begin{align*}
\frac{\partial \psi_\alpha(\mathbf{x}, t)}{\partial t}
&= \left( - \gamma^0 \boldsymbol{\gamma} \cdot \nabla - i m \gamma_0 \right)_{\alpha \beta} \psi_\beta(\mathbf{x}, t), \tag{27}
\end{align*}
where we sum over the spinor index \( \beta \) (repeated use of spinor indices mean summation from now on). You will need the canonical anti-commutator for fermion fields is
\begin{align*}
\left\{ \psi_\alpha(\mathbf{x}_1, t), \psi_\beta^\dagger(\mathbf{x}_2, t) \right\}
&= \delta_{\alpha \beta} \delta(\mathbf{x}_1 - \mathbf{x}_2). \tag{28}
\end{align*}
\end{exercise}
\begin{exercise}[3]
We now define the time-ordering operator for fermionic fields, \( \psi \) and \( \overline{\psi} \) in the following way
\begin{align*}
T \left\{ \psi_\alpha(\mathbf{x}_1, t_1) \overline{\psi}_\beta(\mathbf{x}_2, t_2) \right\}
&= \begin{cases} \psi_\alpha(\mathbf{x}_1, t_1) \overline{\psi}_\beta(\mathbf{x}_2, t_2) & \text{for } t_1 > t_2 \\ - \overline{\psi}_\beta(\mathbf{x}_2, t_2) \psi_\alpha(\mathbf{x}_1, t_1) & \text{for } t_2 > t_1 \end{cases}, \tag{29}
\end{align*}
where this is now a matrix indexed by \( \alpha \) and \( \beta \) since the fields are four-component quantities. Show that
\begin{align*}
\frac{\partial}{\partial t_1} T \left\{ \psi_\alpha(\mathbf{x}_1, t_1) \overline{\psi}_\beta(\mathbf{x}_2, t_2) \right\}
&= \delta(t_1 - t_2) \delta(\mathbf{x}_1 - \mathbf{x}_2) [\gamma_0]_{\alpha \beta} + T \left\{ \frac{\partial \psi_\alpha(\mathbf{x}_1, t_1)}{\partial t_1} \overline{\psi}_\beta(\mathbf{x}_2, t_2) \right\} \tag{30}
\end{align*}
\end{exercise}

\begin{exercise}[4]
Now use Eq. (27) to show that
\begin{align*}
\left[ i \gamma_\mu \partial^\mu - m \right]_{\alpha \eta} T \left\{ \psi_\eta(\mathbf{x}_1, t_1) \overline{\psi}_\beta(\mathbf{x}_2, t_2) \right\}
&= i \delta(t_1 - t_2) \delta(\mathbf{x}_1 - \mathbf{x}_2) \delta_{\alpha \beta}, \tag{31}
\end{align*}
where \( \partial_\mu \) act on \( \mathbf{x}_1 \) and \( t_1 \) only.
\end{exercise}
\begin{solution}
We shall make use of each the results from the previous exercise in turn. Lets start by expanding the four-vector on the left-hand side
\begin{align*}
\left[ i \gamma_0\partial_0 - \mathbf{\gamma}\cdot \nabla - m \right]_{\alpha \eta} T \left\{ \psi_\eta(\mathbf{x}_1, t_1) \overline{\psi}_\beta(\mathbf{x}_2, t_2) \right\}
=
\left[ i \gamma_0 \frac{\partial}{\partial t_1}- \mathbf{\gamma}\cdot \nabla_{x_1} - m \right]_{\alpha \eta} T \left\{ \psi_\eta(\mathbf{x}_1, t_1) \overline{\psi}_\beta(\mathbf{x}_2, t_2) \right\}
\end{align*}
We can distribute these terms, noting that the $\alpha \eta$ index is equivalent to multiplying by a kronecker delta. Also note that since we are using einstein summation, things commute (not the fields themselves however). We can therefore apply the kronecker delta to the $\Psi_{\eta }$ term changing it to an $\alpha$. This yields,
\begin{align*}
    =& i \gamma_0 \frac{\partial}{\partial t_1}T \left\{ \psi_{\alpha}(\mathbf{x}_1, t_1) \overline{\psi}_\beta(\mathbf{x}_2, t_2) \right\} 
    - \mathbf{\gamma}\cdot \nabla _{x_1}T \left\{ \psi_{\alpha}(\mathbf{x}_1, t_1) \overline{\psi}_\beta(\mathbf{x}_2, t_2) \right\}
    -m T \left\{ \psi_{\alpha}(\mathbf{x}_1, t_1) \overline{\psi}_\beta(\mathbf{x}_2, t_2) \right\}
.\end{align*}
Let's focus on the first term, which is similar to what we calculated in the previous ecxercise,
\begin{align*}
     i \gamma_0 \frac{\partial}{\partial t_1}T \left\{ \psi_{\alpha}(\mathbf{x}_1, t_1) \overline{\psi}_\beta(\mathbf{x}_2, t_2) \right\} 
&= i \gamma_0 \left(    \delta(t_1 - t_2) \delta(\mathbf{x}_1 - \mathbf{x}_2) [\gamma_0]_{\alpha \beta} + T \left\{ \frac{\partial \psi_\alpha(\mathbf{x}_1, t_1)}{\partial t_1} \overline{\psi}_\beta(\mathbf{x}_2, t_2) \right\} \right)
.\end{align*}
We can substitute in the dirac-equation, and use the fact that $\gamma_0^2 = 1$,
\begin{align*}
&= i \delta(t_1 - t_2) \delta(\mathbf{x}_1 - \mathbf{x}_2)+ i\gamma_0T \left\{ \left( - \gamma^0 \boldsymbol{\gamma} \cdot \nabla - i m \gamma_0 \right)_{\alpha \eta } \psi_\eta (\mathbf{x_1}, t_1) \overline{\psi}_\beta(\mathbf{x}_2, t_2) \right\} \\
&= i \delta(t_1 - t_2) \delta(\mathbf{x}_1 - \mathbf{x}_2)+ i\gamma_0 \left( - \gamma^0 \boldsymbol{\gamma} \cdot \nabla - i m \gamma_0 \right)_{\alpha \eta }T\left\{  \psi_\eta (\mathbf{x_1}, t_1) \overline{\psi}_\beta(\mathbf{x}_2, t_2) \right\} \\
&= i \delta(t_1 - t_2) \delta(\mathbf{x}_1 - \mathbf{x}_2)+ i \left( - \boldsymbol{\gamma} \cdot \nabla - i m  \right)_{\alpha \eta }T\left\{  \psi_\eta (\mathbf{x_1}, t_1) \overline{\psi}_\beta(\mathbf{x}_2, t_2) \right\} \\
.\end{align*}
\end{solution}
\end{document}
