\documentclass[working, oneside]{../../Preambles/tuftebook}
% Import xcolor and define some colors
\usepackage{{xcolor}}
\definecolor{{background}}{{HTML}}{{{background}}}
\definecolor{{foreground}}{{HTML}}{{{foreground}}}
\definecolor{{math}}{{HTML}}{{{color6}}}

%%%%%%%%%%%%%%%%%%%%%%%%%%%%%%%%%%%%%%%% IMPORTS %%%%%%%%%%%%%%%%%%%%%%%%%%%%%%%%%%%%%%%%
\documentclass[11pt,onesize,a4paper,titlepage]{article}

%%%%%%%%%%%%%%% Formatting %%%%%%%%%%%%%%% 
\usepackage[english]{babel}
\usepackage[utf8]{inputenc}
\usepackage{adjustbox}
\usepackage{geometry} % Margins
\usepackage{sectsty} % Custom Sections

%%%%%%%%%%%%%%% Font %%%%%%%%%%%%%%% 
\usepackage{Archivo}
\usepackage[T1]{fontenc}
\sffamily

%%%%%%%%%%%%%%% Graphics %%%%%%%%%%%%%%% 
\usepackage{fontawesome5} % Icons
\usepackage{graphicx} % Images
\usepackage[most]{tcolorbox} % Color Box
\usepackage{xcolor} % Colors
\usepackage{tikz} % For Drawing Shapes
%%%\usepackage{emoji} % For flags
\tcbuselibrary{breakable}
%%%\usepackage{academicons}

%%%%%%%%%%%%%%% Miscelanous %%%%%%%%%%%%%%% 
\usepackage{lipsum} % Lorem Ipsum
\usepackage{hyperref} % For Hyperlinks

%%%%%%%%%%%%%%% Colors %%%%%%%%%%%%%%% 
\definecolor{title}{HTML}{b5bff5} % Color of the title
\definecolor{bars}{HTML}{889af0} % Color of the title
\definecolor{backdrop}{HTML}{f2f2f2} % Color of the side column
\definecolor{lightgray}{HTML}{dfdfdf} % Color for the skill bars

%%% TU green: #639a00
%%% TU gray: #e6e6e6
%\definecolor{title}{HTML}{639a00} % Color of the title TU
%\definecolor{bars}{HTML}{889af0} % Color of the title TU

% \definecolor{backdrop}{HTML}{f2f2f2} % Color of the side column
\definecolor{backdrop}{HTML}{e6e6e6} % Color of the side column

\definecolor{subtitle}{HTML}{606060} % 


%%%%%%%%%%%%%%% Section Format %%%%%%%%%%%%%%% 
\sectionfont{                     
    \LARGE % Font size
    \sectionrule{0pt}{0pt}{-8pt}{1pt} % Rule under Section name
}

\subsectionfont{
    \Large % Font size
    \fontfamily{phv}\selectfont % Font family
    %\sectionrule{0pt}{0pt}{-8pt}{1pt} % Rule under Subsection name
    \sectionrule{5pt}{0pt}{0pt}{0pt} % Rule under Subsection name
}

%%%%%%%%%%%%%%% Margins and Headers %%%%%%%%%%%%%%%
\geometry{
  a4paper,
  left=7mm,
  right=7mm,
  bottom=10mm,
  top=10mm
}

\pagestyle{empty} % Empty Headers

\usepackage{marvosym}

% \renewcommand\qedsymbol{\CoffeeCup}

\usepackage{changepage}

\newenvironment{subexercise}[1]{%
    \begin{mdframed}[linewidth=0.5pt, linecolor=foreground, backgroundcolor=background, leftmargin=0cm, innerleftmargin=1em, innertopmargin=0pt, innerbottommargin=0pt, innerrightmargin=0pt, topline=false, rightline=false, bottomline=false]
    \par\noindent\textcolor{foreground}{\textbf{#1.}}\hspace{1em}\ignorespaces
}{%
    \par\addvspace{\baselineskip}\end{mdframed}\ignorespacesafterend
}
\newenvironment{solution}{%
    % \par\addvspace{\baselineskip}\noindent\makebox[\textwidth]{\textcolor{foreground}{\textbullet\hspace{1em}\textbullet\hspace{1em}\textbullet}}\par\addvspace{\baselineskip}
    \begin{mdframed}[linewidth=0.5pt, linecolor=foreground, backgroundcolor=background, rightmargin=0cm, innerleftmargin=0cm, innertopmargin=0pt, innerbottommargin=0pt, innerrightmargin=1em, topline=false, leftline=false, bottomline=false]
    \par\noindent\textcolor{foreground}{\textit{Solution.}}\hspace{1em}\ignorespaces
}{%
    \par\addvspace{\baselineskip}\noindent\hfill\textcolor{foreground}{\Coffeecup}\par\addvspace{\baselineskip}\end{mdframed}\ignorespacesafterend
}
% Exercise environment

\declaretheoremstyle[
    name= \textcolor{foreground}{Exercise},
    postheadspace = \newline,
    bodyfont = \normalfont\color{foreground},
    postheadhook={\textcolor{math}{\rule[.4ex]{\linewidth}{0.5pt}}\\},
    % numberwithin=chapter,
    mdframed={
        backgroundcolor = background,
        linecolor = foreground,
        linewidth = 0.5pt,
        rightline =  true,
        topline = true,
        bottomline = true,
        skipabove=20pt,
        skipbelow=20pt,
        innerleftmargin=15pt,
        innertopmargin=10pt,
        innerrightmargin=15pt,
        innerbottommargin=10pt}
    ]{exercise}
\declaretheorem[style=exercise,numbered=no]{exercise}

% \etocsetlevel{exercise}{2}

% \AtEndEnvironment{exercise}{%
%   \etoctoccontentsline{exercise}{\protect\numberline{\theexercise}}%
% }%
% \etocsetstyle{exercise}
% {}
% {}
% % this will be rendered like a non-numbered section, but we could have used
% % \numberline here also
% {\etocsavedsectiontocline{Exercise \etocnumber}{\etocpage}}
%     {}

% theorem environment

\declaretheoremstyle[
    name= \textcolor{foreground}{Theorem},
    postheadspace = \newline,
    bodyfont = \normalfont\color{foreground},
    postheadhook={\textcolor{math}{\rule[.4ex]{\linewidth}{1pt}}\\},
    mdframed={
        backgroundcolor = background,
        linecolor = foreground,
        linewidth = 1pt,
        rightline =  true,
        topline = true,
        bottomline = true,
        skipabove=20pt,
        skipbelow=20pt,
        innerleftmargin=15pt,
        innertopmargin=10pt,
        innerrightmargin=15pt,
        innerbottommargin=10pt}
    ]{theorem}
\declaretheorem[style=theorem,numbered=yes]{theorem}

\declaretheoremstyle[
    name= \textcolor{foreground}{Definition},
    postheadspace = \newline,
    bodyfont = \normalfont\color{foreground},
    postheadhook={\textcolor{math}{\rule[.4ex]{\linewidth}{1pt}}\\},
    mdframed={
        backgroundcolor = background,
        linecolor = foreground,
        linewidth = 1pt,
        rightline =  true,
        topline = true,
        bottomline = true,
        skipabove=20pt,
        skipbelow=20pt,
        innerleftmargin=15pt,
        innertopmargin=10pt,
        innerrightmargin=15pt,
        innerbottommargin=10pt}
    ]{definition}
\declaretheorem[style=definition,numbered=yes]{definition}
% Example environment

\declaretheoremstyle[
name= \quad \underline{Proof:},
     headfont = \bfseries\sffamily,
     postheadspace = \newline,
     % notebraces = \bfseries{(}{)a},
     headpunct = {},
     bodyfont = ,
     postheadhook={\textcolor{foreground}{\rule[0.4ex]{\linewidth}{0pt}}\\},
     qed=\qedsymbol,
    % spacebelow = 10pt,
    mdframed={
  backgroundcolor = background,
  linecolor = foreground,
  linewidth = 1pt,
  skipabove=10pt,
  skipbelow=10pt,
  rightline = false,
  topline = false,
  leftline = false,
  bottomline = false,
  innerleftmargin=15pt,
  innertopmargin=15pt,
  innerrightmargin=15pt,
  innerbottommargin=15pt}
]{pro}
    % \declaretheorem[style=pro,numbered=no]{Proof}

\declaretheoremstyle[
name= \quad \underline{\textcolor{foreground}{Example}},
     headfont = \bfseries\sffamily,
     postheadspace = \newline,
     % notebraces = \bfseries{(}{)a},
     headpunct = {},
     bodyfont = \normalfont\color{foreground},
     postheadhook={\textcolor{foreground}{\rule[0.4ex]{\linewidth}{0pt}}\\},
     % spacebelow = 10pt,
    mdframed={
  backgroundcolor = background,
  linecolor = foreground,
  linewidth = 1pt,
  skipabove=10pt,
  skipbelow=10pt,
  rightline = false,
  topline = false,
  leftline = false,
  bottomline = false,
  innerleftmargin=15pt,
  innertopmargin=15pt,
  innerrightmargin=15pt,
  innerbottommargin=15pt}
]{ex}
\declaretheorem[style=ex,numbered=no]{example}

\declaretheoremstyle[
     name=,
     headfont = \bfseries\sffamily,
     notebraces = \bfseries{},
     headpunct = { -},
     bodyfont = \color{foreground}\normalfont,
     % postheadhook={\textcolor{black}{\rule[.4ex]{\linewidth}{0.2pt}}\\},
    % spacebelow = 10pt,
    mdframed={
  backgroundcolor = background,
  linecolor = foreground,
  linewidth = 1pt,
  skipabove=0pt,
  skipbelow=0pt,
  innerleftmargin=10pt,
  innertopmargin=10pt,
  innerrightmargin=10pt,
  innerbottommargin=10pt,
  rightline = false,
  topline = false,
  leftline = false,
  bottomline = true}
]{subexercise}
% \declaretheorem[style=subexercise,numbered=no]{subexercise}

\declaretheoremstyle[
     name= \color{losning}Løsning,
     headfont = \bfseries\sffamily,
     notebraces = \bfseries{},
     postheadspace = \newline,
     headpunct = {:},
     bodyfont = \normalfont,
     % qed = ,
     % postheadhook={\textcolor{black}{\rule[.4ex]{\linewidth}{0.2pt}}\\},
    % spacebelow = 10pt,
    mdframed={
  backgroundcolor = background,
  linecolor = losning!75,
  linewidth = 1pt,
  skipabove=0pt,
  skipbelow=10pt,
  innerleftmargin=10pt,
  innertopmargin=10pt,
  innerrightmargin=10pt,
  innerbottommargin=10pt,
  leftline = false,
  rightline = true,
  topline = false,
  bottomline = true}
]{solution}

\newenvironment{SimpleBox}[1]{%
  \begin{mdframed}%
    \noindent\textbf{#1}\\[1ex]
}{%
  \end{mdframed}%
}


\begin{document}
\let\cleardoublepage\clearpage
\thispagestyle{fancy}
\chapter{6 - Time Ordered Operators and the Klein-Gordon Equation}

Consider a real scalar field operator, $\hat{\phi}(x, t)$, in one dimension ($x$ is a single number, not a vector!) for simplicity with mass $m$. The Klein-Gordon wave equation is assumed to hold for the field operator as well, and it can be written
$$
\left( \frac{\partial^2}{\partial t^2} - \frac{\partial^2}{\partial x^2} + m^2 \right) \hat{\phi}(x, t) = 0.
$$

\begin{exercise}[1]
Prove that
\begin{align*}
T \left[ \hat{\phi}(x_1, t_1) \hat{\phi}(x_2, t_2) \right] = \theta(t_1 - t_2) \hat{\phi}(x_1, t_1) \hat{\phi}(x_2, t_2) + \theta(t_2 - t_1) \hat{\phi}(x_2, t_2) \hat{\phi}(x_1, t_1),
\end{align*}
where $\theta$ is the Heaviside step function.
\end{exercise}
\begin{solution}
Let's recall the time-ordering operator:
\begin{align*}
T \left[ \hat{\phi}(x_1, t_1) \hat{\phi}(x_2, t_2) \right]
&= \begin{cases}
\hat{\phi}(x_1, t_1) \hat{\phi}(x_2, t_2) & t_1 > t_2 \\
\hat{\phi}(x_2, t_2) \hat{\phi}(x_1, t_1) & t_2 > t_1
\end{cases}
\end{align*}

So it's clearly true!

\end{solution}

\begin{exercise}[2]
Show that
\begin{align*}
\frac{d}{dx} \theta(x - a) = \delta(x - a)
\end{align*}
\end{exercise}
\begin{solution}
Let $x > a$ then
\begin{align*}
\frac{d}{dx} \Theta(x-a)
&= \frac{d}{dx} 1 = 0
\end{align*}
for $x < a$ then
\begin{align*}
\frac{d}{dx} \Theta(x-a)
&= \frac{d}{dx} 0 = 0
\end{align*}
We will try to show that the derivative at $x=a$ is infinity
\begin{align*}
\lim_{\epsilon \to 0} \frac{f(a+\epsilon) - f(a-\epsilon)}{\epsilon}
&= \lim_{\epsilon \to 0} \frac{\Theta(a+\epsilon-a) - \Theta(a-\epsilon-a)}{\epsilon} \\
&= \lim_{\epsilon \to 0} \frac{\Theta(\epsilon) - \Theta(-\epsilon)}{\epsilon} \\
&= \lim_{\epsilon \to 0} \frac{1}{\epsilon} = \infty
\end{align*}
\end{solution}
\begin{exercise}[3]
Use the previous results and the commutator $\left[ \hat{\phi}(x_1, t), \hat{\phi}(x_2, t) \right] = 0$ to show that
\begin{align*}
\frac{\partial}{\partial t_1} \left\{ T \left[ \hat{\phi}(x_1, t_1) \hat{\phi}(x_2, t_2) \right] \right\} = \theta(t_1 - t_2) \left( \frac{\partial \hat{\phi}(x_1, t_1)}{\partial t_1} \right) \hat{\phi}(x_2, t_2) + \theta(t_2 - t_1) \hat{\phi}(x_2, t_2) \left( \frac{\partial \hat{\phi}(x_1, t_1)}{\partial t_1} \right),
\end{align*}
\end{exercise}
\begin{solution}
For any equal times $\left[ \hat{\phi}(x_1, t), \hat{\phi}(x_2, t) \right] = 0$
\begin{align*}
\frac{\partial}{\partial t_1} \left\{ T \left[ \hat{\phi}(x_1, t_1) \hat{\phi}(x_2, t_2) \right] \right\}
&= \frac{\partial}{\partial t_1} \left( \Theta(t_1 - t_2) \hat{\phi}(x_1, t_1) \hat{\phi}(x_2, t_2) + \Theta(t_2 - t_1) \hat{\phi}(x_2, t_2) \hat{\phi}(x_1, t_1) \right) \\
&= \delta(t_1 - t_2) \hat{\phi}(x_1, t_1) \hat{\phi}(x_2, t_2) + \Theta(t_1 - t_2) \frac{\partial \hat{\phi}(x_1, t_1)}{\partial t_1} \hat{\phi}(x_2, t_2) \\
&+ \delta(t_2 - t_1) \hat{\phi}(x_2, t_2) \hat{\phi}(x_1, t_1) + \Theta(t_2 - t_1) \hat{\phi}(x_2, t_2) \frac{\partial \hat{\phi}(x_1, t_1)}{\partial t_1}
\end{align*}
The $\delta$-function is symmetric and the fields commute so we get the desired result.
\end{solution}

\begin{exercise}[4]
Using the related commutator $\left[ \hat{\phi}(x_1, t), \frac{\partial \hat{\phi}(x_2, t)}{\partial t} \right] = i \delta(x_1 - x_2)$ show
\begin{align*}
\frac{\partial^2}{\partial t_1^2} \left\{ T \left[ \hat{\phi}(x_1, t_1) \hat{\phi}(x_2, t_2) \right] \right\} = -i \delta(x_1 - x_2) \delta(t_1 - t_2) + T \left[ \frac{\partial^2 \hat{\phi}(x_1, t_1)}{\partial t_1^2} \hat{\phi}(x_2, t_2) \right],
\end{align*}
\end{exercise}
\begin{solution}
\begin{align*}
\frac{\partial^2}{\partial t_1^2} \left\{ T \left[ \hat{\phi}(x_1, t_1) \hat{\phi}(x_2, t_2) \right] \right\}
&= \frac{\partial}{\partial t_1} \left( \Theta(t_1 - t_2) \frac{\partial \hat{\phi}(x_1, t_1)}{\partial t_1} \hat{\phi}(x_2, t_2) + \Theta(t_2 - t_1) \hat{\phi}(x_2, t_2) \frac{\partial \hat{\phi}(x_1, t_1)}{\partial t_1} \right) \\
&= \delta(t_1 - t_2) \frac{\partial \hat{\phi}(x_1, t_1)}{\partial t_1} \hat{\phi}(x_2, t_2) + \Theta(t_1 - t_2) \frac{\partial^2 \hat{\phi}(x_1, t_1)}{\partial t_1^2} \hat{\phi}(x_2, t_2) \\
&- \delta(t_2 - t_1) \hat{\phi}(x_2, t_2) \frac{\partial \hat{\phi}(x_1, t_1)}{\partial t_1} + \Theta(t_2 - t_1) \hat{\phi}(x_2, t_2) \frac{\partial^2 \hat{\phi}(x_1, t_1)}{\partial t_1^2}
\end{align*}
Now flipping the first term costs a commutator and if we are smart with the sign the terms will cancel. The second part is just the Time-Ordering operator
\begin{align*}
&= -i \delta(x_1 - x_2) \delta(t_1 - t_2) + T \left[ \frac{\partial^2 \hat{\phi}(x_1, t_1)}{\partial t_1^2}, \hat{\phi}(x_2, t_2) \right]
\end{align*}
\end{solution}

\begin{exercise}[5]
Argue that the above demonstrates that the so-called two-point function, $T \left[ \hat{\phi}(x_1, t_1) \hat{\phi}(x_2, t_2) \right]$, obeys
\begin{align*}
\left( \frac{\partial^2}{\partial t_1^2} - \frac{\partial^2}{\partial x_1^2} + m^2 \right) T \left[ \hat{\phi}(x_1, t_1) \hat{\phi}(x_2, t_2) \right] = -i \delta(x_1 - x_2) \delta(t_1 - t_2),
\end{align*}
which implies that the two-point function is in fact the *Green's function* for the Klein-Gordon equation. The derivation here is easily generalized to three dimensions of space.
\end{exercise}
\begin{solution}
Argue that the above demonstrates that the so-called two-point function, $T \left[ \hat{\phi}(x_1, t_1) \hat{\phi}(x_2, t_2) \right]$, obeys
\begin{align*}
\left( \frac{\partial^2}{\partial t_1^2} - \frac{\partial^2}{\partial x_1^2} + m^2 \right) T \left[ \hat{\phi}(x_1, t_1) \hat{\phi}(x_2, t_2) \right] = -i \delta(x_1 - x_2) \delta(t_1 - t_2).
\end{align*}

Let’s calculate:
\begin{align*}
\left( - \frac{\partial^2}{\partial x_1^2} + m^2 \right) T \left[ \hat{\phi}(x_1, t_1) \hat{\phi}(x_2, t_2) \right]
&= \Theta(t_1 - t_2) \left( - \frac{\partial^2}{\partial x_1^2} + m^2 \right) \hat{\phi}(x_1, t_1) \hat{\phi}(x_2, t_2) \\
&\quad + \Theta(t_2 - t_1) \left( - \frac{\partial^2}{\partial x_1^2} + m^2 \right) \hat{\phi}(x_2, t_2) \hat{\phi}(x_1, t_1).
\end{align*}
Now we know that the $\hat{\phi}$'s commute, so we can pull them outside:
\begin{align*}
&= \left( \Theta(t_1 - t_2) + \Theta(t_2 - t_1) \right) \left( - \frac{\partial^2}{\partial x_1^2} + m^2 \right) \hat{\phi}(x_1, t_1) \hat{\phi}(x_2, t_2).
\end{align*}
Rearrange a bit:
\begin{align*}
\left( - \frac{\partial^2}{\partial x_1^2} \hat{\phi}(x_1, t_1) + m^2 \hat{\phi}(x_1, t_1) \right) \left( \Theta(t_1 - t_2) + \Theta(t_2 - t_1) \right) \hat{\phi}(x_2, t_2).
\end{align*}
Let’s include the time derivative:
\begin{align*}
\left( \frac{\partial^2}{\partial t_1^2} - \frac{\partial^2}{\partial x_1^2} + m^2 \right) T \left[ \hat{\phi}(x_1, t_1) \hat{\phi}(x_2, t_2) \right]
&= -i \delta(x_1 - x_2) \delta(t_1 - t_2) + T \left[ \frac{\partial^2 \hat{\phi}(x_1, t_1)}{\partial t_1^2}, \hat{\phi}(x_2, t_2) \right] \\
&\quad + \left( - \frac{\partial^2}{\partial x_1^2} + m^2 \right) \hat{\phi}(x_1, t_1) \left( \Theta(t_1 - t_2) + \Theta(t_2 - t_1) \right) \hat{\phi}(x_2, t_2).
\end{align*}
Expand the time-ordering operator:
\begin{align*}
&= -i \delta(x_1 - x_2) \delta(t_1 - t_2) + \Theta(t_1 - t_2) \frac{\partial^2 \hat{\phi}(x_1, t_1)}{\partial t_1^2} \hat{\phi}(x_2, t_2) \\
&\quad + \Theta(t_2 - t_1) \hat{\phi}(x_2, t_2) \frac{\partial^2 \hat{\phi}(x_1, t_1)}{\partial t_1^2} + \text{ the rest.}
\end{align*}
What we see here is that we may commute the differential terms, and we can then collect everything:
\begin{align*}
&= -i \delta(x_1 - x_2) \delta(t_1 - t_2) + \left( \left( \frac{\partial^2}{\partial t_1^2} - \frac{\partial^2}{\partial x_1^2} + m^2 \right) \hat{\phi}(x_1, t_1) \right) \left( \Theta(t_1 - t_2) + \Theta(t_2 - t_1) \right) \hat{\phi}(x_2, t_2).
\end{align*}
Since $\hat{\phi}$ satisfies the Klein-Gordon equation, the second term is zero. And we get the desired result,
\[
\left( \frac{\partial^2}{\partial t_1^2} - \frac{\partial^2}{\partial x_1^2} + m^2 \right) T\left[ \hat{\phi }\left( x_1, t_1 \right) \hat{\phi }\left( x_2, t_2 \right)  \right] = -i\delta\left( x_1- x_2 \right) \delta\left( t_1-t_2 \right) 
.\] 
\end{solution}
\end{document}
