\documentclass[working, oneside]{../../Preambles/tuftebook}
% Import xcolor and define some colors
\usepackage{{xcolor}}
\definecolor{{background}}{{HTML}}{{{background}}}
\definecolor{{foreground}}{{HTML}}{{{foreground}}}
\definecolor{{math}}{{HTML}}{{{color6}}}

%%%%%%%%%%%%%%%%%%%%%%%%%%%%%%%%%%%%%%%% IMPORTS %%%%%%%%%%%%%%%%%%%%%%%%%%%%%%%%%%%%%%%%
\documentclass[11pt,onesize,a4paper,titlepage]{article}

%%%%%%%%%%%%%%% Formatting %%%%%%%%%%%%%%% 
\usepackage[english]{babel}
\usepackage[utf8]{inputenc}
\usepackage{adjustbox}
\usepackage{geometry} % Margins
\usepackage{sectsty} % Custom Sections

%%%%%%%%%%%%%%% Font %%%%%%%%%%%%%%% 
\usepackage{Archivo}
\usepackage[T1]{fontenc}
\sffamily

%%%%%%%%%%%%%%% Graphics %%%%%%%%%%%%%%% 
\usepackage{fontawesome5} % Icons
\usepackage{graphicx} % Images
\usepackage[most]{tcolorbox} % Color Box
\usepackage{xcolor} % Colors
\usepackage{tikz} % For Drawing Shapes
%%%\usepackage{emoji} % For flags
\tcbuselibrary{breakable}
%%%\usepackage{academicons}

%%%%%%%%%%%%%%% Miscelanous %%%%%%%%%%%%%%% 
\usepackage{lipsum} % Lorem Ipsum
\usepackage{hyperref} % For Hyperlinks

%%%%%%%%%%%%%%% Colors %%%%%%%%%%%%%%% 
\definecolor{title}{HTML}{b5bff5} % Color of the title
\definecolor{bars}{HTML}{889af0} % Color of the title
\definecolor{backdrop}{HTML}{f2f2f2} % Color of the side column
\definecolor{lightgray}{HTML}{dfdfdf} % Color for the skill bars

%%% TU green: #639a00
%%% TU gray: #e6e6e6
%\definecolor{title}{HTML}{639a00} % Color of the title TU
%\definecolor{bars}{HTML}{889af0} % Color of the title TU

% \definecolor{backdrop}{HTML}{f2f2f2} % Color of the side column
\definecolor{backdrop}{HTML}{e6e6e6} % Color of the side column

\definecolor{subtitle}{HTML}{606060} % 


%%%%%%%%%%%%%%% Section Format %%%%%%%%%%%%%%% 
\sectionfont{                     
    \LARGE % Font size
    \sectionrule{0pt}{0pt}{-8pt}{1pt} % Rule under Section name
}

\subsectionfont{
    \Large % Font size
    \fontfamily{phv}\selectfont % Font family
    %\sectionrule{0pt}{0pt}{-8pt}{1pt} % Rule under Subsection name
    \sectionrule{5pt}{0pt}{0pt}{0pt} % Rule under Subsection name
}

%%%%%%%%%%%%%%% Margins and Headers %%%%%%%%%%%%%%%
\geometry{
  a4paper,
  left=7mm,
  right=7mm,
  bottom=10mm,
  top=10mm
}

\pagestyle{empty} % Empty Headers

\usepackage{marvosym}

% \renewcommand\qedsymbol{\CoffeeCup}

\usepackage{changepage}

\newenvironment{subexercise}[1]{%
    \begin{mdframed}[linewidth=0.5pt, linecolor=foreground, backgroundcolor=background, leftmargin=0cm, innerleftmargin=1em, innertopmargin=0pt, innerbottommargin=0pt, innerrightmargin=0pt, topline=false, rightline=false, bottomline=false]
    \par\noindent\textcolor{foreground}{\textbf{#1.}}\hspace{1em}\ignorespaces
}{%
    \par\addvspace{\baselineskip}\end{mdframed}\ignorespacesafterend
}
\newenvironment{solution}{%
    % \par\addvspace{\baselineskip}\noindent\makebox[\textwidth]{\textcolor{foreground}{\textbullet\hspace{1em}\textbullet\hspace{1em}\textbullet}}\par\addvspace{\baselineskip}
    \begin{mdframed}[linewidth=0.5pt, linecolor=foreground, backgroundcolor=background, rightmargin=0cm, innerleftmargin=0cm, innertopmargin=0pt, innerbottommargin=0pt, innerrightmargin=1em, topline=false, leftline=false, bottomline=false]
    \par\noindent\textcolor{foreground}{\textit{Solution.}}\hspace{1em}\ignorespaces
}{%
    \par\addvspace{\baselineskip}\noindent\hfill\textcolor{foreground}{\Coffeecup}\par\addvspace{\baselineskip}\end{mdframed}\ignorespacesafterend
}
% Exercise environment

\declaretheoremstyle[
    name= \textcolor{foreground}{Exercise},
    postheadspace = \newline,
    bodyfont = \normalfont\color{foreground},
    postheadhook={\textcolor{math}{\rule[.4ex]{\linewidth}{0.5pt}}\\},
    % numberwithin=chapter,
    mdframed={
        backgroundcolor = background,
        linecolor = foreground,
        linewidth = 0.5pt,
        rightline =  true,
        topline = true,
        bottomline = true,
        skipabove=20pt,
        skipbelow=20pt,
        innerleftmargin=15pt,
        innertopmargin=10pt,
        innerrightmargin=15pt,
        innerbottommargin=10pt}
    ]{exercise}
\declaretheorem[style=exercise,numbered=no]{exercise}

% \etocsetlevel{exercise}{2}

% \AtEndEnvironment{exercise}{%
%   \etoctoccontentsline{exercise}{\protect\numberline{\theexercise}}%
% }%
% \etocsetstyle{exercise}
% {}
% {}
% % this will be rendered like a non-numbered section, but we could have used
% % \numberline here also
% {\etocsavedsectiontocline{Exercise \etocnumber}{\etocpage}}
%     {}

% theorem environment

\declaretheoremstyle[
    name= \textcolor{foreground}{Theorem},
    postheadspace = \newline,
    bodyfont = \normalfont\color{foreground},
    postheadhook={\textcolor{math}{\rule[.4ex]{\linewidth}{1pt}}\\},
    mdframed={
        backgroundcolor = background,
        linecolor = foreground,
        linewidth = 1pt,
        rightline =  true,
        topline = true,
        bottomline = true,
        skipabove=20pt,
        skipbelow=20pt,
        innerleftmargin=15pt,
        innertopmargin=10pt,
        innerrightmargin=15pt,
        innerbottommargin=10pt}
    ]{theorem}
\declaretheorem[style=theorem,numbered=yes]{theorem}

\declaretheoremstyle[
    name= \textcolor{foreground}{Definition},
    postheadspace = \newline,
    bodyfont = \normalfont\color{foreground},
    postheadhook={\textcolor{math}{\rule[.4ex]{\linewidth}{1pt}}\\},
    mdframed={
        backgroundcolor = background,
        linecolor = foreground,
        linewidth = 1pt,
        rightline =  true,
        topline = true,
        bottomline = true,
        skipabove=20pt,
        skipbelow=20pt,
        innerleftmargin=15pt,
        innertopmargin=10pt,
        innerrightmargin=15pt,
        innerbottommargin=10pt}
    ]{definition}
\declaretheorem[style=definition,numbered=yes]{definition}
% Example environment

\declaretheoremstyle[
name= \quad \underline{Proof:},
     headfont = \bfseries\sffamily,
     postheadspace = \newline,
     % notebraces = \bfseries{(}{)a},
     headpunct = {},
     bodyfont = ,
     postheadhook={\textcolor{foreground}{\rule[0.4ex]{\linewidth}{0pt}}\\},
     qed=\qedsymbol,
    % spacebelow = 10pt,
    mdframed={
  backgroundcolor = background,
  linecolor = foreground,
  linewidth = 1pt,
  skipabove=10pt,
  skipbelow=10pt,
  rightline = false,
  topline = false,
  leftline = false,
  bottomline = false,
  innerleftmargin=15pt,
  innertopmargin=15pt,
  innerrightmargin=15pt,
  innerbottommargin=15pt}
]{pro}
    % \declaretheorem[style=pro,numbered=no]{Proof}

\declaretheoremstyle[
name= \quad \underline{\textcolor{foreground}{Example}},
     headfont = \bfseries\sffamily,
     postheadspace = \newline,
     % notebraces = \bfseries{(}{)a},
     headpunct = {},
     bodyfont = \normalfont\color{foreground},
     postheadhook={\textcolor{foreground}{\rule[0.4ex]{\linewidth}{0pt}}\\},
     % spacebelow = 10pt,
    mdframed={
  backgroundcolor = background,
  linecolor = foreground,
  linewidth = 1pt,
  skipabove=10pt,
  skipbelow=10pt,
  rightline = false,
  topline = false,
  leftline = false,
  bottomline = false,
  innerleftmargin=15pt,
  innertopmargin=15pt,
  innerrightmargin=15pt,
  innerbottommargin=15pt}
]{ex}
\declaretheorem[style=ex,numbered=no]{example}

\declaretheoremstyle[
     name=,
     headfont = \bfseries\sffamily,
     notebraces = \bfseries{},
     headpunct = { -},
     bodyfont = \color{foreground}\normalfont,
     % postheadhook={\textcolor{black}{\rule[.4ex]{\linewidth}{0.2pt}}\\},
    % spacebelow = 10pt,
    mdframed={
  backgroundcolor = background,
  linecolor = foreground,
  linewidth = 1pt,
  skipabove=0pt,
  skipbelow=0pt,
  innerleftmargin=10pt,
  innertopmargin=10pt,
  innerrightmargin=10pt,
  innerbottommargin=10pt,
  rightline = false,
  topline = false,
  leftline = false,
  bottomline = true}
]{subexercise}
% \declaretheorem[style=subexercise,numbered=no]{subexercise}

\declaretheoremstyle[
     name= \color{losning}Løsning,
     headfont = \bfseries\sffamily,
     notebraces = \bfseries{},
     postheadspace = \newline,
     headpunct = {:},
     bodyfont = \normalfont,
     % qed = ,
     % postheadhook={\textcolor{black}{\rule[.4ex]{\linewidth}{0.2pt}}\\},
    % spacebelow = 10pt,
    mdframed={
  backgroundcolor = background,
  linecolor = losning!75,
  linewidth = 1pt,
  skipabove=0pt,
  skipbelow=10pt,
  innerleftmargin=10pt,
  innertopmargin=10pt,
  innerrightmargin=10pt,
  innerbottommargin=10pt,
  leftline = false,
  rightline = true,
  topline = false,
  bottomline = true}
]{solution}

\newenvironment{SimpleBox}[1]{%
  \begin{mdframed}%
    \noindent\textbf{#1}\\[1ex]
}{%
  \end{mdframed}%
}


\begin{document}
\let\cleardoublepage\clearpage
\thispagestyle{fancy}
\chapter{Handin 10 - 10 ECTS}
\textit{We have used Generative Artificial Intelligence tools in doing this
assignment, for the following legitimate use cases only: to get background
information or understand the topic / problem, to improve writing of own text, to
find gaps in our knowledge. The solution of the assignment is entirely our own.}
\begin{exercise}[1]
What happens for the following sequences of events if a validation-based scheduler is used and why?
\begin{enumerate}
    \item[a)] R1(A), R2(B), R1(B), V1, R3(C), R2(C), R3(D), V3, W1(C), W3(D), V2, W2(A).
    \item[b)] R1(A), R1(B), V1, R2(B), R2(C), R3(C), V2, V3, W1(A), W2(C), W3(B).
    \item[c)] R1(A), R2(B), R1(B), R2(C), V1, R3(A), R3(D), V2, W1(A), V3, W3(B), W2(A).
\end{enumerate}
where Ri(X) means that transaction i has X in its read set, Vi means that Ti attempts to validate, and Wi(X) means that Ti enters the write phase, finishes and its write set was X. Ti starts at the first Ri(X) which is performed.
\end{exercise}
We shall do these in order,
\subsection*{a)}
When checking whether transactions will be succesful, it is important to determine which order the transactions validate in (which gives their timestamp) and what their respective read and write sets are. For all of the three schedules, we see that the transactions begin in chronogical order, but they dont validate in the same order, so they will get different time-stamps.
The read and write-sets for the transactions are as follows,
 \begin{table}[htpb]
    \centering
    % \label{takb:label}
    \begin{tabular}{c|c|c}
        & read\_set& write\_set \\ \hline
        T1 & A, B & C \\ \hline
        T2 & B, C & A \\ \hline
        T3 & C, D & D
    \end{tabular}
    \caption*{Read and write-sets for schedule 1.}
\end{table}
Now let's analyse the schedule, and check whether there are any conflicts. The first to attempt to validate is T1, and at this point it is assigned $TS(T_1) = 1$. Since T1 is the youngest, the first one to validate and there aren't any other transactions in either the validation or write phase, its validation is succesful. It then enters the write phase. The next to attempt validation T3, and it is assigned the timestamp $TS(T_3) = 2$. Since $T_1$ is still in its write-phase, we have to check whether there are any conflicts. Specifically we have to check whether any of the following conditions are met,
\begin{SimpleBox}{Validation}
    When attempting to validate $T_i$, then for all other recent $T_j$ in write phase or validation phase with $TS(T_j) < TS(T_i)$, then one of the following three conditions must hold,
    \begin{enumerate}
        \item $T_j$ has finished all three phases before $T_i$ began.
        \item  $T_j$  finished its write phase before $T_i$ begins its write phase, and $T_j$ didn't write to any items read by  $T_i$. i.e.
             \[
            \text{write\_set}(T_j) \cap \text{read\_set}(T_i) = \emptyset
            .\] 
        \item $T_j$ completes its read phase before  $T_i$ begins its read phase and,
             \begin{align*}
                 \text{write\_set}(T_j) \cap \text{read\_set}(T_i) &= \emptyset\\
                 \text{write\_set}(T_j) \cap \text{write\_set}(T_i) &= \emptyset
            .\end{align*}
    \end{enumerate}
\end{SimpleBox}
We see that $T_1$ does finish its write phase before $T_3$ begins, as $W_1\left( C \right) $ happens before $W_3\left( D \right) $, however there is overlap between their read and write sets,
\[
\text{write\_set}\left( T_1 \right) \cap \text{read\_set}\left( T_2 \right) = C  
.\] 
And since this condition is required for the third condition aswell, $T_3$ fails and aborts. The third to attempt validation is $T_{2}$ which is given the timestamp $TS\left( T_2 \right) = 3$. This transaction runs into the same issue.
\[
\text{write\_set}\left( T_1 \right) \cap \text{read\_set}\left( T_3 \right) = C  
.\] 
And is therefore also forced to abort. Since $T_1$ is still running, it will be allowed to write it's changes to $C$ and commit them.
\subsection*{b)}
We start by listing the read and write sets again.
 \begin{table}[htpb]
    \centering
    % \label{takb:label}
    \begin{tabular}{c|c|c}
        & read\_set& write\_set \\ \hline
        T1 & A, B & A \\ \hline
        T2 & B, C & C \\ \hline
        T3 & C & B
    \end{tabular}
    \caption*{Read and write-sets for schedule 2.}
\end{table}
The first to validate is $T_1$, it is given $TS\left( T_1 \right) = 1$ and as before the validation is succesful. Next $T_2$ attempts to validate and it is given $TS\left( T_2 \right) = 2 $. In this case there is no overlap between the write set of $T_1$ and read and write sets of $T_2$,
\begin{align*}
 \text{write\_set}(T_1) \cap \text{read\_set}(T_2) &= \emptyset\\
 \text{write\_set}(T_1) \cap \text{write\_set}(T_2) &= \emptyset
.\end{align*}
And furthermore, $T_1$ completed its read-phase before $T_2$ began its read phase. This implies that condition 3 is met, and $T_2$'s validation is succesful. The last to attempt validation is $T_3$, $TS\left( T_3 \right) = 3$. Since both $T_1$ and $T_2$ are still running, we have to check for conflicts with both of them. However, we see that $T_2$ writes to $C$, which $T_3$ reads,
\[
 \text{write\_set}(T_2) \cap \text{read\_set}(T_3) &= C\\
.\] 
Which yields a conflict, and therefore $T_3$ is forced to abort. Since $T_1$ and $T_2$ were able to validate, they are allowed to write their changes and commit.
\subsection{c)}
We again list the read and write sets,
 \begin{table}[htpb]
    \centering
    % \label{takb:label}
    \begin{tabular}{c|c|c}
        & read\_set& write\_set \\ \hline
        T1 & A, B & A \\ \hline
        T2 & B, C & A\\ \hline
        T3 & A, D & B
    \end{tabular}
    \caption*{Read and write-sets for schedule 3.}
\end{table}
As before $T_1$ validates first and this is allowed. The next to attempt validation is $T_2$, since $T_1$ is still running, we need to check for conflicts. Let's start by checking for overlaps between read and write sets,
\begin{align*}
 \text{write\_set}(T_1) \cap \text{read\_set}(T_2) &= \emptyset\\
 \text{write\_set}(T_1) \cap \text{write\_set}(T_2) &= A
.\end{align*}
Note that, $T_1$ does finish its write phase before $T_2$ begins its write phase, so condition 2 is met, and $T_2$ is allowed to validate. Finally $T_3$ attempts to validate, and as before we have to check for conflicts with $T_1$ and $T_2$. We check for overlaps,
\begin{align*}
 \text{write\_set}(T_1) \cap \text{read\_set}(T_3) &= A\\
 \text{write\_set}(T_1) \cap \text{write\_set}(T_3) &= \emptyset \\
 \text{write\_set}(T_2) \cap \text{read\_set}(T_3) &= A\\
 \text{write\_set}(T_2) \cap \text{write\_set}(T_3) &= \emptyset
.\end{align*}
We see that neither $T_1$ or $T_2$ finish before $T_3$ begins, and the overlaps between write and read sets, imply that $T_3$ will have to abort. As before $T_1$ and $T_2$ are allowed to write and commit their changes.
\begin{exercise}[2]
Given the granularity hierarchy, where T1 has already acquired the locks in red, answer for each of the transactions below if it will be allowed and what locks will need to be acquired for the transaction to proceed. Each transaction starts on the granularity hierarchy as it is in the figure, i.e. T3 does not need to take locks acquired by T2 into account.
\begin{enumerate}
    \item[a)] T2: Update all records on page p1.
    \item[b)] T3: Read everything on page p2 and update record r4.
    \item[c)] T4: Read record r4 and update record r5.
\end{enumerate}
\end{exercise}
\subsection*{a)}
Since $T_2$ is updating $p_1$ it will eventually have to acquire an $X$ lock on $p_1$. According to the multiple granularity locking rules, $T_2$ is then required to have $IX$ or $SIX$ locks on all ancestor nodes. We obtain the following sequence of locks,
\[
SIX\left( db \right) \to SIX\left( f_1 \right) \to X\left( p_1 \right) 
.\] 
This sequence is allowed as $T_1$ only holds $IS$ locks on $db$ and $f_1$, and these do not conflict with $SIX$.
\subsection*{b)}
This will not be allowed, as $T_3$ will have to require an $X$ lock on $r_4$, which is in conflict with the $S$ lock already held by $T_1$.
\subsection*{c)}
Since $T_3$ will be doing both reading and writing on different nodes beneath $p_2$, it essentially has different requirements for locks on the ancestor nodes. In this case the most restrictive requirements of the different operations are the ones that should be applied. Since it will be needing an $X$ lock on $r_5$, it will $SIX$ locks on all the ancestors. These do not conflict with $T_1$'s locks. It will also need to an acquire $S$ lock on r4, which is likewise fine.
\[
SIX\left( db \right) \to SIX\left( f_1 \right) \to SIX\left( p_2 \right) \to S\left( r_4 \right) \to X\left( r_5 \right) 
.\] 
\begin{exercise}[3]
Consider the log below after a crash.
\begin{verbatim}
<START T>
<T, A, 10, 11>
<START U>
<U, B, 20, 21>
<T, C, 30, 31>
<U, D, 40, 41>
<COMMIT U>
\end{verbatim}

\begin{enumerate}
    \item[a)] Using the UNDO/REDO recovery algorithm, identify which transactions must be redone and which must be undone.
    \item[b)] Furthermore, list the actions that must be taken for the recovery manager to bring the database back to a consistent state and why.
\end{enumerate}
\end{exercise}
\subsection*{a)}
When using the UNDO/REDO the key is to use the log to check which transaction where able to finish. These should be redone, as the user has commited and expect their changes to persist. Transaction that were underway, but never got to commit should be undone. So in this case $T$ should be undone and $U$ should be redone.
\subsection*{b)}
The specific actions that should be taken are.
\begin{itemize}
    \item Restore the variables $A, C$ to what they were before $T$ ran.
    \item Set the values of $B, D$ to what $U$ intended.
\end{itemize}
\end{document}
