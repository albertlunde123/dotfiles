\documentclass[working, oneside]{tuftebook}
% Import xcolor and define some colors
\usepackage{{xcolor}}
\definecolor{{background}}{{HTML}}{{{background}}}
\definecolor{{foreground}}{{HTML}}{{{foreground}}}
\definecolor{{math}}{{HTML}}{{{color6}}}

%%%%%%%%%%%%%%%%%%%%%%%%%%%%%%%%%%%%%%%% IMPORTS %%%%%%%%%%%%%%%%%%%%%%%%%%%%%%%%%%%%%%%%
\documentclass[11pt,onesize,a4paper,titlepage]{article}

%%%%%%%%%%%%%%% Formatting %%%%%%%%%%%%%%% 
\usepackage[english]{babel}
\usepackage[utf8]{inputenc}
\usepackage{adjustbox}
\usepackage{geometry} % Margins
\usepackage{sectsty} % Custom Sections

%%%%%%%%%%%%%%% Font %%%%%%%%%%%%%%% 
\usepackage{Archivo}
\usepackage[T1]{fontenc}
\sffamily

%%%%%%%%%%%%%%% Graphics %%%%%%%%%%%%%%% 
\usepackage{fontawesome5} % Icons
\usepackage{graphicx} % Images
\usepackage[most]{tcolorbox} % Color Box
\usepackage{xcolor} % Colors
\usepackage{tikz} % For Drawing Shapes
%%%\usepackage{emoji} % For flags
\tcbuselibrary{breakable}
%%%\usepackage{academicons}

%%%%%%%%%%%%%%% Miscelanous %%%%%%%%%%%%%%% 
\usepackage{lipsum} % Lorem Ipsum
\usepackage{hyperref} % For Hyperlinks

%%%%%%%%%%%%%%% Colors %%%%%%%%%%%%%%% 
\definecolor{title}{HTML}{b5bff5} % Color of the title
\definecolor{bars}{HTML}{889af0} % Color of the title
\definecolor{backdrop}{HTML}{f2f2f2} % Color of the side column
\definecolor{lightgray}{HTML}{dfdfdf} % Color for the skill bars

%%% TU green: #639a00
%%% TU gray: #e6e6e6
%\definecolor{title}{HTML}{639a00} % Color of the title TU
%\definecolor{bars}{HTML}{889af0} % Color of the title TU

% \definecolor{backdrop}{HTML}{f2f2f2} % Color of the side column
\definecolor{backdrop}{HTML}{e6e6e6} % Color of the side column

\definecolor{subtitle}{HTML}{606060} % 


%%%%%%%%%%%%%%% Section Format %%%%%%%%%%%%%%% 
\sectionfont{                     
    \LARGE % Font size
    \sectionrule{0pt}{0pt}{-8pt}{1pt} % Rule under Section name
}

\subsectionfont{
    \Large % Font size
    \fontfamily{phv}\selectfont % Font family
    %\sectionrule{0pt}{0pt}{-8pt}{1pt} % Rule under Subsection name
    \sectionrule{5pt}{0pt}{0pt}{0pt} % Rule under Subsection name
}

%%%%%%%%%%%%%%% Margins and Headers %%%%%%%%%%%%%%%
\geometry{
  a4paper,
  left=7mm,
  right=7mm,
  bottom=10mm,
  top=10mm
}

\pagestyle{empty} % Empty Headers

\usepackage{marvosym}

% \renewcommand\qedsymbol{\CoffeeCup}

\usepackage{changepage}

\newenvironment{subexercise}[1]{%
    \begin{mdframed}[linewidth=0.5pt, linecolor=foreground, backgroundcolor=background, leftmargin=0cm, innerleftmargin=1em, innertopmargin=0pt, innerbottommargin=0pt, innerrightmargin=0pt, topline=false, rightline=false, bottomline=false]
    \par\noindent\textcolor{foreground}{\textbf{#1.}}\hspace{1em}\ignorespaces
}{%
    \par\addvspace{\baselineskip}\end{mdframed}\ignorespacesafterend
}
\newenvironment{solution}{%
    % \par\addvspace{\baselineskip}\noindent\makebox[\textwidth]{\textcolor{foreground}{\textbullet\hspace{1em}\textbullet\hspace{1em}\textbullet}}\par\addvspace{\baselineskip}
    \begin{mdframed}[linewidth=0.5pt, linecolor=foreground, backgroundcolor=background, rightmargin=0cm, innerleftmargin=0cm, innertopmargin=0pt, innerbottommargin=0pt, innerrightmargin=1em, topline=false, leftline=false, bottomline=false]
    \par\noindent\textcolor{foreground}{\textit{Solution.}}\hspace{1em}\ignorespaces
}{%
    \par\addvspace{\baselineskip}\noindent\hfill\textcolor{foreground}{\Coffeecup}\par\addvspace{\baselineskip}\end{mdframed}\ignorespacesafterend
}
% Exercise environment

\declaretheoremstyle[
    name= \textcolor{foreground}{Exercise},
    postheadspace = \newline,
    bodyfont = \normalfont\color{foreground},
    postheadhook={\textcolor{math}{\rule[.4ex]{\linewidth}{0.5pt}}\\},
    % numberwithin=chapter,
    mdframed={
        backgroundcolor = background,
        linecolor = foreground,
        linewidth = 0.5pt,
        rightline =  true,
        topline = true,
        bottomline = true,
        skipabove=20pt,
        skipbelow=20pt,
        innerleftmargin=15pt,
        innertopmargin=10pt,
        innerrightmargin=15pt,
        innerbottommargin=10pt}
    ]{exercise}
\declaretheorem[style=exercise,numbered=no]{exercise}

% \etocsetlevel{exercise}{2}

% \AtEndEnvironment{exercise}{%
%   \etoctoccontentsline{exercise}{\protect\numberline{\theexercise}}%
% }%
% \etocsetstyle{exercise}
% {}
% {}
% % this will be rendered like a non-numbered section, but we could have used
% % \numberline here also
% {\etocsavedsectiontocline{Exercise \etocnumber}{\etocpage}}
%     {}

% theorem environment

\declaretheoremstyle[
    name= \textcolor{foreground}{Theorem},
    postheadspace = \newline,
    bodyfont = \normalfont\color{foreground},
    postheadhook={\textcolor{math}{\rule[.4ex]{\linewidth}{1pt}}\\},
    mdframed={
        backgroundcolor = background,
        linecolor = foreground,
        linewidth = 1pt,
        rightline =  true,
        topline = true,
        bottomline = true,
        skipabove=20pt,
        skipbelow=20pt,
        innerleftmargin=15pt,
        innertopmargin=10pt,
        innerrightmargin=15pt,
        innerbottommargin=10pt}
    ]{theorem}
\declaretheorem[style=theorem,numbered=yes]{theorem}

\declaretheoremstyle[
    name= \textcolor{foreground}{Definition},
    postheadspace = \newline,
    bodyfont = \normalfont\color{foreground},
    postheadhook={\textcolor{math}{\rule[.4ex]{\linewidth}{1pt}}\\},
    mdframed={
        backgroundcolor = background,
        linecolor = foreground,
        linewidth = 1pt,
        rightline =  true,
        topline = true,
        bottomline = true,
        skipabove=20pt,
        skipbelow=20pt,
        innerleftmargin=15pt,
        innertopmargin=10pt,
        innerrightmargin=15pt,
        innerbottommargin=10pt}
    ]{definition}
\declaretheorem[style=definition,numbered=yes]{definition}
% Example environment

\declaretheoremstyle[
name= \quad \underline{Proof:},
     headfont = \bfseries\sffamily,
     postheadspace = \newline,
     % notebraces = \bfseries{(}{)a},
     headpunct = {},
     bodyfont = ,
     postheadhook={\textcolor{foreground}{\rule[0.4ex]{\linewidth}{0pt}}\\},
     qed=\qedsymbol,
    % spacebelow = 10pt,
    mdframed={
  backgroundcolor = background,
  linecolor = foreground,
  linewidth = 1pt,
  skipabove=10pt,
  skipbelow=10pt,
  rightline = false,
  topline = false,
  leftline = false,
  bottomline = false,
  innerleftmargin=15pt,
  innertopmargin=15pt,
  innerrightmargin=15pt,
  innerbottommargin=15pt}
]{pro}
    % \declaretheorem[style=pro,numbered=no]{Proof}

\declaretheoremstyle[
name= \quad \underline{\textcolor{foreground}{Example}},
     headfont = \bfseries\sffamily,
     postheadspace = \newline,
     % notebraces = \bfseries{(}{)a},
     headpunct = {},
     bodyfont = \normalfont\color{foreground},
     postheadhook={\textcolor{foreground}{\rule[0.4ex]{\linewidth}{0pt}}\\},
     % spacebelow = 10pt,
    mdframed={
  backgroundcolor = background,
  linecolor = foreground,
  linewidth = 1pt,
  skipabove=10pt,
  skipbelow=10pt,
  rightline = false,
  topline = false,
  leftline = false,
  bottomline = false,
  innerleftmargin=15pt,
  innertopmargin=15pt,
  innerrightmargin=15pt,
  innerbottommargin=15pt}
]{ex}
\declaretheorem[style=ex,numbered=no]{example}

\declaretheoremstyle[
     name=,
     headfont = \bfseries\sffamily,
     notebraces = \bfseries{},
     headpunct = { -},
     bodyfont = \color{foreground}\normalfont,
     % postheadhook={\textcolor{black}{\rule[.4ex]{\linewidth}{0.2pt}}\\},
    % spacebelow = 10pt,
    mdframed={
  backgroundcolor = background,
  linecolor = foreground,
  linewidth = 1pt,
  skipabove=0pt,
  skipbelow=0pt,
  innerleftmargin=10pt,
  innertopmargin=10pt,
  innerrightmargin=10pt,
  innerbottommargin=10pt,
  rightline = false,
  topline = false,
  leftline = false,
  bottomline = true}
]{subexercise}
% \declaretheorem[style=subexercise,numbered=no]{subexercise}

\declaretheoremstyle[
     name= \color{losning}Løsning,
     headfont = \bfseries\sffamily,
     notebraces = \bfseries{},
     postheadspace = \newline,
     headpunct = {:},
     bodyfont = \normalfont,
     % qed = ,
     % postheadhook={\textcolor{black}{\rule[.4ex]{\linewidth}{0.2pt}}\\},
    % spacebelow = 10pt,
    mdframed={
  backgroundcolor = background,
  linecolor = losning!75,
  linewidth = 1pt,
  skipabove=0pt,
  skipbelow=10pt,
  innerleftmargin=10pt,
  innertopmargin=10pt,
  innerrightmargin=10pt,
  innerbottommargin=10pt,
  leftline = false,
  rightline = true,
  topline = false,
  bottomline = true}
]{solution}

\newenvironment{SimpleBox}[1]{%
  \begin{mdframed}%
    \noindent\textbf{#1}\\[1ex]
}{%
  \end{mdframed}%
}

\renewcommand{\thesection}{\arabic{section}}
\newcommand{\subpoint}[1]{%
    \begin{list}{}{%
        \setlength{\leftmargin}{1em} % Set the indentation width
    }
    \item § \thesubsection \hspace{0.5em} #1
    \end{list}
    \vspace*{-0.6cm}
    \addtocounter{subsection}{1}
}
% \newcommand{\subpoint}[1]{%
%     \hspace{1em} \textbf{§ \thesubsection } #1 \newline
%     \addtocounter{subsection}{1}
% }


\begin{document}
\thispagestyle{fancy}
\chapter{Vedtægter for den almennyttige kulturforening Aarhus Board Forening}
\section{Foreningens navn og hjemsted}
Foreningens navn er "Aarhus Board Forening" og har hjemsted i Aarhus.
\section{Foreningens formål}
BAS NED er en almennyttig kulturforening, der gennem frivillig ulønnet arbejde har til
formål at:
- Være et socialt foretagende, der faciliterer kulturarrangementer til et bredt
publikum
- Sprede kendskabet til basorienteret musik
- Samarbejde med andre kulturforeninger i Danmark, der har lignende formål
BAS NED filosofien:
BAS NED er skabt ud fra et socialt ønske om at kunne være med til at forme en større
inklusion i kulturlivet gennem bas orienteret musik og fællesskabet der opstår deromkring.
Vi tror på solidaritet, kreativitet, tryghed og frihed til selvudfoldelse, og vi tror på at disse
værdier bedst kommer til udtryk i et rum hvor bassen vibrerer, og hvor der er plads til
introvert dans, glæde og sammenhørighed. BAS NED’s filosofi er simpel og ligger i
navnet: slap af og hyg dig, og netop dette frirum er savnet i det Aarhussianske natteliv.
Ordene BAS NED, som er et velkendt slangudtryk i ungdomskulturen, er således en kærlig
opfordring til at tage den med ro, vise omsorg for andre og holde igen med alkohollen. Vi
tror altså på, at man ved at tilbyde kulturdeltagere i Aarhus et andet format end det man er
vant til, kan skabe et stort, sundt og stærkt nyt fællesskab i byen.
\section{Generalforsamlingen}
\subpoint{Generalforsamlingen er foreningens øverste myndighed}
\subpoint{Dagsorden for ordinær generalforsamling skal som minimum indeholde følgende punkter:
\begin{enumerate}
    \item Valg af dirigent
    \item Valg af referent
    \item Valg af kasserer 
    \item Valg af bestyrelsesmedlemmer 
    \item Vedtægtsændringer
    \item Eventuelt 
\end{enumerate}}
\subpoint{Alle beslutninger på en generalforsamling træffes ved kvalificeret stemmeflertal i form af 2/3 af stemmerne.}
\subpoint{Generalforsamlingen er kun beslutningsdygtig, hvis minimum 50\% af foreningens medlemmer er til stede ved generalforsamlingen.}
\subpoint{Alle medlemmer har stemmeret (1 stemme)}
\subpoint{Det er ikke muligt at stemme ved fuldmagt}
\subpoint{Generalforsamlingen ledes af en dirigent, der vælges af forsamlingen.}
\subpoint{Afstemninger skal ske skriftligt såfremt blot ét af de fremmødte medlemmer ytrer ønske herom.}
\subpoint{Hvert aktive medlem har én stemme til hver bestyrelsespost, der er på valg.}
\section{Foreningens bestyrelse}
\subpoint{Foreningens øverst ledende organ er bestyrelsen, som vælges af generalforsamlingen. Bestyrelsen er foreningens kompas, der lægger de store linjer for, hvor vi skal hen som forening og som kulturinstitution. Bestyrelsen har ansvaret for at foreningen holder sig aktuel, tryg og sund. Bestyrelsen har derudover ansvar for at foreningen bevarer en demokratisk, flad struktur.}
\subpoint{Bestyrelsen består af mindst 4 medlemmer og højst 5 medlemmer.}
\subpoint{Foreningens bestyrelse er på valg hvert år.}
\subpoint{Bestyrelsen kan nedsætte arbejdsgrupper mm.}
\subpoint{Bestyrelsen skal som minimum mødes 4 gange årligt, eller såfremt et medlem af bestyrelsen ytrer ønske herom.}
\subpoint{Bestyrelsen vælger selv sine poster til den stiftende generalforsamling}
\subpoint{Udover de 4 valgte til bestyrelsen, kan foreningen stemme 1 eller 2 suppleanter ind, som kan overtage i tilfælde af fratrædelse eller sygdom.}
\section{Medlemskab}
\subpoint{Foreningen optager som medlem enhver, som tilslutter sig foreningens formål.}
\subpoint{Medlemmer forbliver medlemmer indtil de melder sig ud.}
\section{Vedtægtsændringer}
\subpoint{Vedtægtsændringer kræver et flertal på 2/3 af generalforsamlingens fremmødte medlemmer.}
\section{Ekstraordinær generalforsamling}
\subpoint{Indkaldelse sker, hvis et flertal af bestyrelsen ønsker det, eller hvis 1/3 af medlemmerne ønsker det. Indkaldelse med dagsorden sker med mindst to ugers varsel og senest fire uger efter, at der er indgået ønske om det.}
\section{Regnskab og økonomi}
\subpoint{Regnskabsår følger kalenderåret.}
\subpoint{Foreningen hæfter kun for sine forpligtelser med den af foreningen til enhver tid tilhørende formue. Der påhviler ikke foreningens medlemmer eller bestyrelsen nogen personlig hæftelse.}
\section{Opløsning}
\subpoint{Opløsning af foreningen kræver et flertal på 2/3 af generalforsamlingen. Opløsningen skal herefter godkendes på en efterfølgende ekstraordinær generalforsamling.}
\subpoint{Ved opløsning af foreningen skal foreningens midler overdrages til almennyttige, lokale og kulturelle formål. Kulturforeningen BAS NED’s bestyrelse skal beslutte hvilket.}
\section{Arrangementer}
\subpoint{Det påhviler medlemmerne at varetage alle aspekter af foreningens drift. Inklusiv, ledelse og drift, events, arbejdsmiljø, myndighedskrav}
\section{Eksklusion af medlemmer}
\subpoint{Medlemmer der er til skade for foreningen eller forenings ejendele kan ekskluderes af bestyrelsen, såfremt et flertal i bestyrelsen stemmer herfor.}
\section{Vedligeholdelse af væg}
\subpoint{Det er samtlige medlemmers ansvar at BAS NED’s udstyr til enhver tid er funktionelt. Vedtaget på den stiftende generalforsamling d. 17-12-2023.}
\end{document}
